Namen magistrske naloge je bila predstavitev podatkovne strukture kompaktno priponsko drevo ter primerjava le te s podatkovno strukturo priponsko drevo. Obe podatkovni strukturi sta bili primerjani med seboj, bodisi teoretično bodisi empirično.

Pred samo primerjavo obeh podatkovnikih struktur, je bila vsaka podatkovna struktura predstavljena. Predstavitev ni vsebovala samo predstavitev implementacije podatkovne strukture, ampak tudi predstavitev algoritmov za izgradnjo podatkovne strukture. Sama teoretična primerjava se je osredotočila na tri različne primerjave, in sicer na primerjavo osnovnih operacij, primerjavo osnovnih poizvedb ter primerjavo prostorske zahtevnosti in časovne zahtevnosti algoritmov za izgradnjo. Iz primerjave osnovnih operacij in primerjave osnovnih poizvedb je razvidno, da imajo nekatere operacije ter poizvedbe različen čas izvajanja v kompaktnem priponskem drevesu, in sicer se potreben čas pri osnovnih operacijah lahko dvige za $O(t_{SA})$-krat ali $O(t_\Psi)$-krat. Pri poizvedbah se čas dvigne zgolj za $O(t_\Psi)$ pri poizvedbi $prisotnost(vzorec)$, pri ostalih poizvedbah pa se potrebni čas zmanjša za $O(n)$. Časa $O(t_\Psi)$ in $O(t_{SA})$ sta odvisna od implementacije kompaktnega prionskega polja, ki je uporabljeno v kompaktnem priponskem drevesu. Če je kompaktno priponsko drevo implementirano s pomočjo časovno najbolj učinkovite implementacije kompaktnega priponskega polja, potem je čas $t_\Psi=O(1)$ in $t_{SA}=O(\log^\epsilon{n})$, kar pomeni da vse osnovne operacije, ki so bile za $O(t_\Psi)$-krat počasnejše, ohranijo isto časovno zahtevnost, kot jo potrebujejo v priponskem drevesu. 

Glavna razlika med priponskim drevesom in kompaktnim priponskim drevesom, je v časovni zahtevnosti izgradnje ter prostorski zahtevnosti. Priponsko drevo potrebuje $O(n)$ povezav, kar pomeni, da potrebuje $O(n\log{n})$ bitov ali $O(nw)$ bitov, če se uporabljajo sistemski naslovi. Kompaktno priponsko drevo pa potrebuje $|CSA|+6n+o(n)$ bitov, kar v obeh predstavljenih implementacijah kompaktnega priponskega polja ohrani prostorsko zahtevnost $O(n)$ bitov. Časovna zahtevnost izgradnje priponskega drevesa se dvigne iz $O(n)$ za priponska drevesa na $O(n\log{n})$ za kompaktna priponska drevesa, kar pa omogoča, da je celoten postopek izgradnje kompaktnega priponskega drevesa v celoti storjen s kompaktnimi podatkovnimi strukturami.

Z empirično primerjavo so bile potrjene razlike v prostorski zahtevnosti ter v razliki časovnih zahtevnosti operacij, ki so bile predhodno predstavljene. Empirična primerjava je bila opravljena nad zaporedjem genov ter nad besedilom v naravnem jeziku (Slovenščina). Empirična primerjava pa je merila časovno zahtevnost poizvedbe $prisotnost(vzorec)$ nad besedili različnih velikosti ter vzorcev različnih dolžin, časovno zahtevnost izgradnje priponskega drevesa različnih velikosti ter prostorsko zahtevnost le teh priponskih dreves. Z empirično primerjavo se lahko potrdijo teoretične časovne razlike operacij in poizvedb med priponskimi drevesi in kompaktnimi priponskimi drevesi. Pri tem pa se tudi opazi razlika med časovno zahtevnostjo operacij, ko je priponsko drevo v celoti shranjeno na delovnem spominu ter ko je del drevesa shranjen na \verb|Swap| razdelku. Te razlike ni mogoče opaziti na kompaktnih priponskih drevesih, saj dolžina besedila je prekratka, da bi bilo potrebno shraniti kompaktno priponsko drevo na \verb|Swap| razdelku. Pri tem se tudi lahko opazi, da je smiselno uporabljati kompaktna priponska drevesa za besedila, ki imajo dolžino vsaj 8000 znakov, saj takrat kompaktna priponska drevesa potrebujejo manj prostora ter manj časa, da se izgradijo. 

Rezultati testiranja so bili izdelani na računalniku z zgolj 4 GB delovnega pomnilnika in 8 GB \verb|Swap| razdelka, kar je relativno malo spomina, saj imajo novi osebi računalniki vsaj 8 GB delovnega spomina ter dodaten \verb|Swap| razdelek in procesorji za strežnike podpirajo več 100 GB delovnega spomina. Torej se pojavi vprašanje ali so kompaktne podatkovne strukture sploh še potrebne, saj imajo trenutni računalniki na razpolago dovolj delovnega spomina za shraniti celotno priponsko drevo. 
%Prevertvori v trpno obliko.
Po mojem mnenju so še vedno potrebne, saj omogočajo shranjevanje in iskanje po večjem številu priponskih dreves hkrati. Iskanje vzorcev v večjem številu priponskih dreves na enkrat je mogoče, saj večina procesorjev podpira izvajanje večjega števila procesov na enkrat. Računalnik, na katerem je bilo izdelano testiranje, omogoča izvajanje do 4 procesov na enkrat, saj ima 2 jedri in 4 niti.

Nadaljnje raziskave bi se morale osredotočiti na implementacijo kompaktnega priponskega polja, ki zniža obe časovni zahtevnosti $t_{SA}$ in $t_\Psi$ na $O(1)$. Na ta način bi se znižala razlika med kompaktnimi priponskimi drevesi ter priponskimi drevesi, kar bi omogočalo vse prednosti priponskega drevesa ter vse prednosti kompaktnih podatkovnih struktur.