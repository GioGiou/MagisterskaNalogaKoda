Namen magistrske naloge je bil predstavitev podatkovne strukture kompaktno priponsko drevo ter primerjava le-te s priponskim drevesom in priponskim poljem. Podatkovne strukture so bile med seboj empirično primerjane.

Najprej smo vsako podatkovno strukturo predstavili. Predstavitev ni vsebovala samo predstavitev implementacije podatkovne strukture, ampak tudi predstavitev algoritmov za izgradnjo podatkovne strukture ter implementacijo poizvedb nad besedo s strukturo.

Glavna razlika med priponskim drevesom, priponskim poljem in kompaktnim priponskim drevesom je v časovni zahtevnosti izgradnje ter prostorska zahtevnost. Priponsko drevo za besedo $T$ dolžine $n$ potrebuje $O(n)$ povezav, kar pomeni, da potrebuje $O(n\log{n})$ bitov ali $O(nw)$ bitov, če se uporabljajo sistemski naslovi. Podobno priponsko polje, ki mu lahko dodamo $LCP$ polje za pospešiti poizvedbe, potrebuje $O(n)$ celih števil, ki so shranjena z $O(n\log{n})$ biti ali $O(nw)$ biti. Kompaktno priponsko drevo pa potrebuje $|CSA|+6n+o(n)$ bitov, kar v obeh predstavljenih implementacijah kompaktnega priponskega polja ohrani prostorsko zahtevnost $O(n)$ bitov. Časovna zahtevnost izgradnje priponska drevesa in priponska polja potrebuje $O(n)$ časa. Le-ta se poslabša na $O(n\log{n})$ časa za kompaktna priponska drevesa, kar omogoča, da je celoten postopek izgradnje kompaktnega priponskega drevesa v celoti storjen s kompaktnimi podatkovnimi strukturami.

Z empirično primerjavo so bile potrjene razlike v prostorski zahtevnosti ter razlike v časovnih zahtevnosti operacij. Empirična primerjava je bila opravljena nad zaporedjem genov ter nad besedilom v naravnem jeziku (slovenščina). Empirična primerjava je merila časovno zahtevnost poizvedbe $\Prisotnost{T}{P}$ nad besedili različnih velikosti ter vzorcev različnih dolžin. Merila je tudi časovno zahtevnost gradnje podatkovnih struktur različnih velikosti ter prostorsko zahtevnost le-teh. Z empirično primerjavo se lahko potrdijo teoretične časovne razlike gradnje in poizvedb med priponskimi drevesi, priponskimi polji in kompaktnimi priponskimi drevesi. Pri tem pa se tudi opazi razlika med časovno zahtevnostjo gradnje in poizvedb, če je priponsko drevo v celoti shranjeno na delovnem pomnilniku ter če je del drevesa shranjen na Swap razdelku. Te razlike ni mogoče opaziti na priponskih poljih in kompaktnih priponskih drevesih, saj dolžina besedila je prekratka, da bi jih bilo potrebno shraniti na Swap razdelku.

Rezultati testiranja so bili izdelani na računalniku z zgolj 4 GB delovnega pomnilnika in 8 GB Swap razdelka, kar je relativno malo spomina, saj imajo novi osebni računalniki vsaj 16 GB delovnega pomnilnika in procesorji za strežnike podpirajo več 100 GB delovnega pomnilnika. Torej se pojavi vprašanje, ali so kompaktne podatkovne strukture sploh še potrebne, saj imajo trenutni računalniki na razpolago dovolj delovnega pomnilnika za shraniti celotno priponsko drevo. Menimo, da so še vedno potrebne, saj omogočajo shranjevanje in iskanje po večjem številu priponskih dreves hkrati. Iskanje vzorcev v večjem številu priponskih dreves naenkrat je mogoče, saj večina procesorjev podpira izvajanje večjega števila procesov naenkrat. Računalnik, na katerem je bilo izdelano testiranje, omogoča izvajanje do štirih procesov naenkrat, saj ima dve jedri in štiri niti.


V prihodnosti bi bilo zanimivo raziskovati uporabo kompaktnih priponskih dreves za reševanje problema, kot je iskanje v drsečem oknu. Ta problem se do sedaj rešuje s priponskimi drevesi. Zato je potrebno izdelati dinamično verzijo kompaktnih priponskih dreves, ki omogoča dodajanje in brisanje pripon iz drevesa. Druga zanimiva raziskava pa je izdelava kompaktnega priponskega drevesa, ki potrebuje konstantni čas za dostopanje do podnizov besede. To bi omogočalo, da se poizvedba $\SeznamPojavov{T}{P}$ izvede v času $O(m+occ)$, to je čas, ki ga ta poizvedba potrebuje v priponskem drevesu.