Priponsko polje je bilo izgrajeno kot alternativa priponskemu drevesu, ki potrebuje manj prostora. Zato se ga uporablja za indeksiranje besede $T$ in posledično iskanje vzorcev v njej. V tem podpoglavju bodo predstavljene implementacije za priponsko polje istih poizvedb, ki so bile predstavljene za priponsko drevo.

Najbolj osnovna poizvedba, ki bo uporabljena kot osnova za ostali dve poizvedbi, je \textit{prisotnost}$(T,P)$. Prisotnost vzorca $P$ dolžine $m$ v besedi $T$ lahko preverimo z bisekcijo, saj so pripone v priponskem polju urejene. V vsakem koraku bisekcije preverimo, ali se pripona na sredini interval $[L,R]$ (v prvem koraku je $L=1$ in $R=n$) ujema z vzorcem $P$, pri čemer označimo indeks te pripone kot $M$. Če je $T[SA[M],SA[M]+m-1]=P$, potem je vzorec $P$ prisoten v besedi $T$, sicer pa obstaja tak $k$, za katerega velja $P[k]\ne T[SA[M]+k-1]$. V tem primeru obstajata dve možnosti: $P[k]<T[SA[M]+k-1]$ in zato nadaljujemo z iskanjem v intervalu $[L,M]$ ali pa $P[k]>T[SA[M]+k-1]$ in se nadaljuje z iskanjem v intervalu $[M,R]$. Postopek se nadaljuje dokler $R-L>1$, ko je $R-L=1$ in $P[k]\ne T[SA[M]+k]$ potem $P$ in prisoten v besedi $T$. Tako opisan postopek potrebuje $O(m\log{n})$ časa, saj bisekcija potrebuje $O(\log{n})$ primerjav, vsaka primerjava pa potrebuje dodatnih $O(m)$ primerjav, da ugotovimo, ali se pripona in vzorec ujemata.

V predhodnem podpoglavju je bila predstavljena podatkovna struktura $LCP$ polje, ki se jo lahko uporabi za pospešiti iskanje, in sicer implementacija s polji \textit{L-LCP} in \textit{R-LCP}. Poizvedba z $LCP$ polji še vedno temelji na bisekciji, saj je bisekcija najbolj učinkovit način iskanja po urejenem polju. V osnovni različici poizvedbe v vsakem koraku bisekcije preverimo, ali je vzorec manjši, večji ali enak srednji priponi intervala $[L,R]$ z indeksom $M$. Pri tem lahko preštejemo, koliko znakov se ujema med pripono $SA[M]$ in $P$ ter označimo to vrednost s $k$. To znanje lahko uporabimo za znižati število primerjav črk med $P$ in $SA[M]$ na $O(m)$ skozi celotno izvajanje bisekcije. Torej na začetku vsakega koraka vemo, da se $P$ ujema z vsaj $k$ znaki in če smo v levem oziroma desnem podintervalom predhodnega intervala. Recimo, da smo v desnem, torej $L$ je bila predhodna sredinska točka. Vemo, da se $P$ in $SA[L]$-ta pripona ujemata v $k$ znakih ter $SA[L]$-ta pripona in $SA[M]$-ta pripona se ujemata v \textit{L-LCP}$[M]$ in pripona $SA[L]$ je leksikografsko manjša od $SA[M]$. Torej obstajajo tri možnosti:
\begin{enumerate}
    \item $k<\textit{L-LCP}[M]$, potem se $SA[L]$-ta pripona in $SA[M]$-ta pripona bolj ujemata kot $SA[L]$-ta pripona in $P$. Ker smo v predhodnem koraku izvedeli, da je $P$ večji od sredne predhodnega intervala, po tem pomeni, da je $P[k+1]>T[SA[L+k+1]]=T[SA[M+k+1]]$.Torej naslednji pregledani interval je $[M,R]$.
    \item $k>\textit{L-LCP}[M]$, potem se $SA[L]$-ta pripona in $SA[M]$-ta pripona manj ujemata kot $SA[L]$-ta pripona in $P$ in označimo \textit{L-LCP}$[M]=l$. Torej $P[l+1]=T[SA[L+l+1]]<T[SA[M+l+1]]$, saj je $L<M$, ker je $SA[L]$-ta pripona leksikografsko manjša $SA[M]$-ta pripona. Torej se bisekcija nadaljuje na intervalu $[L,M]$.
    \item $k=\textit{L-LCP}[M]$, potem je potrebno preveriti, ali je $P$ večji od $SA[M]$-te pripone. Pri tem ni potrebno preveriti prvih $k$ znakov, saj vemo, da se ujemajo, ker je se $SA[L]$-ta pripona ujema s $SA[M]$-ta pripona v $k$ znakih in se $SA[L]$-ta pripona tudi ujema z $P$ v $k$ znakih. Med preverjanjem štejemo, koliko znakov se ujema in to zabeležimo kot $k'$. Če je $P[k+k'+1]<T[SA[M]+k+k'+1]$ potem nadaljujemo v interval $[L,M]$, sicer je $P[k+k'+1]>T[SA[M]+k+k'+1]$ in nadaljujemo v interval $[M,R]$. Preden nadaljujemo v naslednji interval popravimo $k\leftarrow k+k'$, pri čemer $k\le m$.
\end{enumerate}
Simetrično velja tudi, če smo v levem podintervalu, pri tem pa uporabimo \textit{R-LCP}, saj je $R$ predhodna srednja točka intervala.

Poizvedba \textit{prisotnost}$(T,P)$ potrebuje $O(m+\log{n})$ časa, da preveri prisotnost vzorca v besedi. Pri tem potrebujemo $O(\log{n})$ primerjav, saj se uporablja bisekcijo za učinkovito iskanje po urejenem polju. Skozi celotno izvajanje poizvedbe pa se potrebuje dodatnega $O(m)$ časa za primerjave med znaki vzorca $P$ in zanki pripon na sredini intervalov bisekcije.


Idejo poizvedbe \textit{prisotnost}$(T,P)$ lahko uporabimo za implementacijo poizvedbe \textit{številoPonovitev}$(T,P)$. Poizvedba vrne število $occ$, ki je število ponovite vzorca $P$ v besedi $T$. Naivna implementacija bi bila štetje pripon levo in desno od pripone $SA[M]$, ki je prva pripona v bisekciji, ki se ujema z $P$. Ta postopek potrebuje $O(m + \log{n}+occ)$ časa, pri čemer se lahko uporabi $LCP$ polje za iskanje ponovitev, saj sta vrednosti $LCP[M]$ in $LCP[M+1]$ večji ali enaki $m$ natako tedaj, ko se priponi $SA[M-1]$ in $SA[M+1]$ začneta z vzorcem $P$. Poizvedbo se lahko pohitri na $O(m + \log{n})$ z dvema bisekcijama. Prva bisekcija je potrebna za iskanje začetka intervala $L$ vseh pripon, ki se začnejo z $P$, druga bisekcija pa je potrebna za iskanje konca intervala $R$ vseh takih pripon. Število pripon, ki se začne s $P$, je $occ=R-L+1$. Vsaka bisekcija potrebuje $O(m+\log{n})$ časa, torej tudi poizvedba \textit{številoPonovitev}$(T,P)$ potrebuje $O(m+\log{n})$ časa.

Na podoben način poizvedba \textit{seznamPojavov}$(T,P)$ uporabi interval vseh pripon, ki se začnejo s $P$, $[L,R]$ za izdelati seznam indeksov ponovitev vzorca $P$ v besedi $T$. Za najti interval $[L,R]$ je potrbeno $O(m+\log{n})$ časa, za pretveriti interval v priponskem polju v seznam indeksov pripon pa je potrebnih dodatnih $occ$ dostopov do priponskega polja oziroma $O(occ)$ časa. Torej poizvedba \textit{seznamPojavov}$(T,P)$ potrebuje $O(m+\log{n}+occ)$ časa.


