Priponsko polje je bilo izgrajeno kot alternativa priponskemu drevesu, zato se ga uporablja za indeksiranje besede $T$ in posledično iskanje vzorcev v njej. V tem podpoglavju bodo predstavljene implementacije za priponsko polje istih poizvedb, ki so bile predstavljene za priponsko drevo.

Najbolj osnovna poizvedba, ki bo uporabljena kot osnova za ostali dve poizvedbi, je \textit{prisotnost}$(T,P)$. Prisotnost vzorca $P$ dolžine $m$ v besedi $T$ lahko preverimo z razpolavljanjem, saj so pripone v priponskem polju urejene. V vsakem koraku razpolavljanja preverimo, ali se pripona na sredini interval $[L:R]$ (v prvem koraku je $L=1$ in $R=n$) ujema z vzorcem $P$, pri čemer označimo indeks te pripone kot $M$. Če je $P=T[SA[M]:SA[M]+m-1]$, potem je vzorec $P$ prisoten v besedi $T$, sicer pa obstaja tak $k$, za katerega velja $P[k]\ne T[SA[M]+k-1]$. V tem primeru obstajata dve možnosti: $P[k]<T[SA[M]+k-1]$ in zato nadaljujemo z iskanjem v intervalu $[L:M]$ ali pa $P[k]>T[SA[M]+k-1]$ in se nadaljuje z iskanjem v intervalu $[M:R]$. Postopek se nadaljuje dokler $R-L>1$ oziroma $P=T[SA[M]:SA[M]+m-1]$. Ko je $R-L=1$ in $P[k]\ne T[SA[M]+k]$ potem $P$ ni prisoten v besedi $T$, sicer pa je vzorec prisoten v T, saj je $P=T[SA[M]:SA[M]+m-1]$. Tako opisan postopek potrebuje $O(m\log{n})$ časa, saj razpolavljanje potrebuje $O(\log{n})$ primerjav, vsaka primerjava pa potrebuje dodatnih $O(m)$ primerjav, ali se pripona in vzorec ujemata. Ta način iskanja je $O(\log{n})$-krat počasnejši od iskanja v priponskem drevesu. To razliko si želimo znižati, pri tem pa ne želimo shraniti celotne topologije drevesa, ampak zgolj informacije, ki pospešijo iskanje po priponskem polju. Informacijo, ki jo želimo shraniti, je število znakov na začetku pripon, ki se ujemajo med dvema priponama, ali z drugimi besedami dolžino najdaljše skupne predpone. To informacijo shanimo v polje najdaljših skupnih predpon (angl. \textit{Longest common prefix} oziroma LCP), ki za vsak pari indeksov $i,j$ hrani
    \begin{equation*} 
        \begin{split}
        \textit{LCP}[i,j]|&=\lcp{T[SA[i],n]}{T[SA[j],n]}=\\
         &=\max\{k\in[1,n];T[SA[i],SA[i]+k]=T[SA[j],SA[j]+k]\}.\\
        \end{split}
    \end{equation*}
Tako definirano $LCP$ polje potrebje $O(n^2)$ prostora. Problem, ka ga želimo rešiti z $LCP$ poljem, je ponovnega štetja, znakov, ki vemo, da se ujemajo med pripono $T[SA[M]:]$ in $P$. Le teh je v vsakem koraku $n_s$. S tem vedenjem lahko zmanjšamo število primerjav črk med $P$ in priponami $T[SA[M]:]$ na $O(m)$ skozi celotno izvajanje razpolavljanja. 

\begin{figure}[htb] 
    \begin{subfigure}{0.3\linewidth}   
        \includesvg[scale=0.6]{Slike/LCP/LCPAbstraknoKManjsi.svg}
        \centering
        \vspace{1.3cm}
        \subcaption{$k<l=LCP[L,M]$}
        
        \label{fig:manj}
    \end{subfigure}
    \hfill
    \begin{subfigure}{0.3\linewidth}        
        \includesvg[scale=0.6]{Slike/LCP/LCPAbstraknoKVecji.svg}
        
        \centering
        \vspace{1.3cm}
        \subcaption{$k>l=LCP[L,M]$}
        \label{fig:Vec}
    \end{subfigure}
        \hfill
    \begin{subfigure}{0.3\linewidth}        
        \includesvg[scale=0.6]{Slike/LCP/LCPAbstraknoKEnak.svg}
        \centering
        \vspace{1.3cm}
        \subcaption{$k=l=LCP[L,M]$}
        \label{fig:Enako}
    \end{subfigure}
    \vfill
    \caption{Prikaz razmerja med $P$ in priponsama}
\end{figure}

Na začetku vsakega koraka vemo, da se $P$ ujema z vsaj $n_s$ znaki in če smo v levem oziroma desnem podintervalom predhodnega intervala. Recimo, da smo v desnem, torej $L$ je bila predhodna sredinska točka. Vemo, da se $P$ in $SA[L]$-ta pripona ujemata v $k$ znakih ter $SA[L]$-ta pripona in $SA[M]$-ta pripona se ujemata v $LCP[L,M]$-tih znakih ter pripona $SA[L]$ je leksikografsko manjša od $SA[M]$. Torej obstajajo tri možne relacije med $n_s$ in $LCP[L,M]$:
\begin{enumerate}
    \item $n_s<LCP[L,M]$, potem se $SA[L]$-ta pripona in $SA[M]$-ta pripona bolj ujemata kot $SA[L]$-ta pripona in $P$. Ker smo v predhodnem koraku izvedeli, da je $P$ večji od srednje vrednosti predhodnega intervala, to pomeni, da je $P[n_s+1]>T[SA[L+n_s+1]]=T[SA[M]+n_s+1]$. Torej naslednji pregledani interval je $SA[M:R]$.
    \item $n_s>LCP[L,M]$, potem se $SA[L]$-ta pripona in $SA[M]$-ta pripona manj ujemata kot $SA[L]$-ta pripona in $P$. Označimo \textit{LCP}$[L,M]=l$. Torej $P[l+1]=T[SA[L+l+1]]<T[SA[M+l+1]]$, saj je $L<M$, ker je $SA[L]$-ta pripona leksikografsko manjša od $SA[M]$-te pripone. Torej se bisekcija nadaljuje na intervalu $SA[L:M]$.
    \item $n_s=LCP[L,M]$, potem je potrebno preveriti, ali je $P$ večji od $SA[M]$-te pripone. Pri tem ni potrebno preveriti prvih $n_s$ znakov, saj vemo, da se ujemajo, ker se $SA[L]$-ta pripona ujema s $SA[M]$-ta pripono v $n_s$ znakih in se $SA[L]$-ta pripona tudi ujema s $P$ v $n_s$ znakih. Med preverjanjem štejemo, koliko znakov se ujema in to zabeležimo kot $n_s'$. Če je $P[n_s+n_s'+1]<T[SA[M]+n_s+n_s'+1]$ potem nadaljujemo v interval $SA[L:M]$, sicer je $P[n_s+n_s'+1]>T[SA[M]+k+k'+1]$ in nadaljujemo v interval $SA[M:R]$. Preden nadaljujemo v naslednji interval popravimo $n_s\leftarrow n_s+n_s'$, pri čemer novi $n_s\le m$. Če je $n_s=m$ potem je vzorec $P$ ptisoten v besedi $T$.
\end{enumerate}
Simetrično velja tudi, če smo v levem podintervalu, pri tem pa uporabimo celico $LCP[R,M]$, saj je $R$ predhodna srednja točka intervala. Iz tega sledi naslednja lema.


\begin{lema}\label{lema:LCPKvadrat}
    Poizvedba $\Prisotnost{T}{P}$, implementirana s priponskim poljem in do sedaj predstavljenim $LCP$ poljem, potrebuje $O(m+\log{n})$ časa in $O(n^2 +n)$ prostora.
\end{lema}

%%%%%%%%%%%%%%%%%%%%%%%%

Pri tem se opazi, da velika večina celic $LCP$ polja ne bo nikoli uporabljena pri bisekciji. V vsakem koraku bisekcije se preverja, ali je vzorec večji od sredinske točke $M$ na intervalu $L:R$ indeksov v priponskem polju. Torej za vsak možen interval bisekcije je dovolj, da se hrani dolžina najdaljše predpone med $M$ in $L$ ter med $M$ in $R$. Ker je vsaka pripona srednja točka natanko enega intervala v bisekciji, potem potrebujemo dve \textit{LCP} polji, in sicer prvega za shraniti dolžino najdaljše predpone med $M$ in $L$, ki ga imenujemo \textit{L-LCP}, in drugega za shraniti dolžino najdaljše predpone med $M$ in $R$, ki ga imenujemo \textit{R-LCP}. Primer teh dveh polj je prikazan na Sliki \ref{fig:RlcpLlcpSuffuxArray}, na kateri je prikazano tudi drevo sledi bisekcije. Na vsakem vozlišču drevesa je prikazan tudi interval indeksov priponskega polja $(L,M,R)$, pri čemer $L$ predstavlja začetek intervala, $R$ predstavlja konec intervala indeksov in $M$ predstavlja sredinski indeks, katerega predstavljena pripona bo bila primerjana z vzorcem $P$ \cite{Manber1990}. 

Poizvedbo spremenimo tako, da se polje $LCP$ zamenja z \textit{L-LCP} poljem, ko se primerja priponi $SA[M]$ in $SA[L]$, oziroma z \textit{R-LCP} poljem, ko se primerja priponi $SA[M]$ in $SA[R]$. Iz tega sledi naslednja lema.

\begin{lema}\label{lema:LRLCP}
    Poizvedba $\Prisotnost{T}{P}$, implementirana s priponskim poljem in \textit{L-LCP} in \textit{R-LCP} poljema, potrebuje $O(m+\log{n})$ časa in prostora za $3n$ celih števil.
\end{lema}

\begin{figure}[tb] 
    \includesvg[scale=.8]{Slike/LLCP_RLCP.svg}
    \centering
    \caption{Primer \textit{L-LCP} in \textit{R-LCP} polji za priponskega polja nad besedo \enquote{KOKOŠ$\$$}.} 
    \label{fig:RlcpLlcpSuffuxArray}
\end{figure}

Priponsko polje s to izboljšavo $LCP$ polja potrebuje manj kot tretjino prostora, ki ga potrebuje ekvivalentno priponsko drevo. Pri tem pa poizvedba $\Prisotnost{T}{P}$ potrebuje zgolj $O(\log{n})$ dodatnega časa. Opazimo še, da se za vsak interval uporablja bodisi \textit{L-LCP} polje bodisi \textit{R-LCP} polje, nikoli pa obe polji hkrati. Torej vrednosti, ki bodo uporabljene, se lahko zapiše v \textit{Q-LCP} polje, tako da $\textit{Q-LCP}[M]= \textit{R-LCP}[M]$, če se primerja priponi $SA[M]$ in $SA[R]$, sicer je $\textit{Q-LCP}[M]= \textit{L-LCP}[M]$, ko se primerja priponi $SA[M]$ in $SA[L]$. Na ta način se lahko v poizvedbi zamenja \textit{L-LCP} in \textit{R-LCP} polji z $\textit{Q-LCP}$ poljem. Iz tega sledi izrek.

\begin{izr}\label{izr:LCP}
    Poizvedba $\Prisotnost{T}{P}$, implementirana s priponskim poljem in $\textit{Q-LCP}$ poljem, potrebuje $O(m+\log{n})$ časa in prostora za $2n$ celih števil.
\end{izr}


%%%%%%%%%%%%%%%%%%%%



Idejo poizvedbe $\Prisotnost{T}{P}$ lahko uporabimo za implementacijo poizvedbe \textit{številoPonovitev}$(T,P)$. Poizvedba vrne število $occ$, ki je število ponovite vzorca $P$ v besedi $T$. Naivna implementacija bi bila štetje pripon levo in desno od $M$-te pripone, ki je prva pripona v bisekciji, ki se ujema z $P$. Ta postopek potrebuje $O(m + \log{n}+occ)$ časa. Poizvedbo se lahko pohitri na $O(m + \log{n})$ z dvema bisekcijama. Prva bisekcija je potrebna za iskanje začetka intervala $L$ vseh pripon, ki se začnejo s $P$, druga bisekcija pa je potrebna za iskanje konca intervala $R$ vseh takih pripon. Število pripon, ki se začnjo s $P$, je $occ=R-L+1$. Vsaka bisekcija potrebuje $O(m+\log{n})$ časa, torej tudi poizvedba \textit{številoPonovitev}$(T,P)$ potrebuje $O(m+\log{n})$ časa.

Na podoben način poizvedba \textit{seznamPojavov}$(T,P)$ uporabi interval vseh pripon $SA[L:R]$, ki se začnejo s $P$, za izdelati seznam indeksov ponovitev vzorca $P$ v besedi $T$. Za najti interval $SA[L:R]$ je potrebno $O(m+\log{n})$ časa. Za izdelavo seznama indeksov pripon pa je potrebnih še $occ$ dostopov do priponskega polja oziroma $O(occ)$ časa. Torej poizvedba \textit{seznamPojavov}$(T,P)$ potrebuje $O(m+\log{n}+occ)$ časa.


