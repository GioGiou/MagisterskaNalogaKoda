Priponsko polje je bilo izgrajeno kot alternativa priponskemu drevesu, ki potrebuje manj prostora. Zato se ga uporablja za indeksiranje besede $T$ in posledično iskanje vzorcev v njej. V tem podpoglavju bodo predstavljene implementacije za priponsko polje istih poizvedb, ki so bile predstavljene za priponsko drevo.

Najbolj osnovna poizvedba, ki bo uporabljena kot osnova za ostali dve poizvedbi, je \textit{prisotnost}$(T,P)$. Prisotnost vzorca $P$ dolžine $m$ v besedi $T$ lahko preverimo z razpolavljanjem, saj so pripone v priponskem polju urejene. V vsakem koraku razpolavljanja preverimo, ali se pripona na sredini interval $[L:R]$ (v prvem koraku je $L=1$ in $R=n$) ujema z vzorcem $P$, pri čemer označimo indeks te pripone kot $M$. Če je $T[SA[M]:SA[M]+m-1]=P$, potem je vzorec $P$ prisoten v besedi $T$, sicer pa obstaja tak $k$, za katerega velja $P[k]\ne T[SA[M]+k-1]$. V tem primeru obstajata dve možnosti: $P[k]<T[SA[M]+k-1]$ in zato nadaljujemo z iskanjem v intervalu $[L:M]$ ali pa $P[k]>T[SA[M]+k-1]$ in se nadaljuje z iskanjem v intervalu $[M:R]$. Postopek se nadaljuje dokler $R-L>1$, ko je $R-L=1$ in $P[k]\ne T[SA[M]+k]$ potem $P$ in prisoten v besedi $T$. Tako opisan postopek potrebuje $O(m\log{n})$ časa, saj razpolavljanje potrebuje $O(\log{n})$ primerjav, vsaka primerjava pa potrebuje dodatnih $O(m)$ primerjav, ali se pripona in vzorec ujemata. Ta način iskanja je $O(\log{n})$-krat počasnejši od iskanja v priponskem drevesu. To razliko si želimo znižati, pri tem pa ne želimo shraniti celotne topologije drevesa, ampak zgolj informacije, ki pospešijo iskanje po priponskem polju. Informacijo, ki jo želimo shraniti, je število znakov, ki se ujemajo med dvema priponama, ali z drugimi besedami dolžino najdaljše predpone. To informacijo shanimo v polje najdaljših skupnih predpon (angl. \textit{Longest common prefix} oziroma LCP), ki za vsak pari indeksov $i,j$ hrani
    \begin{equation*} 
        \begin{split}
        \textit{LCP}[i,j]|&=\lcp{T[SA[i],n]}{T[SA[j],n]}=\\
         &=\max_{k}(T[SA[i],SA[i]+k]=T[SA[j],SA[j]+k]).\\
        \end{split}
    \end{equation*}
Tako definirano $LCP$ polje je potrebno $O(n^2)$ prostora. Problem, katerega si želimo znebiti z $LCP$ poljem, je ponovnega štetja, znakov, ki vemo, da se ujemajo med pripono $T[SA[M]:]$ in $P$. Le teh je v vsakem koraku $k$. S tem vedenjem lahko zmanjšamo število primerjav črk med $P$ in priponami $T[SA[M]:]$ na $O(m)$ skozi celotno izvajanje razpolavljanja. 

\begin{figure}[htb] 
    \begin{subfigure}{0.3\linewidth}   
        \includesvg[scale=0.6]{Slike/LCP/LCPAbstraknoKManjsi.svg}
        \centering
        \vspace{1.3cm}
        \subcaption{$k<l=LCP[L,M]$}
        
        \label{fig:manj}
    \end{subfigure}
    \hfill
    \begin{subfigure}{0.3\linewidth}        
        \includesvg[scale=0.6]{Slike/LCP/LCPAbstraknoKVecji.svg}
        
        \centering
        \vspace{1.3cm}
        \subcaption{$k>l=LCP[L,M]$}
        \label{fig:Vec}
    \end{subfigure}
        \hfill
    \begin{subfigure}{0.3\linewidth}        
        \includesvg[scale=0.6]{Slike/LCP/LCPAbstraknoKEnak.svg}
        \centering
        \vspace{1.3cm}
        \subcaption{$k=l=LCP[L,M]$}
        \label{fig:Enako}
    \end{subfigure}
    \vfill
    \caption{Prikaz razmerja med $P$ in priponsama}
\end{figure}

Na začetku vsakega koraka vemo, da se $P$ ujema z vsaj $k$ znaki in če smo v levem oziroma desnem podintervalom predhodnega intervala. Recimo, da smo v desnem, torej $L$ je bila predhodna sredinska točka. Vemo, da se $P$ in $SA[L]$-ta pripona ujemata v $k$ znakih ter $SA[L]$-ta pripona in $SA[M]$-ta pripona se ujemata v $LCP[L,M]$-tih znakih ter pripona $SA[L]$ je leksikografsko manjša od $SA[M]$. Torej obstajajo tri možne relacije med $k$ in $LCP[L,M]$:
\begin{enumerate}
    \item $k<LCP[L,M]$, potem se $SA[L]$-ta pripona in $SA[M]$-ta pripona bolj ujemata kot $SA[L]$-ta pripona in $P$. Ker smo v predhodnem koraku izvedeli, da je $P$ večji od sredne predhodnega intervala, po tem pomeni, da je $P[k+1]>T[SA[L+k+1]]=T[SA[M+k+1]]$. Torej naslednji pregledani interval je $SA[M:R]$.
    \item $k>LCP[L,M]$, potem se $SA[L]$-ta pripona in $SA[M]$-ta pripona manj ujemata kot $SA[L]$-ta pripona in $P$ in označimo \textit{LCP}$[L,M]=l$. Torej $P[l+1]=T[SA[L+l+1]]<T[SA[M+l+1]]$, saj je $L<M$, ker je $SA[L]$-ta pripona leksikografsko manjša $SA[M]$-ta pripona. Torej se bisekcija nadaljuje na intervalu $SA[L:M]$.
    \item $k=LCP[L,M]$, potem je potrebno preveriti, ali je $P$ večji od $SA[M]$-te pripone. Pri tem ni potrebno preveriti prvih $k$ znakov, saj vemo, da se ujemajo, ker je se $SA[L]$-ta pripona ujema s $SA[M]$-ta pripona v $k$ znakih in se $SA[L]$-ta pripona tudi ujema z $P$ v $k$ znakih. Med preverjanjem štejemo, koliko znakov se ujema in to zabeležimo kot $k'$. Če je $P[k+k'+1]<T[SA[M]+k+k'+1]$ potem nadaljujemo v interval $SA[L:M]$, sicer je $P[k+k'+1]>T[SA[M]+k+k'+1]$ in nadaljujemo v interval $SA[M:R]$. Preden nadaljujemo v naslednji interval popravimo $k\leftarrow k+k'$, pri čemer novi $k\le m$.
\end{enumerate}
Simetrično velja tudi, če smo v levem podintervalu, pri tem pa uporabimo celico $LCP[R,M]$, saj je $R$ predhodna srednja točka intervala. Iz tega sledi sledeča lema.


\begin{lema}\label{lema:LCP}
    Poizvedba $\Prisotnost{T}{P}$ implementiran s priponskim poljem in do sedaj predstavljenim $LCP$ poljem potrebuje $O(m+\log{n})$ časa in $O(n^2 +n)$ prostora.
\end{lema}

%%%%%%%%%%%%%%%%%%%%%%%%

Pri tem se opazi, da velika večina celic $LCP$ polja ne bo nikoli uporabljena pri bisekciji. V vsakem koraku bisekcije se preverja, ali je vzorec večji od sredinske točke $M$ na intervalu $L:R$ indeksov v priponskem polju. Torej za vsak možen interval bisekcije je dovolj, da se hrani dolžina najdaljše predpone med $M$ in $L$ ter med $M$ in $R$. Ker je vsaka pripona srednja točka natanko enega intervala v bisekciji, potem potrebujemo dve \textit{LCP} polji, in sicer prvega za shraniti dolžina najdaljše predpone med $M$ in $L$, ki ga imenujemo \textit{L-LCP}, in drugega za shraniti dolžina najdaljše predpone med $M$ in $R$, ki ga imenujemo \textit{R-LCP}. Primer teh dveh polj je prikazan na Sliki \ref{fig:RlcpLlcpSuffuxArray}, na kateri je prikazano tudi drevo sledi bisekcije. Na vsakem vozlišču drevesa je prikazan tudi interval v indeksov priponskega  polja $(L,M,R)$, pričemer $L$ predstavlja začetek intervala, $R$predstavlja konec intervala indeksev in $M$ predstavlja sredinski indeks, katerega predstavljena pripona bo bila primerjana z vzorcem $P$ \cite{Manber1990}. 

Poizvedbo spremenimo tako, da se zamenja $LCP$ polje z \textit{L-LCP} poljem, ko se primerja priponi $SA[M]$ in $SA[L]$, oziroma z \textit{R-LCP} poljem, ko pa se primerja priponi $SA[M]$ in $SA[R]$. Iz tega sledi sledeča lema.

\begin{lema}\label{lema:LRLCP}
    Poizvedba $\Prisotnost{T}{P}$ implementiran s priponskim poljem in \textit{L-LCP} in \textit{R-LCP} polji poljem potrebuje $O(m+\log{n})$ časa in prostora za $3n$ celih števil.
\end{lema}

\begin{figure}[tb] 
    \includesvg[scale=.8]{Slike/LLCP_RLCP.svg}
    \centering
    \caption{Primer \textit{L-LCP} in \textit{R-LCP} polji za priponskega polja nad besedo \enquote{KOKOŠ$\$$}.} 
    \label{fig:RlcpLlcpSuffuxArray}
\end{figure}

Priponsko polje s to izboljšavo $LCP$ polja potrebuje manj kot tretjino prostora, ki ga potrebuje ekvivalentno priponsko drevo. Pri tem pa poizvedba potrebuje $\Prisotnost{T}{P}$ potrebuje zgolj $O(\log{n})$ dodatnega časa. Pri tem se lahko opazi, da se za vsak interval uporablja bodisi \textit{L-LCP} polje bodisi \textit{R-LCP} polje, nikoli pa oba polja hkrati. Torej vrednosti, ki bodo uporabljene, se lahko zapiše v \textit{P-LCP} polje, tako da $\textit{Q-LCP}[M]= \textit{-LCP}[M]$, če se primerja priponi $SA[M]$ in $SA[R]$, sicer je $\textit{Q-LCP}[M]= \textit{L-LCP}[M]$, ko se primerja priponi $SA[M]$ in $SA[R]$. Na ta način se lahko v poizvedbi zamenja \textit{L-LCP} in \textit{R-LCP} polji z $\textit{Q-LCP}$ poljem. Iz tega sledi izrek.

\begin{izr}\label{izr:LCP}
    Poizvedba $\Prisotnost{T}{P}$ implementiran z priponskim poljem in $\textit{Q-LCP}$ poljem potrebuje $O(m+\log{n})$ časa in prostoror za $2n$ celih števil.
\end{izr}


%%%%%%%%%%%%%%%%%%%%



Idejo poizvedbe \textit{prisotnost}$(T,P)$ lahko uporabimo za implementacijo poizvedbe \textit{številoPonovitev}$(T,P)$. Poizvedba vrne število $occ$, ki je število ponovite vzorca $P$ v besedi $T$. Naivna implementacija bi bila štetje pripon levo in desno od $M$-te pripone, ki je prva pripona v bisekciji, ki se ujema z $P$. Ta postopek potrebuje $O(m + \log{n}+occ)$ časa. Poizvedbo se lahko pohitri na $O(m + \log{n})$ z dvema bisekcijama. Prva bisekcija je potrebna za iskanje začetka intervala $L$ vseh pripon, ki se začnejo z $P$, druga bisekcija pa je potrebna za iskanje konca intervala $R$ vseh takih pripon. Število pripon, ki se začne s $P$, je $occ=R-L+1$. Vsaka bisekcija potrebuje $O(m+\log{n})$ časa, torej tudi poizvedba \textit{številoPonovitev}$(T,P)$ potrebuje $O(m+\log{n})$ časa.

Na podoben način poizvedba \textit{seznamPojavov}$(T,P)$ uporabi interval vseh pripon $SA[L:R]$, ki se začnejo s $P$, za izdelati seznam indeksov ponovitev vzorca $P$ v besedi $T$. Za najti interval $SA[S:E]$ je potrebno $O(m+\log{n})$ časa, za pretveriti interval v priponskem polju v seznam indeksov pripon pa je potrebnih dodatnih $occ$ dostopov do priponskega polja oziroma $O(occ)$ časa. Torej poizvedba \textit{seznamPojavov}$(T,P)$ potrebuje $O(m+\log{n}+occ)$ časa.


