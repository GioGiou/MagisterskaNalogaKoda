V priponskem polju so shranjene pripone besede $T$, ki so urejene v leksikografskem vrstnem redu, zato je najbolj učinkovita metoda iskanja bisekcija. Z bisekcijo je potrebnih $O(\log{n})$ primerjav med sredinsko pripono trenutnega intervala, na katerem se izvaja bisekcija, ter vzorcem $P$, torej je potrebno $O(m\log{n})$ časa za preveriti, ali je vzorec $P$ prisoten v besedi $T$. Ta način iskanja je $O(\log{n})$-krat počasnejši od iskanja v priponskem drevesu. To razliko si želimo znižati, pri tem pa ne želimo shraniti celotne topologije drevesa, ampak zgolj informacije, ki pospešijo iskanje po priponskem polju. V tem podpoglavju bo predstavljena predstavljen podatkovna struktura \textit{LCP} polje. V podpoglavju \ref{sec:SAPoizvedbe} pa bo predstavljena uporab te podatkovne strukture za iskanje po priponskem polju.

Polje najdaljših skupnih predpon (angl. \textit{Longest common prifix} oziroma LCP) hrani v vsaki celici 
    $$\textit{LCP}[i][j]=lcp(T[SA[i],n],T[SA[j],n]),$$
kar je dolžina predpone, ki je skupna obema priponama. Za tako definirano $LCP$ polje je potrebno $O(n^2)$ prostora. Pri tem velika večina celic ne bo nikoli uporabljena pri bisekciji. V vsakem koraku bisekcije se preverja, ali je vzorec večji od sredinske točke $M$ na intervalu $[L,R]$. Torej za vsak možen interval bisekcije je dovolj, da se hrani dolžina najdaljše predpone med $M$ in $L$ ter med $M$ in $R$. Ker je vsaka pripona srednja točka natanko enega intervala v bisekciji, potem potrebujemo dve \textit{LCP} polji, in sicer prvega za shraniti dolžina najdaljše predpone med $M$ in $L$, ki ga imenujemo \textit{L-LCP}, in drugega za shraniti dolžina najdaljše predpone med $M$ in $R$, ki ga imenujemo \textit{R-LCP}. Primer teh dveh polj je prikazan na Sliki \ref{fig:RlcpLlcpSuffuxArray}, na kateri je prikazano tudi drevo sredinskih točk bisekcije \cite{Manber1990}. 

\begin{figure}[htb] 
    \includesvg[scale=.8]{Slike/LLCP_RLCP.svg}
    \centering
    \caption{Primer \textit{L-LCP} in \textit{R-LCP} polji za priponskega polja nad besedo \enquote{KOKOŠ$\$$}.} 
    \label{fig:RlcpLlcpSuffuxArray}
\end{figure}

Prostorska zahtevnost te implementacije \textit{LCP} polja je $O(n)$. Če želimo biti bolj natančni \textit{L-LCP} in \textit{R-LCP} polji hranita vsak $n$ \enquote{celih števil}, torej skupaj s priponskim poljem potrebujejo $3n$ \enquote{celih števil}, kaj je še vedno $2$- do $5$-krat manj prostora kot ga potrebuje ekvivalentno priponsko priponsko drevo.
