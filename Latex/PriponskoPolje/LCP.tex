DODATa podatkovna struktura

SA + LCP: simulacija priposnksega drevesa


Osnova za kompaktno priponsko drevo


\cite{Abouelhoda2004}   

\cite{Manber1990}

\subsection{Simulacija priponskega drevesa}\label{sec:STsimulacija}

uporaba za bottom up pregled drevesa. algoritem potrebuje isto časa kot v priponskem drevesu
\cite{Kasai2001}

drevo lcp intervalov 

$l$-iterval 

\begin{defi}
    Interval $[i,j]$ je $l$-interval v LCP polju, natanko tedaj ko velja:
    \begin{enumerate}
        \item LCP$[i]<l$,
        \item obstaja vsaj en $k$, za katerega velja, da je $i< k< j$ in LCP$[k]=l$,
        \item za vsak $k$ velja, da je $i< k< j$ in LCP$[k]\ge l$
        \item LCP$[j]<l$.      
    \end{enumerate}
\end{defi}

\begin{figure}[htb]
    \begin{subfigure}[T]{0.45\linewidth}
        
        \includesvg[scale=1]{Slike/McCreigov/KOKOŠMcCreightS.svg}
        \centering
        \subcaption*{}
        \label{fig:aSADrevo}
    \end{subfigure}
    \begin{subfigure}[T]{0.45\linewidth}        
        \includesvg[scale=0.7]{Slike/IntervalnoDrevo.svg}
        \centering
        \subcaption*{}
        \label{fig:aSAPolje}
    \end{subfigure}
    \caption{Primer intervalnega drevesa nad LCP poljem besede \enquote{KOKOŠ$\$$}.} 
    \label{fig:intervalTree}
\end{figure}

-popravimo LCP tabelo tako da $LCP[1]=-1$ in $LCP[n+1]=-1$

-vsako vozlišče v drevesu je notranje vozlišče priponnskega drevesa

-listi priponskega drevesa so pripone v najglobjem intervalu $[i,j]$, ki vsebuje pripono,vključno z $i$-to pripono in brez $j$-te pripone \cite{Abouelhoda2004}


