Podobno kot priponsko drevo je mogoče zgraditi priponsko polje za besedo $T$ dolžine $n$ z različnimi algoritmi. Časovna zahtevnost teh algoritmov je med $O(n^2\log{n})$ in $O(n)$. Algoritmi bodo predstavljeni od najbolj neučinkovitega do najbolj učinkovitega.

\paragraph{Izgradnja s priponskim drevesom:}
Priponsko polje je ekvivalentno listom priponskega drevesa, zato se lahko za njegovo izgradnjo uporabi priponsko drevo. Pri tem pa ponovno predpostavimo, da so listi v drevesu leksikografsko urejeni. Priponsko polje je zgrajeno v dveh korakih:

\begin{enumerate}
    \item izgradnja priponskega drevesa z $O(n)$ algoritmom, recimo z Ukkonenovim algoritmom (korak ni potreben, če je priponsko drevo že zgrajeno),
    \item sprehod v globino po drevesu in zapis indeksov pripon iz listov v polje v vrstnem redu obiska. %%% POPRAVI
\end{enumerate}
Med sprehodom je mogoče zgraditi tudi $LCP$ polje. Vrednosti v $LCP$ polju so črkovne dolžine vozlišč, ki so najgloblji predhodnik dveh zaporednih listov. V teh vozliščih se obrne smer sprehoda, saj je do takrat sprehod potekal od lista navzgor in se je v vozlišču obrnil ter bo potekal od vozlišča do lista navzdol.

Opisan algoritem zgradi priponsko polje in $LCP$ polje v $O(n)$ časa. Vsak korak potrebuje $O(n)$ časa, saj smo izbrali algoritem za izgradnjo priponskega drevesa s časovno zahtevnostjo $O(n)$ in sprehod po drevesu z $O(n)$ vozlišči potrebuje $O(n)$ časa. Pri tem pa algoritem potrebuje dodatni prostor za največ $10n-7$ števil. Razlog za izgradnjo priponskega polja je zmanjšanje količine spomina, ki ga zasede indeks besede. Zato bi potrebovali algoritem, ki zgradi priponsko polje v $O(n)$ časa in ne zgradi priponskega drevesa.

\paragraph{Izgradnja z urejanjem pripon:}
Priponsko polje je polje indeksov pripon, urejenih v leksikografskem vrstnem redu, zato se lahko uporabi urejanje za njegovo izgradnjo. Najbolj učinkoviti algoritmi za urejanje (\textit{Quck sort} oziroma \textit{Merge sort}) potrebujejo $O(n\log{n})$ primerjav za izgradnjo priponskega polja. Ker so pripone urejene leksikografsko, vsaka primerjava potrebuje $O(n)$ časa, torej celotna izgradnja potrebuje $O(n^2\log{n})$ časa. Pri tem se potrebuje še dodatnega $O(n)$ časa za izgradnjo $LCP$ polja, saj je potrebno izračunati vse $lcp$ vrednosti zaporednih pripon \cite{Kasai2001}.

Namesto tega lahko uporabimo korensko urejanje (angl. \textit{RadixSort}). Torej so v $i$-tem koraku korenskega urejanja pripone urejene v vedrih glede na prvih $i$ znakov. To dejstvo se lahko uporabi za izgradnjo $LCP$ polja, saj se v $i$-tem koraku lahko zapiše vrednost $i$ na začetek vsakega na novo ustvarjenega vedra. Korensko urejanje potrebuje $O(kn)$ časa, pri čemer je $k$ dolžina korena, in v najslabšem primeru je $k=n$, torej metoda potrebuje $O(n^2)$ časa.

Namesto da se v vsakem koraku podaljša koren za en znak, se lahko dolžino korena podvoji. Na ta način potrebuje korensko urejanje $O(\log{n})$ korakov in posledično se potrebuje $O(n\log{n})$ časa za izgradnjo priponskega polja. Pri tem pa vrednost v $LCP$ polju na začetku vsakega na novo ustvarjenega vedra ni enaka številu že opravljenih korakov, ampak je v intervalu med $2^{i-1}$ in $2^{i}$, pri čemer je $i$ število že opravljenih korakov urejanja. Torej je potrebno poiskati natančno vrednost znotraj intervala s primerjavo znakov, pri tem pa ni potrebno primerjati prvih $2^{i-1}$ znakov. Algoritem potrebuje $O(n)$ dodatnega prostora, in sicer 3 polja števil in 2 bitni polji dolžine $n$ \cite{Manber1990}.


Obstaja pa bolj učinkovit algoritem za izgradnjo priponskega polja, ki sta ga predlagala Ko in Aluru \cite{Ko2005} in potrebuje $O(n)$ časa. Ideja algoritma temelji na deljenju pripon na dva tipa: $L$ pripone in $S$ pripone. Pripona $T[i:]$ je $L$ pripona natanko tedaj, ko je $T[i:]$ je manjša od $T[i+1:]$, sicer je $T[i:]$ $S$ pripona, za katero velja, da je $T[i:]$ večja $T[i+1:]$. Pri tem velja tudi, da so $L$ pripone manjše od $S$ pripon, ki se začnejo z istim znakom $c$, torej $L$ pripone nastopijo pred $S$ priponami znotraj intervala pripon z istim začetnim znakom. Algoritem zgradi priponsko polje v štirih korakih:
\begin{enumerate}
    \item deljenje pripon na $S$ in $L$,
    \item ureditev pripon glede na prvi znak pripone,
    \item uredi $S$ pripone ter jih premakni na konec vedra pripon z istim začetnim znakom in premakni konec za eno mesto v levo,
    \item dodatno urejanje $L$ pripon s premikom pripone na začetek intervala, če je pripona, ki je za en znak daljša, tudi $L$ pripona ter se premakne začetek intervala za eno mesto v desno.
\end{enumerate}
Vsak korak je lahko storjen z enim sprehodom po polju oziroma besedi, ki zahteva $O(n)$ časa. Podobno kot pri algoritmu s korenskim urejanjem je prostorska zahtevnost tega algoritma tudi $O(n)$, saj se potrebujejo tri polja števil dolžine $n$ in 3 dodatna bitna polja (dva dolžine $n$ in enega dolžine $n/2$) \cite{Ko2005}.