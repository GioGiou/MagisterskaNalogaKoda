V priponskem drevesu je dolžina najdaljše skupne predpone dveh pripon dolžina podniza najglobljega skupnega predhodnika (angl. \textit{Lowest common ancestor} oziroma LCA) obeh listov, ki predstavljata priponi. Ampak trenutna implementacija \textit{LCP}-polja je namenjena pospešitvi iskanja po priponskem polju in uporablja \textit{L-LCP} in \textit{R-LCP} polji. Zato se potrebuje bolj splošno \textit{LCP}-polje, ki bi nadomestilo \textit{L-LCP} in \textit{R-LCP} polji in bi omogočalo simuliranje priponskega drevesa.

Vrednost v \textit{L-LCP}$[M]$ predstavlja dolžno najdaljše skupne predpone $lcp(T[SA[M], n],$ $T[SA[L], n])$, pri čemer je $L$ začetni indeks intervala bisekcije s sredinsko točko v $M$, in vrednost funkcije $lcp$ označimo kot $k$. Vemo tudi, da je $SA[L]<SA[M]$, torej imajo vse pripone na intervalu med $L$ in $M$ paroma najdaljšo skupno predpono dolžine vsaj $k$, saj vse pripone na tem intervalu so leksikografsko večje od $SA[L]$ in manjše od $SA[M]$. To pomeni, da za vsak $i$, ki je $L<i\le M$, velja $k\le lcp(T[SA[i-1], n], T[SA[i], n])$. Posledično obstaja tak $i$, za katerega velja $lcp(T[SA[i-1], n], T[SA[i], n])=k$. Podoben sklep se lahko naredi tudi za \textit{R-LCP}$[M]$, pri čemer uporabimo interval v priponskem polju med $M$ in $R$. Zato se lahko naredi bolj spološno $LCP$ polje $LCP[2,n]$, za katerega velja
$$
    LCP[i]=lcp(T[SA[i-1], n], T[SA[i], n]).
$$
Primer takega $LCP$ polja je prikazan na Sliki \ref{fig:SuffuxArray}. Na sliki je označena z zeleno barvo vrednost $LCP[3]$, ki je najdaljša skupna predpona med priponama $SA[2]$ in $SA[3]$. Ta vrednost je tudi označena na priponskem drevesu na sliki \cite{Abouelhoda2004, Kasai2001}.

Novo $LCP$ polje potrebuje zgolj $O(n)$ \enquote{celih števil}. Pri tem pa potrebuje dodatno podatkovno strukturo za učinkovito iskanje najmanjše vrednosti na intervalu oziroma $rmq$, ki je potrebna za nadomestiti \textit{L-LCP} in \textit{R-LCP} polji. Prva možnost je izgradnja $rmM$-drevesa, ki v vsakem vozlišču vsebuje zgolj vrednost $m$. Iskanje v drevesu pa potrebuje $O(\log{n})$ časa oziroma $O(m\log{n}+\log{n})$ časa za poizvedbo, kar pa je prepočasno, saj potrebuje iskanje z \textit{L-LCP} in \textit{R-LCP} polji $O(m+\log{n})$ časa za poizvedbo. Zato lahko uporabimo $rmq$ strukturo, ki sta jo predlagala Fischer in Heun \cite{Fischer2007}. Predlagana podatkovna struktura potrebuje $O(n)$ dodatnega prostora in za dani interval vrne najmanjšo vrednost v $O(1)$ času, torej se lahko poizvedba izvrši v $O(m+\log{n})$.

Posplošeno $LCP$ polje se tudi lahko uporabi simuliranje priponskega drevesa. Vrednost $LCP[i]$ predstavlja črkovno dolžino vozlišča $v$, za katerega velja $v=LCA(l_{i-1},l_i)$, pričemer $l_i$ predstavlja pripono $SA[i]$ in $l_{i-1}$ predstavlja pripono $SA[i-1]$. Kasai idr. \cite{Kasai2001} so uporabili to dejstvo za simulacijo pregleda od spodaj navzgor (angl. \textit{Bottom-Up Traversal}) in od desne proti levi (angl. \textit{Post-Order Traversal}). S takim obhodom drevesa lahko rešimo problem sprehoda po podnizih (angl. \textit{substring traversal problem}), ki oštevilči vse ponavljajoče se podnize. Podani algoritem potrebuje $O(n)$ časa in potrebuje enako časa kot obhod priponskega drevesa z $n$-timi listi.

V splošnem vsako notranje vozlišče $v$ v priponskem drevesu predstavlja podniz besede dolžine $l$, s katerim se začnejo vse pripone predstavljene z listi v poddrevesu s korenom v $v$. Ker je priponsko polje ekvivalentno listom priponskega drevesa, so vsi listi poddrevesa s korenom v $v$ ekvivalentni intervalu priponskega polja $SA[L,R]$, pri čemer $SA[L]$-ta pripona je predstavljena s skrajno levim listom in $SA[R]$-ta pripona pa je predstavljena s skrajno desnim listom v poddrevesu. Torej so vse vrednosti intervala $LCP[L+1,R]$ vsaj $l$. Pri tem velja tudi, da pripona $SA[L-1]$ in pripona $SA[R+1]$ se zagotovo razlikujeta s priponami $SA[L]$ in $SA[R]$ vsaj v $l$-tem znaku, sicer bi bile v poddrevesu. Torej velja $LCP[L]< l$ in $LCP[R+1]< l$. Torej lahko imenujem tak interval $[L,R]$ $l$-interval, ki je definiran s sledečo definicijo \cite{Abouelhoda2004}.
\begin{defi}
    Interval $[i,j]$ je $l$-interval v LCP polju, natanko tedaj ko velja:
    \begin{enumerate}
        \item \textit{LCP}$[i]<l$,
        \item obstaja vsaj en $k$, za katerega velja, da je $i< k< j$ in \textit{LCP}$[k]=l$,
        \item za vsak $k$ velja, da je $i< k< j$ in \textit{LCP}$[k]\ge l$
        \item \textit{LCP}$[j+1]<l$.      
    \end{enumerate}
\end{defi}

Koren drevesa predstavlja prazen niz in njegovo poddrevo je celotno drevo, zato koren predstavlja $0$-interval $[1,n]$. Ker pa je $LCP$ polje definirano zgolj za indekse od $2$ do $n$, se lahko definira indeks $1$ kot $LCP[1]=-1$. Ta popravek ne vpliva na pravilnost delovanja priponskega polja, saj ne obstaja pripona $SA[0]$. Znotrja vsakega $l$-intervala $[L,R]$ lahko obstajajo drugi $l'$-intervali, za katere velja $l<l'$. Takih $l'$-intervalov je največ $n_l+1=O(|\Sigma|)$, pri čemer je $n_l$ število ponovitev vrednosti $l$ v $LCP$ polju na intervalu $[L,R]$. Torej lahko predstavimo relacijo med $l$-intervali kot $LCP$ intervalno drevo (angl. \textit{LCP interval tree}). Primer $LCP$ intervalnega drevesa je prikazan na Sliki \ref{fig:intervalTree}. Prikazana je tudi relacija med priponskim drevesom in $LCP$ intervalnim drevesom. \cite{Abouelhoda2004}.

\begin{figure}[tb]
    \begin{subfigure}[T]{0.45\linewidth}        
        \includesvg[scale=1]{Slike/McCreigov/KOKOŠMcCreightS.svg}
        \centering
        \subcaption*{}
        \label{fig:aSADrevo}
    \end{subfigure}
    \hfill
    \begin{subfigure}[T]{0.45\linewidth}        
        \includesvg[scale=0.7]{Slike/IntervalnoDrevo.svg}
        \centering
        \subcaption*{}
        \label{fig:aSAPolje}
    \end{subfigure}
   
    \caption{Primer intervalnega drevesa nad priponskim poljem in LCP poljem besede \enquote{KOKOŠ$\$$} ter njegova predstavitev s tabelo. Na sliki je dodano tudi priponsko drevo besede \enquote{KOKOŠ$\$$}.} 
    \label{fig:intervalTree}
\end{figure}

Za učinkovito simulacijo priponskega drevesa je potrebno učinkovito shraniti $LCP$ intervalno drevo. Najbolj intuitiven način za shraniti topologijo $LCP$ intervalnega drevesa shrani za vsak interval polje $D$ (angl. \textit{down} oziroma dol), ki hrani indekse začetkov podintervalov vsebovanih znotraj intervala. Pri tem je potrebno $O(|\Sigma|n)$ prostora za shraniti vseh $O(n)$ polj $D$. Prostorsko zahtevnost se lahko zniža na $O(n)$ \enquote{celih števil}, tako da vsak interavl hrani vrednost $D$, ki kaže na začetek prvega podintervala, ter verednost $NI$ (angl. \textit{next index} oziroma naslednji indeks ali naslednji brat), ki pa hrani začetni indeks naslednjega podintervala mojega starša oziroma naslednjega brata v $LCP$ intervalnem drevesu. Vsako vozlišče hrani dve \enquote{celi števili}, zato potrebujemo $O(n)$ prostora v $O(\log{n})$ poljih (eno polje za vsako globino). Ker potrebujemo približno $2n$ \enquote{celi števili} se pojavi vprašanje, ali je možno predstaviti ta števila z dvema poljema $D$ in $NI$ dolžine $n$. Problem pri takem zapisu je razločevanje med začetkom intervala starša in začetkom intervala prvega otroka, čessar ni mogoče storiti, torej ni mogoče predstaviti zapisa samo s poljema $D$ in $NI$.

To pa se lahko stori s tremi dodatnimi polji \enquote{celih števil} dolžine $n$, torej potrebujemo $5n$ \enquote{celih števil} za simulirati priponsko drevo z $LCP$ intervali. Pri tem pa se potrebuje med $1,4$- in $2$-krat manj prostora kot ekvivalentno priponsko drevo. V polju $D$ se na mestu $i$ shranjeni začetek drugega intervala (drugega sina) od najdaljšega intervala (najbolj plitvega vozlišča), ki se začne na mestu $i$, oziroma $D[i]$ hrani največji indeks $j$ večji od $i$, za katerega velja $LCP[j]>LCP[i]$ in vse vrednosti v $LCP$ med indeksi $i$ in $j$ so večje od $LCP[j]$. V polju $NI$ na mestu $i$ so shranjeni začetki naslednjega intervala, ki ima istega starša kot interval z začetkom v $i$, oziroma $NI[i]$ hrani prvi indeks $j$ večji od $i$, za katerega velja $LCP[j]=LCP[i]$ in vse vrednosti v $LCP$ med indeksi $i$ in $j$ so večje od $LCP[j]$. S polji $D$ in $NI$ lahko iščemo po intervalih, ki se ne začnejo na istem mestu kot njihovi starši. Če se želimo sprehoditi po prvem podintervalu, pa potrebujemo dodatno polje, ki hrani informacijo o otrocih prvega podintervala. To polje imenujemo $U$ (angl. \textit{up} oziroma gor) in na $i$-tem mestu hrani začetek drugega podinterval predhodnega brata, ki se je končal na mestu $i-1$. Z drugimi besedami $U[i]$ hrani položaj prvega indeksa $j$ manjšega od $i$, za katerega velja $LCP[j]>LCP[i]$ in vse vrednosti v $LCP$ med indeksi $j$ in $i$ so večje ali enake od $LCP[j]$. Torej za interval $[i,j]$ velja $D[i]=U[j+1]$. Začetek intervala drugega otroka v prvem podintevalu intervala $[i,j]$ izračunamo kot $U[D[i]]$. S temi polji se lahko sprehodimo po priponskem polju, kot bi se sprehodili po priponskem drevesu \cite{Abouelhoda2004}. Na Sliki \ref{fig:intervalTree} je prikazan tudi zapis intervalnega drevesa s polji $U$, $D$ in $NI$. 

\paragraph{Poizvedbe z simulacijo:}
Poizvedba se začne z indeksom $i=1$ in $j=n$ in preverili smo prvih $o=0$ znakov. Nato povečamo vrednost $o$ za ena in najdemo interval, ki se začne z vrednostjo $P[o]$, z Algoritmom \ref{alg:interval}. Nato v vsakem koraku iskanja za trenutni interval $[i,j]$ poiščemo vrednost $l=\min\{LCP[k];i<k\le j\}$. Vemo tudi, da se prvih $o$ znakov $P$-ja ujema s priponami na intervalu $[i,j]$. Nato preverimo, ali je $P[o,k]=T[SA[i]+o,SA[i]+k]$, pri čemer je $k=\min\{l,m\}$. Če se ujemata, popravimo $o\leftarrow k$ in začnemo iskati podinterval $[i',j']$, za katerega velja $P[o+1]=T[SA[i']+o+1]$, ki bo postal novi interval $[i,j]$. Če je $o=m$, lahko vrnemo, da je $P$ prisoten v $T$, poročamo, da se $P$ ponovi $(j-i+1)$-krat v $T$, ali pa izgradimo seznam indeksov besede $[SA[i],\dots SA[j]]$, odvisno od poizvedbe. Če pa se $P[o,k]$ ne ujema s $T[SA[i]+c,SA[i]+k]$ pa vrnemo, da $P$ ni prisoten v $T$, 0 ali prazen seznam, odvisno od poizvedbe. Iskanje podintervala je implementirano s polji $D$ in $U$ za najti prvi podinterval ter s poljem $NI$ za najti interval, ki se začne z znakom $P[o+1]$, ter časovna zahtevnost iskanja podintervala je $O(|\Sigma|)=O(1)$, saj se velikost abecede med izvajanjem poizvedbe ne spreminja ter je število podintervalov $O(|\Sigma|)$, ker se vsak podinterval razlikuje od drugih vsaj v znaku na položaju $l+1$. Algoritem za iskanje intervala $[i',j']$ je prikazan na Algoritmu \ref{alg:interval} in ne potrebuje $LCP$ polja, saj smo predhodno že naračunali vrednost $l$, ko se je preverjalo prisotnost podniza vzorca $P[o,k]$. 

\begin{algorithm}[htb]

    \Vhod{Začetek intervala $i$, konec intervala $j$, znak $c$, število pregledanih znakov $l$}
    \Izhod{Interval $[i',j']$}
    \caption{Algoritem za iskanje pod intervala }\label{alg:interval}
    {
        \Ce{$i < U[j+1] \le j $}
                    {$i'\leftarrow U[j+1]$}
        \Sicer{$i'\leftarrow D[i]$}

        \Ce{$T[SA[i]+l+1]==c$}
            {\Vrni{$[i,i'-1]$}}

        \Dokler{$NI[i'] \ne -1$}{

            {$j' \leftarrow NI[i']$}

            \Ce{$T[SA[i']+l+1]==c$}
                {\Vrni{$[i',j'-1]$}}

            {$i' \leftarrow j'$}
        }

        \Ce{$T[SA[i']+l+1]==c$}
                {\Vrni{$[i',j]$}}
        
        {\Vrni{$[-1,-1]$}}    
    }
\end{algorithm}


Torej iskanje s simulacijo priponskega drevesa z uporabo priponskega polja in $LCP$ polja ter $l$-intervalov potrebuje $O(m)$ časa za poizvedbo \textit{prisotnost}$(T,P)$ ter omogoča pospešitev časa poizvedbe \textit{številoPonovitev}$(T,P)$ na $O(m)$. V priponskem drevesu ta poizvedba potrebuje $O(m+occ)$ časa, saj je potrebno prešteti število listov v poddrevesu. Z uporabo simulacije priponskega drevesa tega štetja ni potrebno storiti, saj poznamo velikost intervala, ki ga pokriva vozlišče, ki je koren poddrevesa, čigar pripone se začnejo z iskanim vzorcem. Število pripon je natanko razlika med začetkom in koncem interval, pri čemer konec intervala je začetek naslednjega intervala, ki je shranjen v polju $NI$. Poizvedba \textit{seznamPojavov}$(T,P)$ še vedno potrebuje $O(m+occ)$ časa, ker je potrebno prehoditi priponsko pole.

\paragraph{Opazka:}
Metoda iskanja z $LCP$ intervali se lahko uporabi tudi za iskanje po ostalih kompaktnih številskih drevesih (na primer Patricijinih drevesih). Pri tem se nadomesti priponsko polje s poljem listov $LA$ (angl. \textit{leaf array}), kjer $i$-ta celica polja predstavlja besedo, ki jo predstavlja $i$-ti list številskega drevesa. Pri tem pa ostaja definicija $LCP$ polja podobna, in sicer je definirano tako, da so za vsaki $i$ med 2 in $n$ vrednosti $LCP[i]=lcp(LA[i-1],LA[i])$ in vrednost $LCP[1]=-1$.
