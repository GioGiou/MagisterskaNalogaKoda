V priponskem drevesu je črkovna dolžina najdaljše skupne predpone dveh pripon enaka črkovni dolžini najglobljega skupnega predhodnika (angl. \textit{Lowest Common Ancestor} oziroma LCA) obeh listov, ki predstavljata priponi. Ampak predhodno predstavljena implementacija \textit{LCP} polja je namenjena pospešitvi iskanja po priponskem polju in uporablja \textit{Q-LCP} polje. Zato potrebujemo bolj splošno \textit{LCP} polje, ki bi nadomestilo \textit{Q-LCP} polje in bi omogočalo simuliranje priponskega drevesa.

Če namesto \textit{Q-LCP} polja, vzamemo vrednost v \textit{L-LCP}, in sicer vrednost \textit{L-LCP}$[M]$, ta predstavlja $\lcp{T[SA[M]:]}{T[SA[L]:]}$ dolžino najdaljše skupne predpone, pri čemer je $L$ začetni indeks intervala razpolavljanja s sredinsko točko v $M$. Označimo vrednost funkcije $lcp$ kot $k$. Vemo tudi, da je $SA[L]$ leksikografsko manjši od $SA[M]$, torej imajo vse pripone na intervalu med $L$ in $M$ paroma najdaljšo skupno predpono dolžine vsaj $k$, saj so vse pripone na tem intervalu leksikografsko večje od $SA[L]$ in manjše od $SA[M]$. To pomeni, da za vsak $i$, $L<i\le M$, velja $k\le \lcp{T[SA[i-1]:]}{\; T[SA[i]:]}$. Posledično obstaja tak $i$, za katerega velja $\lcp{T[SA[i-1]:]}{\; T[SA[i]:]}=k$. Podoben sklep se lahko naredi tudi za \textit{R-LCP}$[M]$, pri čemer uporabimo interval v priponskem polju med $M$ in $R$. Zato se lahko naredi bolj splošno $LCP$ polje $LCP[2:n]$, ki ima vrednosti 
$$
    LCP[i]=\lcp{T[SA[i-1]:]}{\; T[SA[i]:]}.
$$
Primer takega $LCP$ polja je prikazan na Sliki \ref{fig:SuffuxArray}. Na sliki je z zeleno barvo označena vrednost $LCP[3]$, ki je najdaljša skupna predpona pripon $SA[2]$ in $SA[3]$. Ta je na sliki označena z zeleno tudi na priponskem drevesu \cite{Abouelhoda2004, Kasai2001}.

Novo $LCP$ polje potrebuje zgolj $n-1$ celih števil. Pri tem pa potrebuje dodatno podatkovno strukturo za učinkovito iskanje najmanjše vrednosti na intervalu oziroma $rmq$, ki je potrebna za nadomestitev \textit{Q-LCP} polja. Prva možnost je izgradnja $rmM$-drevesa, ki v vsakem vozlišču vsebuje zgolj vrednost $m$. Iskanje v drevesu pa potrebuje $O(\log{n})$ časa oziroma $O(m\log{n}+\log{n})$ časa za poizvedbo, kar pa je prepočasno, saj iskanje z \textit{Q-LCP} poljem potrebuje $O(m+\log{n})$ časa za poizvedbo. Zato lahko uporabimo $rmq$ strukturo, ki sta jo predlagala Fischer in Heun \cite{Fischer2007}. Predlagana podatkovna struktura potrebuje $O(n)$ dodatnega prostora in za dani interval vrne najmanjšo vrednost v $O(1)$ času, torej se lahko poizvedba izvrši v $O(m+\log{n})$.

Posplošeno $LCP$ polje se lahko uporabi tudi za simuliranje priponskega drevesa. Vrednost $LCP[i]$ predstavlja črkovno dolžino vozlišča $v$, za katerega velja $v=\Lca{l_{i-1}}{l_i}$, pri čemer $l_i$ predstavlja pripono $SA[i]$ in $l_{i-1}$ predstavlja pripono $SA[i-1]$. Kasai idr. \cite{Kasai2001} so uporabili to dejstvo za simulacijo pregleda od spodaj navzgor (angl. \textit{Bottom-Up Traversal}) in od desne proti levi (angl. \textit{Post-Order Traversal}). S takim obhodom drevesa lahko rešimo problem sprehoda po podnizih (angl. \textit{Substring Traversal Problem}), ki oštevilči vse ponavljajoče se podnize. Podani algoritem potrebuje $O(n)$ časa in potrebuje enako časa kot obhod priponskega drevesa z $n$-timi listi.

V splošnem vsako notranje vozlišče $v$ v priponskem drevesu predstavlja podniz besede dolžine $l$, s katerim se začnejo vse pripone, predstavljene z listi v poddrevesu s korenom v $v$. Ker je priponsko polje ekvivalentno listom priponskega drevesa, so vsi listi poddrevesa s korenom v $v$ ekvivalentni intervalu priponskega polja $SA[L:R]$, pri čemer je $SA[L]$-ta pripona predstavljena s skrajno levim listom, $SA[R]$-ta pripona pa s skrajno desnim listom v poddrevesu. Torej so vse vrednosti intervala $LCP[L+1:R]$ vsaj $l$. Pri tem velja tudi, da se priponi $SA[L-1]$ in $SA[R+1]$ zagotovo razlikujeta od pripon $SA[L]$ in $SA[R]$ vsaj v $l$-tem znaku, sicer bi bile v poddrevesu. Torej velja $LCP[L]< l$ in $LCP[R+1]< l$. Tak interval $[L:R]$ je $l$-interval, ki je definiran z definicijo \ref{def:interval} \cite{Abouelhoda2004}.

\begin{defi}\label{def:interval}
    Interval $[i:j]$ imenujemo $l$-interval v LCP polju, natanko tedaj ko velja:
    \begin{enumerate}
        \item \textit{LCP}$[i]<l$,
        \item obstaja vsaj en $k$, za katerega velja, da je $i< k\le j$ in \textit{LCP}$[k]=l$,
        \item za vsak $k$ velja, da je $i< k\le j$ in \textit{LCP}$[k]\ge l$,
        \item \textit{LCP}$[j+1]<l$.      
    \end{enumerate}
\end{defi}

Ker je $LCP$ polje, definirano zgolj za indekse od $2$ do $n$, in sta v definiciji \ref{def:interval} potrebna tudi indeksa $1$ in $n+1$, se jih definira kot $LCP[1]=LCP[n+1]=-1$. Ta popravek ne vpliva na pravilnost delovanja priponskega polja, saj ne obstajata priponi $SA[0]$ in $SA[n+1]$. 

Koren priponskega drevesa predstavlja prazen niz in njegovo poddrevo je celotno drevo, zato se koren lahko predstavi kot $0$-interval $[1:n]$. Znotraj vsakega $l$-intervala $[L:R]$ lahko obstajajo drugi $l'$-intervali, za katere velja $l<l'$. Takih $l'$-intervalov je največ $n_l+1\le |\Sigma|$, pri čemer je $n_l$ število ponovitev vrednosti $l$ v $LCP$ polju na intervalu $[L+1:R]$. Torej lahko predstavimo relacijo med $l$-intervali kot $LCP$ intervalno drevo (angl. \textit{LCP interval tree}). Primer $LCP$ intervalnega drevesa je prikazan na Sliki \ref{fig:intervalTree}. Na sliki se vidi tudi povezavo med notranjimi vozlišči priponskega drevesa in $LCP$ intervalnim drevesom. Na primer, $2$-interval $[2:3]$ pokriva priponi, ki se začneta na indeksih 1 in 3. V priponskem drevesu pa sta predstavljeni z listi v poddrevesu zelenega vozlišča \cite{Abouelhoda2004}.

\begin{figure}[tb]
    \begin{subfigure}[C]{0.45\linewidth}
        \vfill       
        \includesvg[scale=0.7]{Slike/KOKOŠMcCreightS3.svg}
        \centering
        \vfill
        \subcaption*{}
        \label{fig:aSADrevo}
    \end{subfigure}
    \hfill
    \begin{subfigure}[C]{0.45\linewidth}        
        \includesvg[scale=0.7]{Slike/IntervalnoDrevo.svg}
        \centering
        \subcaption*{}
        \label{fig:aSAPolje}
    \end{subfigure}
   
    \caption{Primer $LCP$ intervalnega drevesa nad priponskim poljem in LCP poljem besede \enquote{KOKOŠ\$} ter njegova predstavitev s tabelo. Na sliki je dodano tudi priponsko drevo besede \enquote{KOKOŠ\$}.} 
    \label{fig:intervalTree}
\end{figure}

\paragraph{Alternativna predstavitev priponskega drevesa.}

Za učinkovito simulacijo operacij nad priponskim drevesom je potrebno učinkovito shraniti $LCP$ intervalno drevo. Ker je $LCP$ intervalno drevo ekvivalentno notranjim vozliščem priponskega drevesa, se lahko zapiše vsako vozlišče $v$ kot četvorko $\left< L, R, l, D \right>$, pri čemer interval $SA[L:R]$ predstavlja vse pripone vozlišča $v$, $l$ predstavlja dolžino niza vozlišča $v$ in polje $D$ (angl. \textit{down} oziroma dol) predstavlja vse otroke vozlišča $v$. Če se vsa vozlišča shrani v polje $V$ dolžine $n_v+n$, so v poljih $D$ shranjeni indeksi v polju $V$. Na ta način potrebujemo $3(n_v+n)$ števil za predstavitev vseh $R$-jev, $L$-jev in $l$-jev ter dodatnih $(n_v+n-1)$ indeksov za vozlišča shranjenih v poljih $D$. Skupaj potrebujemo $4(n_v+n)-1$ števil ter $n$ števil za priponsko polje, torej potrebujemo največ $9n-5$ števil oziroma $n$ manj števil kot za originalno implementacijo priponskega drevesa. 

Polje $D$ se lahko nadomesti z indeksoma $D$ in $NI$ (angl. \textit{next index} oziroma naslednji indeks ali naslednji brat), torej je vsako vozlišče predstavljeno kot peterka $\left< L, R, l, D, NI \right>$, pri čemer $D$ hrani indeks prvega sina v polju $V$, $NI$ pa hrani indeks naslednjega brata vozlišča $v$ v polju $V$. Če prvi sin oziroma naslednji brat vozlišča ne obstajata se v $D$ oziroma $NI$ shrani vrednost $-1$. Tako shranjena vozlišča potrebujejo $5(n_v+n)$ števil ter $n$ števil priponskega polja oziroma največ $11n-5$ števil.

Opazi se lahko, da je $L$ od naslednjega brata od $v$ enak $R+1$ od $v$ oziroma ko je $NI=-1$ potem je vrednosti $R$ od starša enaka vrednosti od $v$. Torej vrednosti $R$ ni potrebno hraniti, saj se jo lahko izračuna. Vrednost $l$ se lahko izračuna s pomočjo $LCP$ polja, kot
$$l=\min_{L<i\le R}LCP[i],$$
torej se lahko $l$ nadomesti z $LCP$ poljem. Zato si želimo shraniti tudi vrednost $D$ in $NI$ v polji dolžine $n$. Pri tem se pojavi problem, da je vrednost $L$ od prvega otroka vozlišča $v$ enaka vrednosti $L$ vozlišča $v$. Problem se lahko reši tako, da $D$ kaže na drugega otroka vozlišča $v$. Na ta način pa je potrebno še zagotoviti možnost sprehajanja po prvem otroku. To se stori tako, da vozlišče postane četvorka $\left< L, D, NI, U \right>$, pri čemer $U$ (angl. \textit{up} oziroma gor) kaže na drugega sina predhodnega brata. Tako predstavljena vozlišča potrebujejo $4(n_v+n)$ ter dodatnih $2n$ števil za priponsko polje in $LCP$ polje, oziroma $10n-4$ števil.

%Najbolj intuitiven način za shraniti topologijo $LCP$ intervalnega drevesa shrani za vsak interval polje $D$ (angl. \textit{down} oziroma dol), ki hrani indekse začetkov podintervalov vsebovanih znotraj intervala. Pri tem je potrebno $O(|\Sigma|n)$ prostora za shraniti vseh $O(n)$ polj $D$. Prostorsko zahtevnost se lahko zniža na $O(n)$ celih števil, tako da vsak interval hrani vrednost $D$, ki kaže na začetek prvega podintervala, ter vrednost $NI$ (angl. \textit{next index} oziroma naslednji indeks ali naslednji brat), ki pa hrani začetni indeks naslednjega podintervala mojega starša oziroma naslednjega brata v $LCP$ intervalnem drevesu. Vsako vozlišče hrani dve celi števili, zato potrebujemo $O(n)$ prostora v $O(\log{n})$ poljih (eno polje za vsako globino). Ker potrebujemo približno $2n$ celi števili se pojavi vprašanje, ali je možno predstaviti ta števila z dvema poljema $D$ in $NI$ dolžine $n$. Problem pri takem zapisu je razločevanje med začetkom intervala starša in začetkom intervala prvega otroka, česar ni mogoče storiti, torej ni mogoče predstaviti zapisa samo s poljema $D$ in $NI$.

Na ta način lahko vrednosti v $D$, $NI$ in $U$ spremenimo v polja dolžine $n$ in jih poravnamo z začetki intervala $L$. Vrednost $D[i]$ hrani največji indeks $j$ večji od $i$, za katerega velja $LCP[j]>LCP[i]$ in vse vrednosti v $LCP$ med indeksoma $i$ in $j$ so večje od $LCP[j]$. Vrednost $NI[i]$ hrani prvi indeks $j$ večji od $i$, za katerega velja $LCP[j]=LCP[i]$ in vse vrednosti v $LCP$ med indeksoma $i$ in $j$ so večje od $LCP[j]$. Vrednost $U[i]$ pa hrani položaj prvega indeksa $j$ manjšega od $i$, za katerega velja $LCP[j]>LCP[i]$ in vse vrednosti v $LCP$ med indeksoma $j$ in $i$ so večje ali enake od $LCP[j]$. Torej za interval $SA[i:j]$ velja $D[i]=U[j+1]$. Začetek intervala drugega otroka v prvem podintevalu intervala $SA[i:j]$ izračunamo kot $U[D[i]]$. S temi polji se lahko sprehodimo po priponskem polju, kot bi se sprehodili po priponskem drevesu, pri tem pa potrebujemo $5n$ števil za shraniti vse te vrednosti \cite{Abouelhoda2004}. Na Sliki \ref{fig:intervalTree} je prikazan tudi zapis intervalnega drevesa s polji $U$, $D$ in $NI$. 

\paragraph{Poizvedbe.}
Poizvedba se začne z indeksom $s=1$ in $e=n$ in preverili smo prvih $o=0$ znakov vzorca $P$. Nato povečamo vrednost $o$ za ena in najdemo interval, ki se začne z vrednostjo $P[o]$, z Algoritmom \ref{alg:interval}. Nato v vsakem koraku iskanja za trenutni interval $[s:e]$ poiščemo vrednost $l=\min_{s<k\le e}LCP[k]$. Vemo tudi, da se prvih $o$ znakov $P$-ja ujema s priponami na intervalu $[s:e]$. Nato preverimo, ali je $P[o:k]=T[SA[s]+o:\; SA[s]+k]$, pri čemer je $k=\min\{l,m\}$. Če se ujemata, popravimo $o\leftarrow k$ in začnemo iskati podinterval $[s':e']$, za katerega velja $P[o+1]=T[SA[s']+o+1]$, ki bo postal novi interval $[s:e]$. Če je $o=m$, lahko vrnemo, da je $P$ prisoten v $T$, poročamo, da se $P$ ponovi $(e-s+1)$-krat v $T$, ali pa izgradimo seznam indeksov besede $[SA[s],\dots,SA[e]]$, odvisno od poizvedbe. Če pa se $P[o:k]$ ne ujema s $T[SA[s]+o:SA[s]+k]$, pa vrnemo, da $P$ ni prisoten v $T$, 0 ali prazen seznam, odvisno od poizvedbe. Iskanje podintervala je implementirano s polji $D$ in $U$ za najti prvi podinterval ter s poljem $NI$ za najti interval, ki se začne z znakom $P[o+1]$. Časovna zahtevnost iskanja podintervala je $O(|\Sigma|)=O(1)$, saj se velikost abecede med izvajanjem poizvedbe ne spreminja ter je število podintervalov $O(|\Sigma|)$, ker se vsak podinterval razlikuje od drugih vsaj v znaku na položaju $l+1$. Algoritem za iskanje intervala $[s':e']$ prikazuje Algoritem \ref{alg:interval} in ne potrebuje $LCP$ polja, saj smo predhodno že izračunali vrednost $l$, ko se je preverjalo prisotnost podniza vzorca $P[o:k]$. 

\begin{algorithm}[htb]

    \Vhod{Začetek intervala $s$, konec intervala $e$, znak $c$, število pregledanih znakov $l$}
    \Izhod{Interval $[i',j']$}
    \caption{Algoritem za iskanje podintervala }\label{alg:interval}
    {
        \Ce{$s < U[e+1] \le e $}
                    {$s'\leftarrow U[e+1]$}
        \Sicer{$s'\leftarrow D[s]$}

        \Ce{$T[SA[s]+l+1]==c$}
            {\Vrni{$[s,s'-1]$}}

        \Dokler{$NI[s'] \ne -1$}{

            {$e' \leftarrow NI[s']$}

            \Ce{$T[SA[s']+l+1]==c$}
                {\Vrni{$[s',e'-1]$}}

            {$s' \leftarrow e'$}
        }

        \Ce{$T[SA[s']+l+1]==c$}
                {\Vrni{$[s',e]$}}
        
        {\Vrni{$[-1,-1]$}}    
    }
\end{algorithm}


Torej iskanje s simulacijo operacij nad priponskim drevesom z uporabo priponskega polja in $LCP$ polja ter $l$-intervalov potrebuje $O(m)$ časa za poizvedbo \textit{prisotnost}$(T,P)$ ter omogoča pospešitev časa poizvedbe \textit{številoPonovitev}$(T,P)$ na $O(m)$. V priponskem drevesu ta poizvedba potrebuje $O(m+occ)$ časa, saj je potrebno prešteti število listov v poddrevesu. Z uporabo simulacije priponskega drevesa tega štetja ni potrebno storiti, saj poznamo velikost intervala, ki ga pokriva vozlišče, ki je koren poddrevesa, čigar pripone se začnejo z iskanim vzorcem. Število pripon je natanko razlika med začetkom in koncem intervala, pri čemer je konec intervala začetek naslednjega intervala, ki je shranjen v polju $NI$. Poizvedba \textit{seznamPojavov}$(T,P)$ še vedno potrebuje $O(m+occ)$ časa, ker je potrebno prehoditi priponsko polje.

\paragraph{Opomba.}
Metoda iskanja z $LCP$ intervali se lahko uporabi tudi za iskanje po ostalih kompaktnih številskih drevesih (na primer Patricijinih drevesih). Pri tem se nadomesti priponsko polje s poljem listov $LA$ (angl. \textit{leaf array}), kjer $i$-ta celica polja predstavlja besedo, ki jo predstavlja $i$-ti list številskega drevesa. Pri tem pa ostaja definicija $LCP$ polja podobna, in sicer je definirano tako, da so za vsak $i$ med 2 in $n$ vrednosti $LCP[i]=lcp(LA[i-1],LA[i])$ in vrednost $LCP[1]=-1$.
