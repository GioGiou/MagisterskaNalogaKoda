Kompaktno priponsko drevo je možno izgradi iz priponskega drevesa. Tak način izgradnje kompaktnega priponskega drevesa ni prostorsko učinkovit, saj je potrebno prvo izgraditi celotno priponsko drevo. Zato bi bilo bolje izgraditi kompaktno priponsko drevo  neposredno iz besedila, kot je to storjeno za priponsko drevo.

Kompaktno priponsko drevo je mogoče izgradi neposredno iz vhodnega besedila $T$. Pri tem so potrebne dodatne podatkovne strukture za izgradnjo posamičnih komponent. Tudi te dodatne podatkovne strukture so kompaktne podatkovne strukture. Izgradnja kompaktnega prionskega drevesa poteka v sledečih treh korakih:
\begin{enumerate}
    \item izgradnja kompaktnega priponskega polja $SA$ iz besedila $T$,
    \item izgradnja polja $LCP$ iz kompaktnega priponskega polja $SA$,
    \item izgradnja topologije drevesa $\tau$ iz $LCP$ polja.
\end{enumerate}

Prva zgrajena podatkovna struktura je kompaktno priponsko polje $SA$. Ker je priponsko polje implementirano s pomočjo funkcije $\Psi$ (shranjena v istoimenskem polju), je potrebno naračunati polje $\Psi$. To je storjeno s pomočjo Burrows–Wheelerjeve preslikave (angl. \textit{Burrows–Wheeler Transform} oziroma BWT), ki je izdelana s pomočjo priponskega polja in besedila. Preslikavo se gradi iz leve proti desni, torej se lahko $BWT$ izgradi postopoma z deli priponskega polja dolžine $b$, saj za $b=n/\log{n}$ je potrebnih $o(n)$ bitov dodatnega prostora in $O(n\log{n})$ časa \cite{Navarro2016}.

Priponsko polje $SA$ se ne razdeli na $b$ delov, ampak je lažje najti najmanjše število predpon, ki se pojavijo na začetku največ $b$-tim priponam, ter se na ta način razdeli priponsko polje $SA$. Sledi izračun pripon, ki sodijo v določeno vedro dolžine $b$ priponskega polja $SA$. Pripone v tem vedru se leksikografsko uredijo, najpogosteje z uporabo korenskega urejanja (angl. \textit{RadixSort}). Vrednost $BWT$ se naračuna s pregledom vseh pripon v vedru. Pripone se shrani v polje $L[1,n]$, kjer je $L[kb+j]=T[A[j]-1]$, pri čemer $k$ predstavlja zaporedno število vedra in $T[0]=\$$ \cite{Navarro2016}.

Ko je polje $L$ izračunano, se kompaktno predstavitev polja $\Psi$ zapiše kot bitna polja $B_c$ za vsak znak $c\in \Sigma$. Vsako bitno polje $B_c$ se inicializira z ničlami. Če je $L[i]=c$ potem se nastavi $B_c[i]=1$. To je lahko storjeno brez dodatnega prostora za polje $L$, saj  se ne zapiše črke $c=T[A[j]-1]$ v polju $L$, ampak se jo lahko neposredno zapiše v bitno polje $B_c[kb+j]=1$ \cite{Navarro2016}.

Čas potreben za izgradnjo kompaktnega priponskega polja $SA$ je $O(n\log{n})$. Večina operacij potrebuje $O(n)$ časa. Iskanje pripon, ki spadajo v določeno vedro, potrebuje $O(n)$ časa za vsako vedro, torej potrebuje $n/b\cdot O(n)=(n\log{n})/n\cdot O(n)=O(n\log{n})$ časa za naračunat vse pripone. 

Bitno polje $H$, ki predstavlja $LCP$ polje, je naslednja izgrajena podatkovna struktura. Ker je $H[j]=1$ za $j=PLCP[i]+2i$, se $H$ izračuna direktno iz polja $PLCP$.

Vrednosti v polju $PLCP$ so izračunane od $1$ do $n$. Vrednost $PLCP[1]$ je izračunan s štetjem zaporednih znakov, ki se ujemajo med $T[1,n]$ in $T[SA[SA^{-1}[1]-1],n]$. Ker je $PLCP[i-1]\le PLCP[i]+1$, potem je vrednost $PLCP[i]$ število skupnih zaporednih znakov med $T[i+d,n]$ in $T[SA[SA^{-1}[i]-1]+d,n]$, pri čemer $d=\max\{PLCP[i-1]-1,0\}$.
Pri izgradnji bitnega polja $H$ ni potrebo prvo naračunati polje $PLCP$, saj število lahko takoj zapiše enico na mesto $PLCP[i]+2i$ v bitnem polju $H$ \cite{Navarro2016}.

Za izgradnjo $LCP$ polja, ki je predstavljeno kot bitno polje $H$, je potrebnih $O(n)$ dostopov do priponskega polja $SA$. Vsak dostop do priponskega polja potrebuje $O(t_{SA})$ časa (natančen čas je odvisen od implementacije priponskega polja), ki je za predhodno predstavljeno implementacijo $t_{SA}= O(\log{n})$, in potemtakem je za izgradnjo potrebno $O(nt_{SA})$ ali za predhodno predstavljeno implementacijo $O(n\log{n})$.

Zadnja izgrajena podatkovna struktura je topologija priponskega drevesa $\tau$. Čeprav se zdi, da ni mogoče izgraditi topologije drevesa, ne da bi se izgradilo celotno priponsko drevo, je to mogoče zgraditi zgolj z uporabo priponskega polja $SA$ in $LCP$ polja. Ideja algoritma za izgradnjo topologija priponskega drevesa $\tau$ je podobna algoritmu Kasai idr. \cite{Kasai2001}.

To je storjeno s sprehodom po priponskem polju $SA$ iz leve proti desni ter za vsako pripono $i$ se doda nov list. Novi dodani list $i$ je novo skrajno desno vozlišče v do sedaj izgrajenem drevesu ter ima skupnega predhodnika $v$ z $i-1$-im listom na besedni globini $sd(v)=LCP[i]$. Če je tako vozlišče $v$ že v drevesu potem list $i$ postane njegov desni otrok. Sicer $sd(v)<LCP[i]$ in obstaja vozlišče $u$, ki je lahko $i-1$-vi list, za katerega velja  $sd(u)>LCP[i]$. V tem primeru se ustvari novo vozlišče $v'$, ki postane desni otrok od $v$ ter ima dva otroka, in sicer levega otroka $u$ ter desnega otroka listi $i$. Na ta način se izgradi priponsko drevo zgolj iz priponskega polja $SA$ in $LCP$ polja. Predstavljena metoda je implementirana z uporabo sklada, ki hrani vozlišče $v$ ter besedno globino $sd(v)$. Na skladu se ne shrani kazalca na vozlišče, ampak se shrani predstavitev poddrevesa s korenom v vozlišču $v$ z zaporedjem uravnoteženih oklepajev $P(v)$. Vozlišče $v$, za katerega velja $sd(v)\le LCP[i]$, se poišče s snetjem vozlišč iz sklada, dokler se pridemo vozlišče, za katerega pogoj drži. Tako vozlišče vedno obstaja, saj ima koren besedno dolžino $sd(koren)=0$. Na sklad je dodanih največ $n-1$ vozlišč, ker ima priponsko drevo največ $n-1$ notranjih vozlišč. Da bo na koncu izgradnje sklad prazen in topologija drevesa $\tau$ pravilna, se sklad sprazni in na ta način se pridobi pravilno topologijo $\tau$, saj vsa vozlišča na skladu ne shranjujejo topologije za njihovo skrajno desno poddrevo, ker le to še raste. Torej vsakič, ko se vozlišče sname iz sklada, se topologija poddrevesa doda staršu vozlišča, ki je vozlišče na vrhu sklada \cite{Navarro2016}.

Izgradnja topologije drevesa poteka v času $O(nt_{SA})$ oziroma $O(n\log{n})$. V vsakem koraku je v topologijo drevesa dodan nov list. Za vsak list je lahko dodano novo vozlišče na sklad, če za besedno dolžino vozliča $v$, ki je trenutno na vrhu sklada, velja $sd(v)<LCP[i]$, pri čemer je  $i$ zaporedno število pripone, ki jo predstavlja list, v priponskem polju. Za vsak list so iz sklada odvzeta vozlišča, za katera velja $sd(v)\ge LCP[i]$. V času izgradnje drevesa, je na sklad potisnjenih in snetih največ $n-1$ vozlišč. Ker postopek ustvari $n$ listov in največ $n-1$ notranjih vozlišč, je $O(nt_{SA})$ oziroma $O(n\log{n})$ potrebni čas izgradnje topologije drevesa $\tau$.

\begin{izr} \label{izr:izgradnjaCST}
    Časovna zahtevnost izgradnje kompaktnega priponskega drevesa za vhodno besedilo $T$ je $O(n\log{n})$ ter v času izgradnje je prostorska zahtevnost vedno kompaktna.
\end{izr}

\begin{proof}
    Kompaktno priponsko drevo je sestavljeno iz treh delov, torej je potrebno analizirati časovno zahtevnost izgradnje vsakega dela. Za izgradnjo kompaktnega priponskega polja je potrebno $O(n\log{n})$ časa, saj je treba najti pripone, ki so del posamičnega vedra priponskega polja.
    
    Za izgradnjo kompaktne različice $LCP$ polja je tudi potrebno $O(n\log{n})$ časa, saj je za vsako pripono potrebo izračunati dolžino predpone, v kateri se ujema s predhodno pripono. Pri tem je potreben za vsako pripono en dostop to priponskega polja, ki traja $O(\log{n})$ časa, torej je za $n$ pripon potrebno $O(n\log{n})$ časa.

    Za izgradnjo topologije drevesa pa je tudi potrebno $O(n\log{n})$ časa, saj je za vsako pripono potrebo ustvariti nov list ter novo notranje vozlišče, če ne obstaja vozlišče na poti do skrajno desnega lista, za katerega velja $sd(v)=LCP[i]$. V vsakem koraku je potrebno izračunati $LCP[i]$, ki potrebuje $O(\log{n})$ časa.

    Torej skupni čas za izgradnjo kompaktnega priponskega drevesa iz besedila $T$ je $O(n\log{n})+O(n\log{n})+O(n\log{n})=O(n\log{n})$.
\end{proof}

