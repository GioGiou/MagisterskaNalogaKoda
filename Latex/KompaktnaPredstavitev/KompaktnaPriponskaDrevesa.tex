Za daljše besede $T$ lahko priponska drevesa presežejo velikost notranjega pomnilnika. Prva možna rešitev tega problema je uporaba priponskega polja ter $LCP$ polja za simulirati operacije nad priponskim drevesom. Za dvakrat daljše besede se problem pojavi tudi za prionska polja. Torej potrebujemo prostorsko bolj učinkovito rešitev. 

Vsako vozlišče priponskega drevesa hrani podniz in reference na otroke, starša ter na vozlišče, na katerega kaže priponska povezava. Zato prostor, ki ga zasedejo drevesa na pomnilniku, ni odvisen zgolj od števila vozlišč, ampak je odvisna tudi od arhitekture računalnika, ki določa velikost pomnilniškega naslova, s katerim se referencira druga vozlišča. Na primer, priponsko drevo človeškega genoma, ki ima dolžino 3 milijarde nukleotidov iz abecede velikosti pet (pri tem je vštet v abecedo tudi, znak za konec besedila označen kot \enquote{\$}), potrebuje med $60$ in $144$ GB notranjega pomnilnika. Ekvivalentno priponsko polje pa potrebuje med $12$ in $60$ GB notranjega pomnilnika odvisno od dodatnih podatkovnih struktur, ki so bile dodane priponskemu polju. Zato prostorsko učinkovita rešitev ne sme temeljiti na referencah pomnilniških naslovov \cite{GENOMEKNOWLEDGEHUB-2024-10-30}.

Ta problem se da rešiti s kompaktno predstavitvijo podatkovnih struktur. Podatkovna struktura, ki bo predstavljena v tem poglavju, je imenovana kompaktno priponsko drevo (angl. \textit{Compressed Suffix Tree} oziroma CST) in predstavlja eno možno kompaktno predstavitev priponskega drevesa. Preden se lahko definira kompaktno predstavitev, je potrebno definirati abstraktno podatkovno strukturo priponsko drevo, ki poda vse operacije, ki so potrebne za pravilno delovanje priponskega drevesa. Večina potrebnih operacij so operacije nad drevesi, ki so bile predstavljene v Definiciji \ref{def:drevo}. Pri tem so dodane tudi specifične operacije, ki so potrebne za pravilno delovanje priponskega drevesa, ter nekatere operacije so popravljene za delovanje z znaki iz abecede $\Sigma$. Operacije, ki jih podpira priponsko drevo, so predstavljene v naslednji definiciji \cite{Sadakane2007}.

\begin{defi}\label{def:AST}
    Abstraktna podatkovna struktura priponsko drevo nad besedilo $T$ podpira sledeče operacije:
    \newpage
    \begin{enumerate}
        \item $\Koren$: vrne koren priponskega drevesa,
        \item $\List{v}$: vrne \textit{true}, če je vozlišče $v$ list, sicer vrne \textit{false},
        \item $\Otrok{v}{z}$: vrne otroka $w$ vozlišča $v$, katerega povezava se začne z znakom $z$. Če otrok ne obstaja vrne 0,
        \item $\PrviOtrok{v}$: vrne vozlišče $w$, ki je prvi otrok vozlišča $v$,
        \item $\NBrat{v}$: vrne vozlišče $w$, ki je naslednji brat vozlišča $v$,
        \item $\PBrat{v}$: vrne vozlišče $w$, ki je predhodni brat vozlišča $v$,
        \item $\Stars{v}$: vrne vozlišče $w$, ki je starš od vozlišča $v$,
        \item $\Povezava{v}{i}$: vrne $i$-ti znak na povezavi do vozlišča $v$,
        \item $\Sd{v}$ vrne število znakov na poti od korena do vozlišča $v$,
        \item $\Lca{v}{w}$: vrne najnižjega skupnega prednika od $v$ in $w$,
        \item $\Sl{v}$: vrne vozlišče $w$, na katerega kaže priponska povezava iz vozlišča $v$.
    \end{enumerate}
\end{defi}

V podpoglavju \ref{sec:STsimulacija} je bil predstavljen način simulacije priponskega drevesa z uporabo priponskega polja $SA$ in polja najdaljših skupnih predpon $LCP$. Ideja kompaktne predstavitve priponskega drevesa bo tudi temeljila na teh dveh podatkovnih strukturah. Pri tem pa bomo uporabili kompaktni različici podatkovnih struktur. Ker je veliko operacij iz Definicije \ref{def:AST} operacij nad drevesi, se lahko doda še kompaktno predstavitev topologije drevesa, saj le ta pospeši operacije nad drevesi. Iz tega sledi definicija za podatkovno strukturo kompaktno priponsko drevo.

\begin{defi}
    Kompaktno priponsko drevo nad besedilom $T$ je sestavljeno iz:
    \begin{enumerate}
        \item kompaktne predstavitve topologije drevesa $\tau$,
        \item kompaktnega zapisa priponskega polja $SA$, %stisnjeno priponsko polje %
        \item kompaktne predstavitve polja najdaljših skupnih predpon $LCP$.
    \end{enumerate}
\end{defi}

V nadaljevanju poglavja bo predstavljena Sadakanejeva \cite{Sadakane2007} implementacija kompaktnega priponskega drevesa, ki je prva kompaktna predstavitev priponskega drevesa. V podpoglavju \ref{sec:kompaktna_drevesa} je bilo predstavljenih več različnih predstavitev topologije dreves. Implementacija kompaktnega priponskega drevesa pa uporablja predstavitev z zaporedjem uravnoteženih oklepajev.

\paragraph{Topologija drevesa.}
Ker se za kompaktno priponsko drevo uporablja BP kot predstavitev topologije drevesa, je potrebno podpirati operacije $rang_p$, $izbira_p$, $zapri$, $odpri$ in $oklepa$, ki se izvajajo v konstantnem času za $p\in\{0,1\} ^*$. Pri tem sta potrebni dodatni podatkovni strukturi za operacijo $rang$ ($rang_0$, $rang_{01}$) in tri za operacijo $izbira$ ($izbira_0$, $izbira_1$ in $izbira_{01}$). Operacije $zapri$, $odpri$ in $oklepa$ pa uporabljajo podatkovno strukturo za višek (razlika med uklepaji in zaklepaji), ki sta jo predlagala Munro in Raman \cite{Munro1997}, implementirano na podoben način kot dodatna podatkovna struktura za $rang$, ki nadomesti $rmM$-drevesa. To podatkovno strukturo imenujemo $L$, ki hrani polje najnižjih viškov $L'$ in v celici $L'[i]$ je shranjen najnižji višek v $i$-tem vedru dolžine $O(\log{n})$. To pomeni, da je za operacijo $otrok(v,i)$ potrebno $O(|\Sigma|)=O(1)$ časa. 

Operacija $\Lca{v}{w}$ pa potrebuje dodatno podatkovno strukturo, ki ji omogoča konstanten čas izvajanja $rmq$ poizvedbe nad viškom. Ta podatkovna struktura potrebuje $o(n)$ dodatnih bitov. In sicer se zgradi dvodimenzionalno tabelo $M$ in vrednost $M[i][k]$ hrani položaj najmanjšega viška na intervalu $L'[i:i+2^k-1]$. Torej operacija $\Lca{v}{w}$ se izračuna kot
\begin{equation*}
    \min\left\{\min(L[v:e]),M[v'][k],M[w'-2^k+1][k],\min(L[s:w])\right\},
\end{equation*}
pri čemer je $k=\lfloor \log{(w'-v')\rfloor}$, $v'$ predstavlja vedro v $L'$, ki je prvo vedro po vedru z vozliščem $v$, $w'$ je vedro v $L'$, ki je zadnje vedro pred vedru z vozliščem $w$, $e$ predstavlja konec vedra v $L$, ki vsebuje $v$, in $s$ predstavlja začetek vedra, ki vsebuje $w$. Pri tem vsaka poizvedba v $L$ in vsaka poizvedba v $M$ potrebuje konstanten čas \cite{Sadakane2007,Valimaki2007}. 

V primeru, da so potrebne dodatne operacije, ki temeljijo na $\rMq{\cdot}{}{}$ operaciji, ali $|\Sigma|>\log{n}$, se lahko doda še $rmM$-drevo (angl \textit{range minumum maximum tree}, predstavljeno v podpoglavju \ref{sec:Bitno_Polje}), ki omogoča izvedbo operaciji v $O(\log{n})$ ter pospeši izvajanje oziroma omogoča implementacijo tudi drugih operacij.

\begin{lema}\label{lema:BP}
 Podatkovna struktura za predstavitev topologije priponskega drevesa $\tau$ nad besedilom $T$ dolžine $n$ potrebuje največ $4n+o(n)$ bitov.
\end{lema}

\begin{proof}
Priponsko drevo nad besedilom $T$ ima največ $2n-1$ vozlišč: od teh je $n-1$ notranjih vozlišč in $n$ listov. Vsako vozlišče se predstavi kot par oklepajev, torej potrebujemo 2 bita za predstaviti vsako vozlišče. Torej je potrebnih največ $4n-2$ bitov za zapis topologije drevesa $\tau$ s sekvenco uravnoteženih oklepajev. Zato celotna podatkovna struktura topologije drevesa potrebuje največ $4n+o(n)$ bitov, saj vsaka dodatna podatkovna struktura potrebuje še dodatnih $o(n)$ bitov.
\end{proof}

\paragraph{Kompaktno priponsko polje.}
Naslednja podatkovna struktura, ki je potrebna za pravilno delovanje kompaktnega priponskega drevesa, je kompaktno priponsko polje (angl. \textit{Compressed Suffix Array} oziroma CSA). Kompaktna priponska polja znižajo prostorsko zahtevnost priponskega polja iz $O(n\log{n})$ oziroma $O(nw)$ na $O(n\log{|\Sigma|})$ bitov \cite{Grossi2000} ali celo na $nH_h +o(n)$ bitov \cite{Grossi2003}, pri čemer je red $h\le\alpha\log_{|\Sigma|}{n};0<\alpha<1$ in $H_h$ je entropija $h$-tega reda, ki se v praksi uporablja kot merilo prostorske zahtevnosti pri kodiranju besedil \cite{Navarro2016}.

Obstajata dve različici kompaktnih priponskih polj: prva implementacija temelji na funkciji $\Psi$, druga implementacija imenovana $FM$-indeks pa uporablja $LF$ funkcijo, ki je inverzna funkcija od $\Psi$. Funkcija $\PsiFunkcija{i}$ vrne položaj pripone $T[i+1:]$ v priponskem polju ali z drugimi besedami
$$
    \PsiFunkcija{i}=SA^{-1}[SA[i]+1],
$$
pri čemer $SA^{-1}[i]$ hrani polažaj pripone $T[i:]$ v priponskem polju $SA$ in polje $SA^{-1}[1:n]$ imenujemo inverzno priponsko polje. Torej se lahko s funkcijo $\Psi$ sprehajamo po priponskem polju oziroma besedi $T$ iz leve proti desni. Funkcija $LF$ pa omogoča sprehod v obratni smeri, iz desne proti levi \cite{Navarro2016, Sadakane2007}.


Implementacije kompaktnega priponskega polja, ki so lahko uporabljajo pri implementaciji kompaktnih priponskih dreves, morajo podpirati sledeče operacije.

\begin{defi}\label{def:csa}
     Kompaktno priponsko polje nad besedilom $T$, ki je uporabljeno v kompaktnem priponskem drevesu, podpira sledeče operacije z dano časovno zahtevnostjo:
    \begin{enumerate}
        \item $\Pripona{i}$: vrne začetni indeks pripone, shranjen v $SA[i]$, v času $t_{SA}$,
        \item $\Inverz{i}$: vrne indeks $j=SA^{-1}[i]$, pri čemer je $SA[j]=i$, v času $t_{SA}$,
        \item $\PsiFunkcija{i}$: vrne $SA^{-1}[SA[i]+1]$ v času $t_\Psi$,
        \item $\Beseda{i}{d}$: vrne $T[SA[i],SA[i]+d-1]$ v času $O(dt_\Psi)$.
    \end{enumerate}    
\end{defi}

Funkcija $\Psi$ je uporabna v kompaktnih priponskih drevesih za izgradnjo priponskih povezav v času $t_\Psi$, čeprav ni potrebna za pravilno delovanje priponskega polja. Iz definicije funkcije $\Psi$ je razvidno, da jo lahko simuliramo z uporabo operacij $\Inverz{\cdot}$ in $\Pripona{\cdot}$. Zato je čas, potreben za izvršiti funkcijo $\Psi$ $t_\Psi\le t_{SA}$. Toliko namreč potrebujeta operaciji $\Inverz{\cdot}$ in $\Pripona{\cdot}$. Iz Definicije \ref{def:csa} sledi, da mora kompaktno priponsko polje podpirati funkcijo $\Psi$, zato implementacije s $FM$-indeksom ne pridejo v poštev, saj morajo funkcijo $\Psi$ simulirati kot $\Inverz{\Pripona{\cdot}}$.

V nadaljevanju tega dela poglavja bo predstavljena preprosta različica kompaktnega priponskega polja. Ta različica je bila izbrana zaradi enostavne implementacije operaciji ter preproste vizualizacije podatkovne strukture. V tej implementacije so vse štiri operacije implementirane z uporabo $\Psi$ funkcije ter z vzorčenjem priponskega polja $SA$ in inverznega priponskega polja $SA^{-1}$. Na ta način ni potrebno hraniti besedila $T$ in celotnega priponskega polja $SA$.

Funkcija $\Psi$ je predstavljena z istoimenskim poljem, katerega $i$-ta celica je $\Psi[i]=\PsiFunkcija{i}=SA^{-1}[SA[i]+1]$. Ideja kompaktnega zapisa priponskega polja temelji na dejstvu, da se lahko premikamo po besedilu in posledično med priponami zgolj z uporabo $\Psi$ funkcije, saj je $SA[\Psi[i]]=SA[i]+1$. Torej se lahko znebimo besede $T$ in jo zamenjamo z bitnim poljem sprememb $D$, pri čemer $D[i]=1$, ko je $i=1$ ali $T[SA[i]]\ne T[SA[i+1]]$, ter s poljem znakov $S$, urejenih v leksikografskem vrstnem redu, tako da je $T[SA[i]]=S[rang_1(D,i)]$ \cite{Navarro2016}.

Kompaktni zapis polja $\Psi$ razdeli polje $\Psi$ na $|\Sigma|$ delov in uvede polje $C$ začetnih točk intervalov v $\Psi$ polju, ki se začnejo s poljubnim znakom $c\in\Sigma$, oziroma $C[c]=i$ za poljuben znak iz $\Sigma$, tako da je $T[SA[i+1]]=c$ in $T[SA[i]]\ne c$. Polje $C$ potrebuje $O(|\Sigma|\log{n})$ bitov. Tako se lahko nadomesti $\Psi$ polje s polji $\Psi_c;\:c\in\Sigma$, kjer je $\Psi[i]=\Psi_c[i']$, za $i'=i-C[S[rang_1(D,i)]]$. Vsako polje $\Psi_c$ se lahko zapiše kot bitno polje $B_c$ dolžine $n$, pri čemer $B_c[\Psi_c[i]]=1$ za $1 \le i \le n_c$, kjer je $n_c$ število ponovitev znaka $c$ v besedilu $T$. Torej velja $\Psi_c[i']=izbira_1(B_c,i')$ ali $\Psi[i]=izbira_1(B_c,i-C[c])$, kjer $c=S[rang_1(D,i)]$. Bitna polja $B_c$ so zelo redka (število enic je bistveno manjše kot število ničel), torej se jih lahko stisne iz $n+o(n)$ bitov na $n_c\log\frac{n}{n_c}+O(n_c)$ bitov, pri čemer ohranimo možnost opravljanja operacije $\textit{izbira}_1$ v konstantnem času, kar pomeni, da funkcija $\Psi$ potrebuje $nH_0(T)+O(n+|\Sigma|w)$ bitov, pri čemer $H_0(T)$ je entropija besedila $T$, ter potrebuje $t_\psi=O(1)$ časa \cite{Navarro2016}.

Do sedaj predstavljena podatkovna struktura omogoča zgolj iskanje števila ponovitev vzorca v besedilu, saj omogoča premikanje med intervali s priponami z istim začetnim znakom, pri tem pa ne omogoča lokacije pojavov vzorca v besedi. Za to sta potrebni dodatni podatkovni strukturi, ki nadomestita priponsko polje $SA$ ter inverzno polje $SA^{-1}$. Polji se lahko nadomestita s poljema vzorcev. Priponsko polje $SA$ se vzorči $l=\Theta(\log{n})$-krat, kar je dober kompromis med časovno in prostorsko zahtevnostjo. Pri tem se uporabi dodatno bitno polje $B[1,n]$, kjer $B[i]=1$ natanko tedaj, ko $i=1$ ali $SA[i]\mod{l} =0$. Polje vzorcev $SA_S$ vsebuje vrednosti $SA_S[rang_1(B,i)]=SA[i]$, ko je $B[i]=1$. Ostale vrednosti $SA[i]$ se izračuna kot $SA[i]=SA_S[rang_1(B,i_k)]-k$, pri čemer je $i_k$ $k$-kratna uporaba funkcije $\Psi$ nad vrednostjo $i$ oziroma $i_k=\Psi^k(i)$ (na primer $i_2=\Psi^2(i)=\Psi[\Psi[i]]$). Ker je $k< l$, je časovna zahtevnost poizvedbe v priponskem polju $t_{SA}=O(l)=O(\log{n})$ \cite{Navarro2016}.

Podobno se vzorči tudi polje inverzov pripon $SA^{-1}$, le da se vzorči polje $SA^{-1}$ na enakomernih intervalih dolžine $l=\Theta(\log{n})$, kar je dober kompromis med časovno in prostorsko zahtevnostjo. Torej ima $i$-ta celica polja $SA^{-1}_S[1,\lfloor n/l\rfloor]$ vrednost $SA^{-1}_S[i]=SA^{-1}[il]$. Poljubno vrednost $SA^{-1}$ se izračuna tako, da se najprej izračuna $i'=\lfloor i/l\rfloor l$, nato se $(i-i')$-krat uporabi funkcija $\Psi$ nad vrednostjo $j'=SA^{-1}_S[i'/l]$ oziroma $SA^{-1}[i]=\Psi^{i-i'}[j']=\Psi^{i-i'}[SA^{-1}_S[i'/l]]$. Isti razmislek kot za časovno zahtevnost dostopa do polja $SA$ velja tudi za časovno zahtevnost dostopa do polja $SA^{-1}$, ki ima časovno zahtevnost dostopa $t_{SA}=O(l)=O(\log{n})$. Pri tem velja omeniti, da če se inverzno priponsko polje uporablja bolj poredko, se lahko vzorči z večjim faktorjem $l$, kot se je vzorčilo priponsko polje $SA$. Na ta način se zmanjša prostorsko zahtevnost, ampak se poslabšajo časovne zahtevnosti operacije, ki se jo bolj poredko uporablja \cite{Navarro2016}.

Tako predstavljeno kompaktno priponsko polje potrebuje $|\Sigma|n + O(n)$ bitov. Ker pa je večina bitnih polji redkih, saj je število bitov z vrednostjo 0 veliko večje kot število bitov z vrednostjo 1, se lahko bitna polja zakodira na bolj kompakten način z enim od algoritmov za stiskanje. Torej stisnjena bitna polja potrebujejo $nH_0(T)+O(n+|\Sigma|w)$ bitov za implementacijo kompaktnega priponskega polja. Potrebni čas za izračun funkcije $\Psi$ ostaja $t_\Psi=O(1)$, potrebni čas za iskanje po priponskem polju in inverznem priponskem polju pa je $t_{SA}=l\cdot t_\Psi = t_\Psi O(\log{n}) = O(\log{n})$.


\paragraph{Polje najdaljših skupnih predpon.}
Zadnja podatkovna struktura, ki sestavlja kompaktno priponsko drevo, je polje najdaljših skupnih predpon oziroma $LCP$ polje. Namesto $LCP$ polja se bomo osredotočili na permutacijo $LCP$ polja imenovano $PLCP$, za katero velja relacija $PLCP[i]=LCP[SA^{-1}[i]]$ ali $LCP[i]=PLCP[SA[i]]$ in jo je lažje predstaviti v kompaktnem zapisu \cite{Navarro2016, Sadakane2007}.

Ideja kompaktnega zapisa $LCP$ polja, če smo bolj natančni $PLCP$ polja, temelji na dejstvu, da je $LCP[i]=LCP[SA^{-1}[i]+1]-1$. Torej če to dejstvo lahko zapišemo kot strogo naraščajoče zaporedje, se lahko zaporedje predstavi kot bitno polje $H$ in vrednost $H[i]=1$ za vsako vrednost v zaporedju, sicer je $H[i]=0$. Torej se $LCP[i]$ lahko izračuna kot $\Izbira{1}{H}{i}$.

Zaporedje $PLCP[i]+2i$ je za vsak $i$ med $1$ in $n-1$ strogo naraščajoče, saj je $PLCP[i+1]\ge PLCP[i]-1$. To lahko enostavno dokažemo, saj iz $PLCP[i]=0$ sledi $PLCP[i+1]=0$. Drugače pa obstajata priponi $T[j:]$ in $T[i:]$, kjer je $T[i:]<T[j:]$ in imata najdaljšo skupno predpono dolžine $PLCP[j]>0$. Potem imata $T[i+1:]$ in $T[j+1:]$ najdaljšo skupno predpono dolžine $PLCP[j]-1$. Torej je $PLCP[j+1] = PLCP[j]-1$. Posledično velja, da je $PLCP[j]+2j<PLCP[j+1]+2(j+1)= PLCP[j]+2j+1$. Vrednost $PLCP[n-1]+2(n-1)< 2n$, saj ima lahko pripona $T[n-1:]$ dolžino najdaljše skupne predpone s poljubno pripono 1. Zato se lahko permutacijo $PLCP$ in posledično $LPC$ polje zapiše kot bitno polje $H[1:2n-1]$. Celica $H[j]=1$ za $j=PLCP[i]+2i$, kjer je $i$ med $1$ in $n-1$, sicer pa je $H[j]=0$. Pri tem bitno polje $H$ potrebuje dodatno podatkovno strukturo za $izbiro_1$ \cite{Navarro2016, Sadakane2007}.

Vrednost $LCP[i]$ se lahko pridobi v času $O(t_{SA})$. Vrednost je izračunana z uporabo formule $LCP[i]=\Izbira{1}{H}{SA[i]}-2SA[i]$. Poizvedba potrebuje $O(t_{SA})$ časa, saj je potrebno izračunati vrednost $SA[i]$, ki se jo izračuna v $O(t_{SA})$ časa, ostale operacije pa potrebujejo konstanten čas. Če je uporabljena predhodno predstavljena implementacija kompaktnega polja, je čas poizvedbe v $LCP$ polj enak $O(\log{n})$. Prostorska zahtevnost polja $LCP$ je predstavljena s sledečo lemo:

\begin{lema}\label{lema:LCP}
 Podatkovna struktura $LCP$ potrebuje $2n+o(n)$ bitov.
\end{lema}

\begin{proof}
    Podatkovna struktura $LCP$ je shranjena kot bitno polje $H[1,2n-1]$. Bitno polje $H$ potrebuje dodatnih $o(n)$ bitov za dodatno podatkovno strukturo, ki omogoča izvajanje operacije $izbira_1$ v konstantnem času. Ker je bitno polje $H$ dolžine $2n-1$ in potrebuje dodatnih $o(n)$ bitov, potem celotna podatkovna struktura potrebuje $2n+o(n)$ bitov za shraniti polje $LCP$.
\end{proof}

Sedaj, ko so bile predstavljene vse podatkovne strukture, ki sestavljajo kompaktno priponsko drevo, je možno izračunati velikost celotnega kompaktnega priponskega drevesa. Ker obstaja več implementacij kompaktnega priponskega polja, ki se lahko uporabijo v kompaktnem priponskem drevesu, bo velikost priponskega drevesa vsebovala člen $|CSA|$, ki predstavlja velikost kompaktnega priponskega polja. Velikost kompaktnega priponskega drevesa je predstavljena v sledečem izreku:

\begin{izr}
    Podatkovna struktura kompaktno priponsko drevo nad besedo $T$ dolžine $n$ potrebuje $|CSA|+6n+o(n)$ bitov, pri čemer $|CSA|$ predstavlja velikost kompaktnega priponskega polja.
\end{izr}
\begin{proof}
    Ker obstajajo različne implementacije kompaktnega priponskega polja, je potrebnih $|CSA|$ bitov za shraniti kompaktno priponsko polje $SA$. Iz Leme \ref{lema:BP} sledi, da je potrebnih $4n+o(n)$ bitov za shraniti topologijo drevesa $\tau$. Iz Leme \ref{lema:LCP} pa sledi, da je potrebnih $2n+o(n)$ bitov za shraniti polje $LCP$.

    Torej je velikost kompaktnega priponskega drevesa $CST$ enaka $|CSA|+6n+o(n)$ bitov.
\end{proof}

Kot je bilo že rečeno, obstaja več različnih implementacij kompaktnega priponskega polja, zato Sadakane \cite{Sadakane2007} predlaga dve implementaciji. Prva predlagana implementacija je prostorsko učinkovita in uporablja kompaktno priponsko polje, ki so ga predlagali Grossi idr. \cite{Grossi2003}. 

\begin{posl}\label{pos:CSAnh}
    Naj bo kompaktno priponsko drevo $CST$ implementirano z uporabo kompaktnega priponskega polja $CSA$, ki potrebuje $|CSA|=nH_h+O(n\log\log{n} / \log_{| \Sigma|}{n})$ bitov. Potem takšno drevo potrebuje $|CST| = nH_h+6n+O(n\log\log{n} / \log_{|\Sigma|}{n})$ bitov.    
\end{posl}

Druga predlagana implementacija pa je časovno učinkovita in uporablja kompaktno priponsko polje, ki ga sta ga predlagala Grossi in Vitter \cite{Grossi2000}. 
\begin{posl}\label{pos:CSAlog}
    Naj bo kompaktno priponsko drevo $CST$ implementirano z uporabo kompaktnega priponskega polja $CSA$, ki potrebuje $O(\epsilon^{-1}n\log{|\Sigma|})$ bitov. Potem takšno drevo potrebuje $|CST|=O(\epsilon^{-1}n\log{|\Sigma|})$ bitov. Pri tem je $\epsilon$ poljubna konstanta, ki ima vrednost $0<\epsilon<1$.
\end{posl}

\begin{figure}[htb]       
        \begin{subfigure}[t]{1\linewidth}       
            \includesvg[scale=1.35]{Slike/KompaktnoPriponskoPolje.svg}
            \centering
            \subcaption*{}
            \label{fig:neKompaktnoDrevo}
        \end{subfigure}
        \begin{subfigure}[t]{1\linewidth}        
            \includesvg[scale=1.1]{Slike/TopologijaInLCPKompaktna.svg}
            \centering
            \subcaption*{}
            \label{fig:KompaktnoPolje}
        \end{subfigure}
    \caption{Primer komponent kompaktnega priponskega drevesa nad besedo \enquote{KOKOŠ$\$$}.} 
    \label{fig:CST}
\end{figure}

Primer kompaktnega priponskega drevesa je prikazan na Sliki \ref{fig:CST}. Prikazana podatkovna struktura potrebuje približno $12$ B (natančno potrebuje $97$ bitov). Pri tem so cela števila zapisana z $\log{n}$ biti ($n=6$). Ekvivalentno priponsko drevo potrebuje vsaj $18$B (oziroma 142 bita), priponsko polje pa potrebuje vsaj $9$ B (oziroma 36 bitov, pri čemer priponsko polje, ki simulira priponsko drevo, potrebuje približno $12$ B oziroma 90 bitov). Pri tem nobena podatkovna struktura ne vključuje dodatnih podatkovnih struktur, ki omogočajo operacije v konstantnem času. Nekompaktne podatkovne strukture po navadi ne uporabljajo zapisa celih števil z $O(\log{n})$ biti, kot je bilo uporabljeno v teh izračunih, ampak se uporablja vsaj $1$ B (8 bitov), po navadi $4$ B, kar pa dodatno poveča razliko.

\paragraph{Izboljšave.}
Prostorsko zahtevnost kompaktnega priponskega drevesa je možno še dodatno znižati na $|CSA|+o(n)$. Russo idr. \cite{Russo2008} so predstavili način, kako doseči to prostorsko zahtevnost. To so dosegli tako, da so vzorčili $O(n/\delta)$ vozlišč, pri čemer $\delta=\omega(\log_{|\Sigma|}{n})$ in predstavlja faktor vzorčenja. Vzorčeno drevo potrebuje $o(n)=O(n/\log_{|\Sigma|}{n})$ bitov. Poleg vzorčenja topologije drevesa so se znebili tudi $LCP$ polja. To jim je omogočalo znižanje prostorske zahtevnosti iz $|CSA|+6n+o(n)$ na $|CSA|+o(n)$ bitov. Na ta račun pa se je dvignila časovna zahtevnost nekaterih operacij za $\omega(\log{n})$ krat, če se želi ohraniti $o(n)$ dodatnega prostora. Nekatere od teh operacij so predhodno potrebovale konstanten čas izvajanja.

Preden nadaljujemo z gradnjo in poizvedbami nad $CST$, si pogledamo primer človeškega genoma iz začetka poglavja. S priponskim drevesom se potrebuje $144$ GB notranjega pomnilnika za indeksirati celoten človeški genom. Z uporabo kompaktnega priponskega drevesa, namesto priponskega drevesa, pa je potrebnih približno $3$ GB delovnega pomnilnika za indeksirati celoten genom. Razlika v zahtevanem prostoru omogoča, da je lahko celotno kompaktno priponsko drevo shranjeno v delovnem spominu, za razliko od priponskega drevesa, za katerega to ni mogoče\footnote{Nekateri strežniki omogočajo večjo količino delovnega pomnilnika, ki presega $144$ GB. Večina potrošniških računalnikov pa še vedno uporablja med $8$ GB in $64$ GB delovnega pomnilnika.}.

\section{POIZVEDBE}\label{sec:CSTpoizvedba}
\import{.}{KompaktnaPredstavitev/Poizvedba}

\section{GRADNJA}\label{sec:CSTizgradnja}
\import{.}{KompaktnaPredstavitev/Izgradnja}




