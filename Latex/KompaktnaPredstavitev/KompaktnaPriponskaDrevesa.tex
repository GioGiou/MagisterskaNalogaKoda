Za daljše besede $T$ lahko priponska drevesa presežejo velikost notranjega pomnilnika. Prva možna reštiev tega problema je uporaba priponskega polja ter $LCP$ polja za simulirat priponsko drevo, ampak se pri $1,4$-krat dalši vhodni besedi ponovno pojavi problem preseženja velikosti notranjega pomnilnika. Torej se potrebuje reštitev, ki bo še bolj prostorsko učinkovita. 

Vsako vozlišče priponskega drevesa hrani niz, ki prestavlja vhodno povezavo, in reference na otroke, starša ter na vozlišče, na katerega kaže priponska povezava. Zato prostor, ki ga zasedejo drevesa na pomnilniku, ni odvisna zgolj od števila vozlišč, ampak je odvisna tudi od arhitekture računalnika, ki določa velikost pomnilniškega naslova, s katerim se referencira druga vozlišča. Na primer priponsko drevo človeškega genoma, ki ima dolžino 3 milijarde nukleotidov iz abecede velikosti pet (pri tem je vštet v abecedo tudi, znak za konec besedila), potrebuje med $60$ in $144$ GB notranjega pomnilnika, ki je odvisno od podatkov shranijih v vozliščih \cite{GENOMEKNOWLEDGEHUB-2024-10-30}. Ekvivalentno priponsko polje pa potrebuje med $12$ in $60$ GB notranjega pomnilnika odvisno od dodatnih podatkovnih struktur, ki so bile dodane priponskemu polju. Zato prostorsko učinkovita reštiev ne sme temeljiti na referencah pomnilniških naslovov.

Ta problem se da rešiti s kompaktno predstavitvijo podatkovnih struktur. Podatkovna struktura, ki se jo bo predstavilo v tem poglavuju, je imenovana kompaktno priponsko drevo (angl. \textit{Compressed Suffix Tree} oziroma CST) in predstavlja eno možno kompaktno predstavitev priponskega drevesa. Preden se lahko definira kompatno predstavitev je potrebno definirati abstraktno podatkovno strukturo priponsko drevo, ki poda vse operacije, ki so potrebne za pravilno delovanje priponskega drevesa \cite{Sadakane2007}.

\begin{defi}\label{def:AST}
    Abstraktna podatkovna struktura priponsko drevo nad besedilo $T$ podpira sledeče operacije:
    \begin{enumerate}
        \item $koren()$: vrne koren priponskega drevesa,
        \item $jeList(v)$: vrne »Da«, če je vozlišče list, sicer vrne »Ne«,
        \item $otrok(v,z)$: vrne otroka $w$, katerega povezava se začne z znakom $z$. Če otrok ne obstaja vrne 0,
        \item $prviOtrok(v)$: vrne vozlišče $w$, ki je prvi otrok vozlišča $v$,
        \item $nbrat(v)$: vrne vozlišče $w$, ki je naslednji brat od vozlišča $v$,
        \item $pbrat(v)$: vrne vozlišče $w$, ki je predhodni brat od vozlišča $v$,
        \item $star\textit{š}(v)$: vrne vozlišče $w$, ki je starš od vozlišča $v$,
        \item $povezava(v,i)$: vrne $i$-to črko na povezavi do vozlišča $v$,
        \item $globinaNiza(v)$ vrne število znakov na poti iz korena do vozlišča $v$,
        \item $lca(v,w)$: vrne najnižjega skupnega prednika od $v$ in $w$,
        \item $sl(v)$: vrne vozlišče $w$, na katerega kaže priponska povezava iz vozlišča $v$.
    \end{enumerate}
\end{defi}

V podpoglavju \ref{sec:STsimulacija} je bil predstavili način simulacije priponskega drevesa s pomočjo priponskega polja $SA$ in polja najdaljših skupnih predpon $LCP$. Z uporabo kompaktnih različic teh dveh podatkovnih struktur je možno implementirati kompaktno priponsko drevo. Ker pa je veliko operacij iz Definicije \ref{def:AST} operacij nad drevesi, se lahko doda še kompaktno predstavitev topologije drevesa, saj le ta pospeši operacije nad drevesi. Torej sledi definicija za podatkovno strukturo kompaktno priponsko drevo.

\begin{defi}
    Kompaktno priponsko drevo nad besedilom $T$ je sestavljeno iz sledečih podatkovnih struktur:
    \begin{enumerate}
        \item topologija drevesa $\tau$,
        \item kompaktno priponsko polje $SA$, %stisnjeno priponsko polje %
        \item kompaktna predstavitev polja najdaljših skupnih predpon $LCP$.
    \end{enumerate}
\end{defi}

V nadaljevanju poglavja bo predstavljena Sadakanejeva \cite{Sadakane2007} implementacija kompaktnega priponskega drevesa, ki je prva kompaktna predstavitev priponskega drevesa. V Poglavju \ref{sec:kompaktna_drevesa} je bilo predstavljenih več različnih predstavitev topologije dreves. Implementacija kompaktnega priponskega drevesa pa uporablja predstavitev z zaporedjem uravnoteženih oklepajev.

\paragraph{Topologija drevesa:}
Ker se za kompaktno priponsko drevo uporablja BP kot topologijoa drevsa, je potrebno podpirati operacije $rang_p$, $izbira_p$, $zapri$, $odpri$ in $oklepa$, ki se izvajajo v konstantnem času za $p\in\{0,1\} ^*$. Pri tem so potrebne dve dodatni podatkovni strukturi za operacijo $rang$ ($rang_0$, $rang_{01}$) in tri za operacijo $izbira$ ($izbira_0$, $izbira_1$ in $izbira_{01}$). Ostale operacije pa uporabljajo podatkovno strukturo za višek predstavljeno, ki sta jo predlagala Munro in Raman \cite{Munro1997} in je implementirana na podoben način kot dodatna podatkovna struktura za $rang$, namesto $rmM$-drevesa. To podatkovno strukturo imenujemo $L$, ki hrani polje najnižjih viškov $L'$ in v celici $L'[i]$ je shranjen najnižji višek v $i$-tem vedru dolžine $O(\log{n})$. To pomeni, da je za operacijo $otrok(v,i)$ potrebno $O(|\Sigma|)=O(1)$ časa. 

Operacija $lca(v,w)$ pa potrebuje dodatno podatkovno strukturo, ki ji omogoča konstanten čas izvajanja $rmq$ poizvedbe nad viškom. Ta podatkovna struktura potrebuje $o(n)$ dodatnih bitov. In sicer se izgradi dvodimenzionalno celico tabele $M[i][k]$, ki shrani položaj najmanjšega viška na intervalu $L'[i..i+2^k-1]$. Torej operacija $lca(v,w)$ se izračuna kot
\begin{equation*}
    \min\left\{
    \begin{array}{l}
        \min\{L[v,e]\},\\ 
        \min\{M[v'][k],M[w'-2^k+1][k]\},\\
        \min\{L[s,w]\}
    \end{array}\right\},
\end{equation*}
pri čemer je $k=\lfloor \log{(w'-v')\rfloor}$, $v'$ predstavlja vedro v $L'$, ki je prvo vedro po vedru z vozliščem $v$, $w'$ je vedro v $L'$, ki je prvo vedro pred vedru z vozliščem $w$, $e$ predstavlja konec vedra v $L$, ki vsebuje $v$, in $s$ predstavlja začetek vedra, ki vsebuje $w$. Pri tem vsaka poizvedba v $L$ in vsaka poizvedba v $M$ potrebuje konstantno časa \cite{Sadakane2007,Valimaki2007}. 

V primeru, da so potrebne dodatne operacije, ki temeljijo na $rMq$ operaciji in $|\Sigma|>\log{n}$, se lahko doda še $rmM$-drevo, ki omogoča izvedbo operaciji v $O(\log{n})$ ter pospeši izvajanje oziroma omogoča implementacijo tudi drugih operacij.

\begin{lema}\label{lema:BP}
 Podatkovna struktura za predstavitev topologije priponskega drevesa $\tau$ nad besedilom $T$ dolžine $n$ potrebuje $4n+o(n)$ bitov.
\end{lema}

\begin{proof}
Priponsko drevo nad besedilom $T$ ima $2n-1$ vozlišč: od tega je $n-1$ notranjih vozlišč in $n$ listov. Zato je potrebnih $4n-2$ bitov za zapis topologije drevesa $\tau$ s sekvenco uravnoteženih oklepajev. Torej celotna podatkovna struktura topologije drevesa potrebuje $4n+o(n)$ bitov, saj vsaka dodatna podatkovna struktura potrebuje še dodatnih $o(n)$ bitov.
\end{proof}

\paragraph{Kompaktno priponsko polje:}
Naslednja podatkovna struktura, ki je potrebna za pravilno delovanje kompaktnega priponskega drevesa, je kompaktno priponsko polje (angl. \textit{Compressed Suffix Array} oziroma CSA). Kompaktna priponska polja znižajo prostorsko zahtevnost priponskega polja iz $O(n\log{n})$ oziroma $O(nw)$ na $O(n\log{|\Sigma|})$ bitov \cite{Grossi2000} ali celo na $nH_h +o(n)$ bitov \cite{Grossi2003}, pri čemer je red $h\le\alpha\log_{|\Sigma|}{n};0<\alpha<1$ in $H_h$ je entropija $h$-tega reda, ki se v praksi uporablja kot merilo prostorske zahtevnosti pri kodiranju besedil \cite{Navarro2016}.

Obstajata dve različici kompaktnih priponskih polj: prva implementacija temelji na funkciji $\Psi$, druga implementacija imenovan $FM$-indeks pa uporablja $LF$ funkcijo, ki je inverzna funkcija od $\Psi$ \cite{Navarro2016, Sadakane2007}.

Implementacije kompaktnega priponskega polja, ki so lahko uporabljajo pri implementaciji kompaktnih priponskih dreves, morajo podpirati sledeče operacije.

\begin{defi}\label{def:csa}
     Kompaktno priponsko polje nad besedilom $T$, ki je uporabljeno za predstavitev kompaktnega priponskega drevesa, podpira sledeče operacije:
    \begin{enumerate}
        \item $pripona(i)$ vrne $SA[i]$ v času $t_{SA}$,
        \item $inverz(i)$ vrne $j=SA^{-1}[i]$, pri čemer je $SA[j]=i$, v času $t_{SA}$,
        \item $\Psi(i)$ vrne $SA^{-1}[SA[i]+1]$ v času $t_\Psi$,
        \item $besedilo(i,d)$ vrne $T[SA[i],SA[i]+d-1]$ v času $O(dt_\Psi)$.
    \end{enumerate}    
\end{defi}

Funkcija $\Psi$ je uporabna v kompaktnih priponskih drevesih za izgradnjo priponskih povezav v času $t_\Psi$, čeprav ni potrebna za pravilno delovanje priponskega polja. Iz definicije funkcije $\Psi$ je razvidno, da jo lahko simuliramo z uporabo operacij $inverz(i)$ in $pripona(i)$. Iz Definicije \ref{def:csa} sledi, da mora kompaktno priponsko polje podpirati funkcijo $\Psi$, zato implementacije z $FM$-indeksom ne pridejo v poštev. 

V nadaljevanju tega dela poglavja bo predstavljena preprosta različica kopaktenga priponskega polja. Ta različica je bila izbrana zaradi enostavne implementacije operaciji ter preproste vizualizacije podatkovne strukture. V tej implementacije so vse štiri operacije implementirane z uporabo $\Psi$ funkcije ter z vzorčenjem priponskega polja $SA$ in inverznega priponskega polja $SA^{-1}$. Na ta način ni potrebno hraniti besedila $T$ in celotnega priponskega polja $SA$.

Funkcija $\Psi$ je predstavljena z istoimenskim poljem, katerega $i$-ta celica je $\Psi[i]=\Psi(i)=SA^{-1}[SA[i]+1]$. Ideja kompaktnega zapisa polja $\Psi$ temelji na dejstvu, da se lahko premikamo po besedilu in posledično med priponami zgolj z uporabo $\Psi$ funkcije, saj je $SA[\Psi[i]]=SA[i]+1$. Torej se lahko znebimo besede $T$ in jo zamenjamo z bitno polje spremeb v prvem tanku $D$, pri čemer $D[i]=1$, ko je $i=1$ ali $T[SA[i]]\ne T[SA[i+1]]$, ter s poljem znakov $S$ urejenih v leksikografskem vrstnem redu, tako da je $T[SA[i]]=S[rang_1(D,i)]$ \cite{Navarro2016}.

Kompaktni zapis polja $\Psi$ razdeli polje $\Psi$ na $|\Sigma|$ delov in uvede polje $C$ začetkov intevalov v $\Psi$ polju, ki se začnejo z poljubnim zankom $c\in\Sigma$, oziroma $C[c]=i$ za poljubn znak iz $\Sigma$, tako da je $T[SA[i+1]]=c$ in $T[SA[i]]\ne c$. Polje $C$ potrebuje $O(|\Sigma|\log{n})$ bitov. Tako se lahko nadomesti $\Psi$ polje z polji $\Psi_c;\:c\in\Sigma$, kjer je $\Psi[i]=\Psi_c[i']$,  za $i'=i-C[S[rang_1(D,i)]]$. Vsako polje $\Psi_c$ se lahko zapiše kot bitno polje $B_c$ dolžine $n$, pri čemer $B_c[\Psi_c[i]]=1$ za $1 \le i \le n_c$, kjer je $n_c$ število ponovitev znaka $c$ v besedilu $T$. Torej velja $\Psi_c[i']=izbira_1(B_c,i')$ ali $\Psi[i]=izbira_1(B_c,i-C[c])$, kjer $c=S[rang_1(D,i)]$. Bitna polja $B_c$ so zelo redka (število enic je bistveno manjše kot število ničel), torej se jih lahko stisne iz $n+o(n)$ bitov na $n_c\log\frac{n}{n_c}+O(n_c)$ bitov, pričemer ohranimomožnost opravljnja operacije $\textit{izbira}_1$ v konstantnem času, kar pomeni, da funkcija $\Psi$ potrebuje $nH_0(T)+O(n+|\Sigma|w)$ bitov, pri čemer $H_0(T)$ je entropija besedila $T$, ter potreuje $t_\psi=O(1)$ časa \cite{Navarro2016}.

Do sedaj predstavljena podatkovna struktura omogoča zgolj iskanje števila ponovitev vzorca v besedilu, saj omogoča premikanje med intervali z priponami z istim začetnim zankom, ne pa lokacije pojavov vzorca v besedilu. Za to sta potrebni dodatni podatkovni strukturi, ki nadomestita priponsko polje $SA$ ter inverzno polje $SA^{-1}$. Oba polja se lahko nadomesti s poljema vzorcev. Priponsko polje $SA$ se vzorči $l=\Theta(\log{n})$-krat, kar je dober kopromis med časovno in postorsko zahtevnostjo. Pri tem se uporabi dodatno bitno polje $B[1,n]$, kjer $B[i]=1$ natanko tedaj, ko $i=1$ ali $SA[i]\mod{l} =0$. Polje vzorcev $SA_S$ vsebuje vrednosti $SA_S[rang_1(B,i)]=SA[i]$, ko je $B[i]=1$. Ostale vrednosti $SA[i]$ se izračuna kot $SA[i]=SA_S[rang_1(B,i_k)]-k$, pri čemer je $i_k$ $k$-kratna aplikacija funkcije $\Psi$ nad vrednostjo $i$ oziroma $i_k=\Psi^k(i)$ (na primer $i_2=\Psi^2(i)=\Psi[\Psi[i]]$). Ker je $k< l$, je časovna zahtevnost poizvedbe v priponskem polju $t_{SA}=O(l)=O(\log{n})$ \cite{Navarro2016}.

Podobno se vzorči tudi polje inverzov pripon $SA^{-1}$ le, da se ta vzorči na enakomernih intervalih dolžine $l=\Theta(\log{n})$, kar je dober kopromis med časovno in postorsko zahtevnostjo. Torej ima $i$-ta celica polja $SA^{-1}_S[1,\lfloor n/l\rfloor]$ vrednost $SA^{-1}_S[i]=SA^{-1}[il]$. Najprej se izračuna $i'=\lfloor i/l\rfloor l$, nato se $i-i'$-krat aplicira funkcija $\Psi$ nad vrednostjo $j'=SA^{-1}_S[i'/l]$, kar se zapiše kot $SA^{-1}[i]=\Psi^{i-i'}[j']=\Psi^{i-i'}[SA^{-1}_S[i'/l]]$. Isti razmisliek kot za polje $SA$ velja tudi za polje $SA^{-1}$, ki ima časovno zahtevnost dostopa $t_{SA}=O(l)=O(\log{n})$. Pri tem velja omeniti, da če se inverzno priponsko polje uporablja bolj poredko, se lahko vzorči z večjim faktorjem $l$ kot se je vzorčilo priponsko polje $SA$. Na ta način se zmanjša prostorsko zahtevnost, ampak se poslabša časovne zahtevnosti operacije, ki se jo poredko uporablja \cite{Navarro2016}.

Tako predstavljeno kompaktno priponsko polje potrebuje $|\Sigma|n + O(n)$ bitov pred stiskanjem zeloredkih bitnih polji. Ker pa je večina bitnih polji redkih, saj je število bitov z vrednostjo 0 veliko večje kot število bitov z vrednostjo 1, se lahko binta polja zakodira na bolj kompakten način z enim od algorimom stiskanja. Torje stisnjena bitna polja potrebujejo $nH_0(T)+O(n+|\Sigma|w)$ bitov za implementacijo kompaktenga priponskega polja. Potrebni čas za izračun funkcije $\Psi$ je $t_\Psi=O(1)$ ter potrebni čas za iskanje po priponskem polju in inverznem priponskem polju je $t_{SA}=lt_\Psi = t_\Psi O(\log{n}) = O(\log{n})$.


\paragraph{Polje najdaljših skupnih predpon:}
Zadnja podatkovna struktura, ki sestavlja kompaktno priponsko drevo, je polje najdaljših skupnih predpon oziroma $LCP$ polje. Namesto $LCP$ polja se bomo osredotočili na permutacijo $LCP$ polja imenovano $PLCP$, za katero velja relacija $PLCP[i]=LCP[SA^{-1}[i]]$ ali $LCP[i]=PLCP[SA[i]]$ ter jo je lažje predstaviti v kompaktnem zapisu \cite{Navarro2016, Sadakane2007}.


Permutacijo $PLCP[1,n-1]$ je lažje kompaktno predstaviti, saj je zaporedje $PLCP[i]+2i$ za vsak $i$ med $1$ in $n-1$ strogo naraščajoče. Zaporedje je strogo naraščajoče, saj je $PLCP[i+1]\ge PLCP[i]-1$. To lahko enostavno dokažemo, saj če je $PLCP[i]=0$ mora biti tudi $PLCP[i+1]=0$. Drugače pa obstajata $T[j,n]$ in $T[i,n]$ ter $T[i,n]<T[j,n]$, ki imata najdaljšo skupno predpono dolžine $PLCP[j]>0$. Potem ima $T[i+1,n]$ najdaljšo skupno predpono s $T[j+1,n]$ dolžine $PLCP[j]-1$ in vse pripone, ki so leksikografsko manjši od $T[j+1,n]$ in imajo skupno predpono, imajo najdaljšo skupno predpono s $T[j+1,n]$ vsaj dolžino $PLCP[j]-1$. Torej je $PLCP[j+1] = PLCP[j]-1$ in poslednično še vedno velja $PLCP[j]+2j<PLCP[j+1]+2(j+1)= PLCP[j]+2j+1$. Ker je $PLCP[n-1]+2(n-1)< 2n$, saj ima lahko $T[n-1,n]$ dolžino najdaljše skupne predpone s poljubno pripono 1, se lahko permutacijo $PLCP$ in posledično $LPC$ polje zapiše kot bitno polje $H[1,2n-1]$. Celica $H[j]=1$ za $j=PLCP[i]+2i$, kjer je $i$ med $1$ in $n-1$, sicer pa je $H[j]=0$. Pri tem bitno polje $H$ potrebuje dodatno podatkovno strukturo za $izbiro_1$ \cite{Navarro2016, Sadakane2007}.

Vrednost $LCP[i]$ se lahko pridobi v času $O(t_{SA})$. Vrednost je izračunana z uporabo formule $LCP[i]=izbira_1(H,SA[i])-2SA[i]$. Poizvedba potrebuje $O(t_{SA})$ časa, saj je potrebno izračunati vrednost $SA[i]$, ki se jo izračuna v $O(t_{SA})$ časa, ostale operacije pa potrebujejo konstanten čas. Če je uporabljena predhono predstavljena implementacija kompaktnega polja je čas poizvedbe v $LCP$ polj enak $O(\log{n})$. Prostorska zahtevnost polja $LCP$ je predstavljena s sledečo lemo:

\begin{lema}\label{lema:LCP}
 Podatkovna struktura $LCP$ potrebuje $2n+o(n)$ bitov za pravilno delovanje.
\end{lema}

\begin{proof}
    Podatkovna struktura $LCP$ je shranjena kot bitno polje $H[1,2n-1]$. Bitno polje $H$ potrebuje dodatnih $o(n)$ bitov za dodatno podatkovno strukturo, ki omogoča izvajanje operacije $izbira_1$ v konstantnem času. Ker je bitno polje $H$ dolžine $2n-1$ in potrebuje dodatnih $o(n)$ bitov, potem celotna podatkovna struktura za shraniti polje $LCP$ potrebuje $2n+o(n)$ bitov.
\end{proof}

Sedaj, ko so bile predstavljene vse podatkovne strukture, ki sestavljajo kompaktno priponsko drevo, je možno izračunati velikost celotnega kompaktnega priponskega drevesa. Ker obstaja več implementacij kompaktnega priponskega polja, ki se lahko uporabijo v kompaktnem priponskem drevesu, bo velikost priponskega drevesa vsebovala člen $|CSA|$, ki predstavlja velikost kompaktnega priponskega polja. Velikost kompaktnega priponskega drevesa je predstavljena v sledečem izreku:

\begin{izr}
    Podatkovna struktura kompaktno priponsko drevo nad besedo $T$ dolžine $n$ potrebuje $|CSA|+6n+o(n)$ bitov, pri čemer $|CSA|$ predstavlja velikost kompaktnega priponskega polja.
\end{izr}
\begin{proof}
    Ker obstajajo različne implementacije kompaktnega priponskega polja, je potrebnih $|CSA|$ bitov za shraniti kompaktno priponsko polje $SA$. Iz Leme \ref{lema:BP} sledi, da je potrebnih $4n+o(n)$ bitov za shraniti topologijo drevesa $\tau$. Iz Leme \ref{lema:LCP} pa sledi, da je potrebnih $2n+o(n)$ bitov za shraniti polje $LCP$.

    Torej je velikost kompaktnega priponskega drevesa $CST$ enaka $|CSA|+6n+o(n)$ bitov.
\end{proof}

Obstaja več različnih implementacij kompaktnega priponskega polja, zato Sadakane \cite{Sadakane2007} predlaga dve implementaciji. Prva predlagana implementacija je prostorsko učinkovita in uporablja kompaktno priponsko polje, ki so ga predlagali Grossi idr. \cite{Grossi2003}. 

\begin{posl}\label{pos:CSAnh}
Kompaktno priponsko drevo implementirano s pomočjo kompaktnega priponskega polja, ki potrebuje $|CSA|=nH_h+O(n\log\log{n} / \log_{| \Sigma|}{n})$ bitov, potrebuje $|CST|= nH_h+6n+O(n\log\log{n} / \log_{| \Sigma|}{n})$ bitov.    
\end{posl}

Druga predlagana implementacija pa je časovno učinkovita in uporablja kompaktno priponsko polje, ki ga sta ga predlagala Grossi in Vitter \cite{Grossi2000}. 
\begin{posl}\label{pos:CSAlog}
Kompaktno priponsko drevo implementirano s pomočjo kompaktnega priponskega polja, ki potrebuje $O(\frac{1}{\epsilon}n\log{| \Sigma|})$ bitov, potrebuje $|CST|=O(\frac{1}{\epsilon}n\log{| \Sigma|})$ bitov. Pri tem je $\epsilon$ poljubna konstanta, ki ima vrednost $0<\epsilon<1$.
\end{posl}

\begin{figure}[htb]
    \begin{subfigure}[t]{0.4\linewidth}
        
        \begin{subfigure}[t]{1\linewidth}        
            \includesvg[scale=0.8]{Slike/KOKOŠMcCreightS2.svg}
            \centering
            \subcaption*{}
            \label{fig:neKompaktnoDrevo}
        \end{subfigure}
        \begin{subfigure}[t]{1\linewidth}        
            \includesvg[scale=0.8]{Slike/KokosSA1.svg}
            \centering
            \subcaption*{}
            \label{fig:KompaktnoPolje}
        \end{subfigure}
        \centering
        \subcaption*{}
        \label{fig:neKompaktno}
    \end{subfigure}
    \begin{subfigure}[t]{0.6\linewidth}        
        \begin{subfigure}[t]{1\linewidth}  
            \hspace{0.25cm}      
            \includesvg{Slike/KompaktnoPriponskoPolje.svg}
            \subcaption*{}
            \label{fig:neKompaktnoDrevo}
        \end{subfigure}
        \begin{subfigure}[t]{1\linewidth}        
            \includesvg[scale=0.9]{Slike/TopologijaInLCPKompaktna.svg}
            \centering
            \subcaption*{}
            \label{fig:KompaktnoPolje}
        \end{subfigure}
        \subcaption*{}
        \label{fig:Kompaktno}
    \end{subfigure}
    \caption{Primer kompaktnega priponskega priponskega drevesa nad besedo \enquote{KOKOŠ$\$$}. Za lažje razumevanje sta na levi strani sliki priazana tudi ekvivalentno priponsko drevo ter ekvivalentno priponsko polje.} 
    \label{fig:CST}
\end{figure}

Primer kompaktnega priponskega drevesa je prikazan na Sliki \ref{fig:CST}. Poleg kompaktne predstavitve (na desni strani slike), so za lažjo berlivost in primerjavo prikazana tudi ekvivalentno priponsko drevo ter ekvivalentno priponsko polje (na levi strani slike). Prikazana podatkovna struktura potrebuje približno $12$ B (natančno potrebuje $97$ bitov). Pri tem so cela števila zapisana z $\log{n}$ biti. Ekvivalentno priponsko drevo potrebuje vsaj 15 B (oziroma 102 bita), priponsko polje pa potrebuje vsaj $4$ B (oziroma 33 bitov, pričemer priponsko polje, ki simulra priponsko drevo potrebuje približno 11 B oziroma 87 bitov). Pri tem nobena podatkovna struktura ne vklučuje dodatnih podatkovnih struktur za opravljanje operacij v konstantnem času. Ne kompaktne podatkovne strukture ponavadi ne uporabljajo zapisa celih števil z $O(\log{n})$ biti, kot je bilo uporabljeno v teh kalkulacijah, ampak se uporablja vsaj 1 B (8 bitov) ponavadi 4 B, kar poveča razliko.

\paragraph{Izboljšave:}
Prostorsko zahtevnost kompaktnega priponskega drevesa je možno še dodatno znižati na $|CSA|+o(n)$. Russo idr. \cite{Russo2008} so predstavili način kako doseči to prostorsko zahtevnost. To so dosegli, tako da so vzorčili $O(n/\delta)$ vozlišč, pri čemer $\delta=\omega(\log_{|\Sigma|}{n})$ in predstavlja faktor vzorčenja. Vzorčeno drevo potrebuje $o(n)=O(n/\log_{|\Sigma|}{n})$ bitov. Poleg vzorčenja topologije drevesa, so se znebili tudi $LCP$ polja. To jim je omogočalo znižanje prostorske zahtevnosti iz $|CSA|+6n+o(n)$ bitov na $|CSA|+o(n)$ bitov. Na ta račun pa se je dvignila časovna zahtevnost nekaterih operacij za $\omega(\log{n})$ krat, če se želi ohraniti $o(n)$ dodatnega prostora. Nekatere od teh operacji so predhodno potrebovale konstanten čas izvajanja.

Preden se nadaljuje z izgradnjo in poizvedbami nad $CST$, se lahko nadaljuje s primerom človeškega genoma iz začetka poglavja. S priponskim drevesom se potrebuje $144$ GB notranjega spomina za indeksirati celoten človeški genom. Z  uporabo kompaktnega priponskega drevesa, namesto priponskega drevesa, pa je potrebnih približno $3$ GB delovnega spomina za indeksirati celoten genom. Razlika v zahtevanem prostoru omogoča, da je lahko celotno kompaktno priponsko drevo shranjeno v delovnem spominu, za razliko od priponskega drevesa, za katerega to ni mogoče\footnote{Nekateri strežniki omogočajo večjo količino delavnega spomina, ki presega $144$ GB delovnega spomina. Večina potrošniških računalnikov pa še vedno uporablja med $8$ GB in $64$ GB delovnega spomina.}.

\section{IZGRADNJA}\label{sec:CSTizgradnja}
\import{.}{KompaktnaPredstavitev/Izgradnja}

\section{POIZVEDBE}\label{sec:CSTpoizvedba}
\import{.}{KompaktnaPredstavitev/Poizvedba}


