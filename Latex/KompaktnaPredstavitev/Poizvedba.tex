Kompaktna predstavitev priponskega drevesa je ekvivalenta priponskemu drevesu, zato mora podpirati iste poizvedbe nad besedo $T$, ki jih podpira priponsko drevo. Najbolj preprosta poizvedba je $\Prisotnost{T}{P}$. V kompaktnem priponskem drevesu se prisotnost vzorca $P$ išče s pomočjo vzvratnega iskanja (angl. \textit{Backward Search}) vzorca v priponskem polju $SA$. Vzvratno iskanje za predhodno predstavljeno kompaktno priponsko polje je prikazano na Algoritmu \ref{alg:prisotnostCST} in potrebuje $O(mt_\Psi)$ časa, da se izvrši. Za drugačno implementacijo kompaktnega priponskega polja, se morda mora nadomesti vrstico 6 v Algoritmu \ref{alg:prisotnostCST} z binarnim iskanjem nad $\Psi_c$ in zato je potrebno $O(m\log{n}t_\Psi)$ časa, oziroma se uporablja drugi algoritem za iskanje vzorca. Algoritem vrne interval v priponskem polju, zato je potrebno preveriti, ali je $[s,e]\ne[-1,-1]$, za kar je potrebno konstantno časa. Torej je potrebno $O(mt_\Psi)$ časa za preveriti prisotnost vzorca ali $O(m\log{n}t_\Psi)$ z uporabo binarnega iskanja. V primeru, da se uporabi predstavljeno kompaktno priponsko polje, pa se poizvedba izvrši v času $O(m)$. 

\begin{algorithm}[tb]

\Vhod{Kompaktno priponsko drevo $CST$, vzorec $P$}
\Izhod{Del priponskega polja, ki se začne z $P$}
\caption{Iskanje intervala v SA (del CST-ja), v katerem je prisoten vzorec $P$ \cite{Navarro2016}}\label{alg:prisotnostCST}    
{
    {$(s,e)=(C[P[m]]+1,C[P[m]+1])$}
    
    \Za{$i=(m-1)..1$}{
    
        \Ce{$s>e$}{\KwRet{$[-1,-1]$}}

        {$c=P[i]$}

        {$(s',e')=(rang_1(B_c,s-1)+1,rang_1(B_c,e))$}

        {$(s,e)=(C[c]+s',C[c]+e')$}
        
    }

    \Ce{$s>e$}{\KwRet{$[-1,-1]$}}

    {\KwRet{$[s,e]$}}
}

\end{algorithm}

Algoritem vzvratnega iskanje, ki je predstavljen na Algoritmu \ref{alg:prisotnostCST}, preveri prisotnost vzorca v priponskem polju, tako da se v koraku $k$ najde intreval pripon $[s,e]$ v priposnkem polju, ki se začne z znakom $P[k]$ in se nadaljuje z nizom $P[k+1:m]$. Prvi interval $[s,e]$ predstavlja vse pripone, ki se začnejo z znakom $P[m]$ \cite{Navarro2016}. Na primer če želimo preveriti $\Prisotnost{\text{\enquote{KOKOŠ\$}}}{\text{\enquote{KO}}}$, lahko to storimo z kompaktnim priposnkim drevesom prikaznem na Sliki \ref{fig:CST}. Začetni interval $[s,e]=[4,5]$, pri čemer je $C[P[m]]=C[O]$ oziroma je $C[3]$, saj je $S[3]=O$. V edinem koraku zanke je interval $[s',e']$ enak $[1,2]$, saj je $c=K$ oziroma je $c=2$, saj je $S[2]=K$. Postopek za Vrednost $s'$ je izračunana kot $\Rang{1}{B_2}{3}+1$ in s pomočjo Slike \ref{fig:CST} izračunamo, da je $s'=0+1=1$. Podobno lahko prevrimo tudi za $e'=\Rang{1}{B_2}{5}=2$. Zadnji korak izračuna je novi inerval $[s,e]=[C[2]+1,C[2]+2]=[1+1,1+2]=[2,3]$. Ker $s$ ni večji od $e$, vzvratno iskanje vrne $[2,3]$, kar pomeni, da je vzorec prisoten v besedi.

Naslednja poizvedba je $\SteviloPonovitev{T}{P}$, ki vrne število pojavov vzorca $P$ v besedilu. V kompaktnem priponskem drevesu, pa je operacija ponovno implementirana s pomočjo vzvratnega iskanja, prikazanega na Algoritmu \ref{alg:prisotnostCST}. Operacija $\SteviloPonovitev{T}{P}$ je implementiran kot razlika $e-s+1$, kjer $s$ predstavlja prvo pripono, ki se začne z vzorcem $P$, in $e$ je zadnja tako pripona, torej je razlika $e-s+1$ število pripon. Torej operacija ponovno potrebuje $O(mt_\Psi)$ oziroma $O(mt_\Psi\log{n})$ časa, da se izvrši.

Zadnja predstavljena poizvedba pa je $\SeznamPojavov{T}{P}$, ki vrne vse indekse pojavov vzorca $P$ v besedi $T$. V kompaktnem priponskem drevesu pa je poizvedba ponovno implementirana s pomočjo vzvratnega iskanja, prikazanega na Algoritmu \ref{alg:prisotnostCST}. Z vzvratnim iskanjem se naračuna interval v priponskem polju $SA[s,e]$. Za pridobitev položajev ponovitev vzorca v besedilu, je potrebno ustvariti seznam $[SA[s],SA[s+1],\dots,SA[e]]$, kar zahteva še dodatnih $e-s$ korakov, pri čemer vsak korak potrebuje $O(t_{SA})$ časa za se izvršit. Torej poizvedba $\SeznamPojavov{T}{P}$ potrebuje $O(mt_\Psi+occ\cdot t_{SA})$ oziroma $O(mt_\Psi \log{n}+occ\cdot t_{SA})$ časa. Z predhodno predstavljeno implementacijo kompaktnega priponskega drevesa pa je potrebno $O(m+occ\cdot\log{n})$ časa.

Iz implementacij poizvedb nad kompaktnim priponskim drevesom se lahko vidi, da so vse tri poizvedbe implementirane zgolj s pomočjo kompaktnega priponskega polja. Iz tega se lahko sklepa, da sta topologija drevesa $\tau$ in $LCP$ polje odvečni podatkovni strukturi. To je res zgolj za osnovne poizvedbe nad besedilom $T$, ki so lahko izvršene zgolj s uporabo kompaktnega priponskim polje v enakem času. Poizvedbe, kot so najdaljši ponavljajočise podniz, najdaljši palindrom, ki je implementirana s pomočjo priponskega drevesa konkatenacije obrata $T_z'$ vhodne besede $T_z$, $T=T_{z}\#T_{z}'\$$, in najdaljši skupni niz besed $T_1$ in $T_2$, ki je implementirana s pomočjo priponskega drevesa konkatenacje obeh besed $T=T_1 \# T_2\$ $. Na primer poizvedbe najdaljši ponavljajoči podniz je podniz $T[SA[i],SA[i]+globinaNiza(v)]$, pri čemer $i$ je skrajno levi list poddrevesa s korenom v notranjem vozlišču $v$ in velja, da je $LCP[i]$ največji element v $LCP$ polju, torej zahteva $O(nt_{SA})$ časa. Operacije $globinaNiza(v)$ ni potrebno naračunati, saj je enaka $LCP[i]$, torej se še vedno izvede $O(nt_{SA})$ časa, pri tem pa ni uporabljena topologija drevesa. V primeru, da želimo najti drugi najdaljši ponavljajoč se podniz $T[SA[j],SA[j]+globinaNiza(u)]$, pri čemer je $j$ skrajno levi list poddrevesa s korenom v notranjem vozlišču $u=sl(v)$. Za izračunat le tega pa je potrebo $O(nt_{SA}+t_\Psi)$ časa ter se uporabi vse tri podatkovne strukture kompaktnega priponskega drevesa \cite{Valimaki2007, Weiner1973, Navarro2016}.