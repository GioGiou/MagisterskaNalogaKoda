Podniz besedila $T$, ki se začne na $i$-tem in konča na $j$-tem znaku ($1\leq i \leq j \leq n$), se zapiše kot $T[i,j]$.  Podobno je definiran tudi vzorec $P$, ki ima dolžino $m$.  Vzorec je del besedila, če obstajata $i$ in $j$, za katera velja $P[1,m]=T[i,j]$.

Priponska drevesa so posebna implementacija številskega drevesa, pri čemer vsak list ne predstavlja posamične besede, ampak predstavlja posamezno pripono besede. Na ta način priponsko drevo ne predstavlja zgolj vhodne besede $T$ dolžine $n$, ki je sestavljeno iz znakov abecede $\Sigma$ in je shranjena na pomnilniku kot polje črk \texttt{T[1..n]}, ampak zakodira tudi njegovo strukturo. Zato se pogosto uporabi za indeksiranje besedila. Z uporabo indeksa je iskanje prisotnosti vzorca $P$ dolžine $m$ v besedilu primerljivo s KMP algoritmom, ki potrebuje $O(n+m)$ za preveriti ali se vzorec nahaja v besedilu. Prednost uporabe priponskega drevesa, namesto KMP algoritma, se izraža pri iskanju različnih vzorcev v istem besedilu $T$. Za najti več vzorcev v priponskem drevesu je potrebno: $O(n)$ časa za izgradnjo priponskega drevesa ter $O(m)$ časa za vsak iskani vzorec. Pri iskanju več vzorcev v besedilu $T$ algoritem KMP potrebuje $O(n+m)$ časa za vsak vzorec \cite{Gusfield1997,KMP}.


%
Priponsko drevo nad besedo, ki je niz nad abecedo $\Sigma$, definiramo na sledeči način \cite{Gusfield1997}:

\begin{defi}\label{def:priposkoDrevo}
    Priponsko drevo nad nizom $T$ dolžine $n$ je številsko drevo, ki zadošča sledečim zahtevam:
    \begin{enumerate}
        \item drevo ima natanko $n$ listov, od katerih je vsak list s številom med $1$ in $n$,
        \item vsako notranje vozlišče, razen korena, ima vsaj dva otroka,
        \item vsaka povezava predstavlja neprazni podniz besedila $T$,
        \item ne obstajata dve povezavi, ki se začneta v istem vozlišču in z istim znakom,
        \item niz pridobljen s konkatenacijo podnizov, na poti iz korena do lista $i$, predstavlja pripono $T[i,n]$ za vsak $i$, kjer je $1 \le i\le n$.        
    \end{enumerate}
\end{defi}

\begin{defi}
    Niz $\alpha\in\Sigma^*$ je podniz besedila $T$, ko obstaja tak $1\le i\le n$, za katerega velja, da je  $\alpha=T[i,i+|\alpha|]$. 
\end{defi}

\begin{defi}
    Konkatenacija nizov $\alpha\cdot\beta$ je niz $\gamma=\alpha\cdot\beta$, pri čemer je $\alpha=\gamma[1,|\alpha|]$ in $\beta=\gamma[|\alpha|+1,|\alpha|+|\beta|]$.
\end{defi}

%Podatkovna struktura se bisteveno poenostavi, ko mobena priponsa ni predpona druge pripone, to dosežemo z dodajanjem dolarja na konec besedila

%POsledično so pripone shranjene v listih

Primer priponskega drevesa besede »KOKOŠ« je prikazan na Sliki \ref{fig:PriponskoDrevo}. Znak »\$«, ki predstavlja konec besedila, omogoča razlikovanje med posameznimi priponami, če te niso že eksplicitno predstavljene v drevesu. V tem primeru je »\$« odveč, saj znak »Š« jasno določi vse pripone besede »KOKO«. Znak »\$« omogoča izgraditev drevesa, ki ima vse pripone eksplicitno predstavljene, in zadošča Definiciji \ref{def:priposkoDrevo}.

\begin{figure}[htb]
    \begin{center}
        \includesvg{Slike/McCreigov/KOKOŠMcCreightS.svg}
        \caption{Primer priponskega drevesa nad besedilom »KOKOŠ$\$$«.} 
        \label{fig:PriponskoDrevo}
    \end{center}
\end{figure}


%%Briši
Priponska drevesa so posebna implementacija dreves, zato je potrebno nad njimi definirati globino vozlišča, kar poenostavi iskanje in izgradnjo priponskega drevesa. Globina vozlišča je definirana na sledeči način:

\begin{defi}
   Globina vozlišča $D(v)$ je število vozlišč na poti od korena drevesa do vozlišča $v$. 
\end{defi}

Iz definicije .. V globino sta všteta tudi vozlišče $v$ ter koren drevesa.


Globina vozlišča $D(v)$ je uporabna zgolj, če vsaka povezava predstavlja en sam znak, kar vodi v kršitev druge zahteve Definicije  \ref{def:priposkoDrevo}. V primeru, da povezava predstavlja niz z dvema znakoma, bi se le ta spremenila v dve povezavi, kjer vsaka povezava predstavlja en znak, in vozlišče med njima z enim otrokom, kar krši drugo zahtevo Definicije \ref{def:priposkoDrevo}. Bolj smiselna definicija je besedna globina vozlišča:

\begin{defi}
    Črkovna globina vozlišča $Sd(v)$ je dolžina podniza pridobljenega s stikom vseh podnizov na povezavah na poti od korena drevesa do vozlišča $v$. 
\end{defi}

Primer razlike med globino vozlišča in besedno globino vozlišča se lahko vidi na Sliki \ref{fig:PriponskoDrevo}: vozlišča, do katerih se pride s povezavama »KO« in »O«, imata globino $1$. Besedna globina vozlišča, v katerega kaže povezava »KO«, je $2$ za razliko od vozlišča, v katerega kaže povezava »O«, ki ima besedno globino $1$. 

\section{IZGRADNJA}\label{sec:izgradnja}
\import{.}{PriponskaDrevesa/Izgradnja}