Priponska drevesa so posebna implementacija številskega drevesa, pri čemer vsak list predstavlja posamezno pripono besede. Na ta način priponsko drevo ne predstavlja zgolj vhodne besede $T$ dolžine $n$, ki je sestavljeno iz znakov abecede $\Sigma$ in je shranjena na pomnilniku kot polje črk \texttt{T[1..n]}, ampak zakodira tudi njegovo strukturo. Zato se pogosto uporabi za indeksiranje besede in posledično iskanja vzorcev v njej. Vzorec $P[1,m]$ se nahaja v besedi $T$, če obstaja podniz $T[i,j]$, za katerega velja $P[1,m]=T[i,j]$. Z uporabo indeksa je iskanje prisotnosti vzorca $P$ dolžine $m$ v besedi primerljivo s KMP algoritmom, ki potrebuje $O(n+m)$ časa za preveriti, ali se vzorec nahaja v besedi. Za razliko od KMP algoritma je prednost uporabe priponskega drevesa, da lahko iščemo različne vzorcev v isti besedi $T$. Za najti več vzorcev v priponskem drevesu je potrebno: $O(n)$ časa za izgradnjo priponskega drevesa ter $O(m)$ časa za vsak iskani vzorec. Pri iskanju več vzorcev v besedi $T$ algoritem KMP potrebuje $O(n+m)$ časa za vsak iskan vzorec \cite{Gusfield1997,KMP}.


%
Priponsko drevo nad besedo, ki je niz nad abecedo $\Sigma$, definiramo na sledeči način \cite{Gusfield1997}:

\begin{defi}\label{def:priposkoDrevo}
    Priponsko drevo nad nizom $T$ dolžine $n$ je številsko drevo, ki zadošča sledečim zahtevam:
    \begin{enumerate}
        \item drevo ima natanko $n$ listov oštevilčenih s števili med $1$ in $n$,
        \item vsako notranje vozlišče, razen korena, ima vsaj dva otroka,
        \item vsaka povezava predstavlja neprazni podniz besedila $T$,
        \item ne obstajata %dve
        povezavi, ki se začneta v istem vozlišču in z istim znakom,
        \item podniz, ki je pridobljen s stikom podnizov na poti od korena do lista $i$, predstavlja pripono $T[i,n]$ za vsak $i$, kjer je $1 \le i\le n$.        
    \end{enumerate}
\end{defi}



%Podatkovna struktura se bisteveno poenostavi, ko mobena priponsa ni predpona druge pripone, to dosežemo z dodajanjem dolarja na konec besedila

%POsledično so pripone shranjene v listih

Primer priponskega drevesa besede »KOKOŠ« je prikazan na Sliki \ref{fig:PriponskoDrevo}. Znak »\$«, ki predstavlja konec besedila, omogoča bistveno poenostavitev podatkovne strukture, saj tako ni nobena pripona predpona druge pripone. Posledično je vsaka pripona shranjena v listu priponskega drevesa. V primeru, predstavljenem na Sliki \ref{fig:PriponskoDrevo}, znak »\$« ne bi bil potreben, ker že znak »Š« jasno določi vse pripone besede »KOKOŠ«. Da zagotovimo, da vse so vse pripone shranjene v listih, se besedi na konec pripne znak »\$«.

\begin{figure}[htb]
    \begin{center}
        \includesvg{Slike/McCreigov/KOKOŠMcCreightS.svg}
        \caption{Primer priponskega drevesa nad besedilom »KOKOŠ$\$$«.} 
        \label{fig:PriponskoDrevo}
    \end{center}
\end{figure}


%%Briši
%Priponska drevesa so posebna implementacija dreves, zato se lahko nad njimi definira globina vozlišča. Le ta se lahko uporablja pri iskanju in izgradnji drevesa. Globina vozlišča je definirana na sledeči način:
%
%\begin{defi}\label{def:Globina}
%   Globina vozlišča $D(v)$ je število vozlišč na poti od korena drevesa do vozlišča $v$. 
%\end{defi}
%
%Iz Definicije \ref{def:Globina} je vidno, da sta v globino všteta tudi vozlišče $v$ ter koren drevesa. 
Ker povezave v priponskih drevesih predstavljajo podnize v besedi $T$, se lahko nad priponskimi drevesi definira tudi črkovna globina vozliča. Ta je definirana na sledeči način:


%%Globina vozlišča $D(v)$ je uporabna zgolj, če vsaka povezava predstavlja en sam znak, kar vodi v kršitev druge zahteve Definicije  \ref{def:priposkoDrevo}. V primeru, da povezava predstavlja niz z dvema znakoma, bi se le ta spremenila v dve povezavi, kjer vsaka povezava predstavlja en znak, in vozlišče med njima z enim otrokom, kar krši drugo zahtevo Definicije \ref{def:priposkoDrevo}. Bolj smiselna definicija je besedna globina vozlišča:

\begin{defi}
    Črkovna globina vozlišča $Sd(v)$ je dolžina podniza, pridobljenega s stikom vseh podnizov na povezavah na poti od korena drevesa do vozlišča $v$. 
\end{defi}

Primer razlike med globino vozlišča in črkovno globino vozlišča se lahko vidi na Sliki \ref{fig:PriponskoDrevo}: vozlišči, do katerih se pride s povezavama »KO« in »O«, imata globino $1$. Črkovna globina vozlišča, v katerega kaže povezava »KO«, je $2$, medtem ko ima vozlišče, v katerega kaže povezava »O«, črkovno globino $1$. 

\section{IZGRADNJA}\label{sec:izgradnja}
\import{.}{PriponskaDrevesa/Izgradnja}

\section{POIZVEDBE}\label{sec:poizvedba}
\import{.}{PriponskaDrevesa/Poizvedba}