Do sedaj so bile predstavljene različne implementacije indeksov besed. V tem poglavju so ti indeksi med seboj empirično primerjani. Testirajo se naslednje podatkovne strukture:
\begin{enumerate}
        \item priponsko drevo, %in
        \item kompaktno priponsko drevo, 
        \item priponsko polje in 
        \item priponsko polje z LCP poljem.
\end{enumerate}
Za vsako od naštetih podatkovnih struktur so izmerjene tri vrednosti: velikost podatkovne strukture med testiranjem, čas gradnje in čas za poizvedbe v podatkovni strukturi.

To poglavje je razdeljeno na tri dele. Najprej bomo predstavili testno okolje, metodo testiranja in testne besede. Nato bodo predstavljeni rezultati testov. V zadnjem delu poglavja pa bo predstavljena razprava rezultatov primerjav.

\section{EKSPERIMENTALNO OKOLJE}\label{sec:opis}
\import{.}{Operacije/Opis}


\section{REZULTATI}\label{sec:primerjava}
\import{.}{Operacije/Evalvacija}

\section{RAZPRAVA}\label{sec:razprava}
\import{.}{Operacije/Razprava}