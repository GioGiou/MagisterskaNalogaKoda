Do sedaj so bile teoretično predstavljene različne implementacije prionskih dreves. V tem poglavju pa bodo te implementacije priponskih dreves empirično primerjane med seboj. In sicer se bodo testirale sledeče podatkovne strukture:
\begin{enumerate}
        \item priponsko drevo, %in
        \item kompaktno priponsko drevo, 
        \item priponsko polje in 
        \item priponsko polje z LCP poljem.
\end{enumerate}
Za vsako od naštetih podatkovnih struktur bodo testirane tri strari: velikost podatkovne strukture, čas potreben za gradnjo in čas potreben za poizvedbe s podatkovno strukturo.

To poglavje bo razdeljeno na tri dele, in sicer prvo bo predstavljeno testno okolje, metoda testiranja ter vhodna besedila. Za tem bodo predstavljeni rezultati testov. V zadnjem delu poglavja pa bo narejena razprava rezultatov primerjav.

\section{EKASIPERIMENTALNO OKOLJE}\label{sec:opis}
\import{.}{Operacije/Opis}


\section{PRIMERJAVA}\label{sec:primerjava}
\import{.}{Operacije/Evalvacija}

\section{RAZPRAVA}\label{sec:razprava}
\import{.}{Operacije/Razprava}