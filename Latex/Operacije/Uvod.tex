V tem poglavju bodo analizirane implementacije operacij nad priponskimi drevesi iz Definicije \ref{def:AST}, časovna zahtevnost izgradnje priponskega drevesa in prostorska zahtevnost priponskega drevesa ter kako je implementirano iskanje vzorcev s pomočjo priponskih dreves.

Poglavje je razdeljeno na tri dele. V prvem delu so predstavljene teoretične razlike med implementacijami (različne implementacije kompaktnega priponskega polja ter implementacija priponskega drevesa). V drugem delu je predstavljena metoda empiričnega testiranja med kompaktnim priponskim drevesom in priponskim drevesom. V zadnjem delu pa so predstavljeni rezultati empirične primerjave.

%\section{TEORETIČNA ANALIZA}\label{sec:analiza}
%\import{.}{Operacije/Analiza}
%\import{.}{Operacije/AnalizaSamoTabele}


\section{EKASIPERIMENTALNO OKOLJE}\label{sec:opis}
\import{.}{Operacije/Opis}


\section{PRIMERJAVA}\label{sec:primerjava}
\import{.}{Operacije/Evalvacija}

\section{RAZPRAVA}\label{sec:razprava}
\import{.}{Operacije/Razprava}