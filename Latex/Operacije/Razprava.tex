Iz rezultatov testiranja opazimo, da priponsko drevo zasede 3,56 GB pomnilnika pri izgradnji za DNK sekvenco, ki je prikazan z viola barvo na Sliki \ref{fig:VelikostGraf}, ter 3,27 GB pomnilnika pri grandji za roman Na klancu, ki pa je prikazan z viola barvo na Sliki \ref{fig:VelikostGrafSLO}. Čeprav priponsko drevo potrebuje manj pomnilnika kot ga razpolaga sistem (4 GB), ga operacijski sistem zasede 1,3 GB, torej je potrebno nekatere strani shraniti na Swap razdelku. Premalo pomnilnika je tudi razlog za omejitev testov na besede dolžine 1024000 znakov, saj je računalnik potreboval več kot 5 minut za zgraditi eno priponsko drevo. Ti rezultati so v skladu z razliko v hitrosti dostopa do podatkov med notranjim pomnilnikom in zunanjim pomniliko, ki je v tem primeru trdi disk. Trdi disk potrebuje 7-krat več časa za dostopati do zaporedno zapisani podatko, kar se zgodi pri ne polno zasedenem Swap razdelku med testiranjem priponskega drevesa za besedo dolžine 1024000. Meritev hitrosti branja in pisanja med različnimi tip pomnilnika je izmeril tudi Jacobs \cite{Jacobs2009}, ki je prišel do istih izmerjenih rezultatov. Ostali indeksi zasedejo približno isto prostora. To pa je povzročilo problem pri natančnem razlikovanju prostora dveh podatkovnih struktur s pomočjo pomnilniškega profilerja.

Naslednje vprašanje, ki ga želimo odgovoriti, je vpliv tipa abecede vhodne besede. Iz samega testiranja ni razvidnih bistvenih razlik, pri času potrebnem za gradnjo indeksov ter zasedenem prostoru. Pri tem pa se pojavi razlika v rezultatih iskanja vzorca dolžine 5 znakov s pomočjo priponskega drevesa. Pri testiranju z vhodno besedo v naravnem jeziku je vzorec najverjetneje prisoten v besedi za besede dolžine vsaj 16000 znakov, glede na skok v potrebovanem času, ki je viden na prvem grafu na Sliki \ref{fig:IskanjeGrafSLO} označeno z viola barvo.

Zadnje vprašanje, ki ga želimo odgovoriti, pa je o razlikah med podatkovnimi strukturami ter kdaj se uporabi posamična struktura. Prvo stavr, ki jo vidimo, je približno ista časovna zahtevnost poizvedb med priponskim poljem brez in z dodatno LCP strukturo. Najbolj verjetni razlog je predčasna prekinitev bisekcije, kar zniža potrebi čas. Opazi se tudi dodatni čas potreben za izgradnjo LCP polja. Najbolj verjetni razlogi so neučinkoviti rekurzivni klici. Če pa se vrnemo nazaj na vprašanje, kdaj se uporabi posamični indeks, je potrebno ločiti tri scenarije: 
\begin{enumerate}
        \item nizko število poizvedb, ki ne omogočajo amortizacije izgradnje,
        \item visoko število poizvedb, ki omogočajo amortizacije izgradnje, in indeks je lahko v celoti shranjen v delovnem pomnilniku,
        \item visoko število poizvedb, ki omogočajo amortizacije izgradnje, ampak indeks ni v celoti shranjen v delovnem pomnilniku.
\end{enumerate} 
Podatkovna struktura primerna za prvi scenarij je priponsko polje, saj se ga najhitreje izgradi (Sliki \ref{fig:IzgradnjaGraf} in \ref{fig:IzgradnjaGrafSLO}) ter zasede malo prostora. Pri tem se iskanje lahko pospeši z uporabo LCP polja. Za drugi scenarij je najbolj primerna podatkovna struktura priponsko drevo, saj od testiranih indeksov je najhitrejši za iskanje vzorcev v besedi, če nismo prostorsko omejeni. Če pa smo prostorsko omejeni, kot smo v scenariju tri, pa uporabimo kompaktno prionsko drevo, saj potrebuje malo prostora, čas izgradnje je nizek ter je drugi najhitrejši indeks za iskanje vzorcev v besedi.

Dobljene rezultate lahko primrejamo tudi z rezultati od Välimäki idr. \cite{Valimaki2007}. Moji rezultati sovpadajo z rezultati, ki so jih dobili, za DNK sekvenco. Pri tem pa ne moremo primerjati rezultate za naravni jeziki, saj sem jaz uporabljal kot vhodno besedo besedilo v slovenščini, oni pa so uporabljali besedilo v angleščini.