Iz rezultatov testiranja opazimo, da priponsko drevo zasede 3,56 GB pomnilnika pri izgradnji za DNK sekvenco, ki je prikazan z viola barvo na Sliki \ref{fig:VelikostGraf}, ter 3,27 GB pomnilnika pri gradnji za roman Na klancu, ki je prikazan z viola barvo na Sliki \ref{fig:VelikostGrafSLO}. Čeprav priponsko drevo potrebuje manj pomnilnika kot ga razpolaga sistem (4 GB), ga operacijski sistem zasede 1,3 GB, torej je potrebno nekatere strani shraniti na Swap razdelku. Premalo pomnilnika je tudi razlog za omejitev testov na besede dolžine 1024000 znakov, saj je računalnik potreboval več kot 5 minut za gradnjo enega priponskega drevesa. Ti rezultati so v skladu z razliko v hitrosti dostopa do podatkov med notranjim pomnilnikom in zunanjim pomnilnikom, ki je v tem primeru trdi disk. Trdi disk potrebuje 7-krat več časa za dostopati do zaporedno zapisanih podatkov, kar se zgodi pri ne polno zasedenem Swap razdelku med testiranjem priponskega drevesa za besedo dolžine 1024000. Meritev hitrosti branja in pisanja med različnimi tipi pomnilnika je izmeril tudi Jacobs \cite{Jacobs2009}, ki je prišel do istih izmerjenih rezultatov. Ostali indeksi zasedejo približno isto prostora. To pa je povzročilo problem pri natančnem razlikovanju prostora dveh podatkovnih struktur s pomočjo pomnilniškega profilerja.

Naslednje vprašanje, ki ga želimo odgovoriti, je vpliv tipa abecede vhodne besede. Iz samega testiranja ni razvidnih bistvenih razlik pri času, potrebnem za gradnjo indeksov ter zasedenem prostoru. Pri tem pa se pojavi razlika v rezultatih iskanja vzorca dolžine 5 znakov z uporabo priponskega drevesa. Pri testiranju z vhodno besedo v naravnem jeziku je vzorec najverjetneje prisoten v besedi za besede dolžine vsaj 16000 znakov, glede na skok v potrebovanem času, ki je viden na prvem grafu na Sliki \ref{fig:IskanjeGrafSLO} (označeno z viola barvo).

Zadnje vprašanje, ki ga želimo odgovoriti, je o razlikah med podatkovnimi strukturami ter kdaj se uporabi posamična struktura. Prvo stvar, ki jo vidimo, je približno ista časovna zahtevnost poizvedb med priponskim poljem brez in z dodatno LCP strukturo. Najbolj verjeten razlog je predčasna prekinitev razpolavljanja, kar zniža čas poizvedbe. Opazi se tudi dodatni čas, potreben za izgradnjo LCP polja. Najbolj verjeten razlog je neučinkoviti rekurzivni klici. Če pa se vrnemo nazaj na vprašanje, kdaj se uporabi posamični indeks, je potrebno ločiti tri scenarije: 
\begin{enumerate}
        \item nizko število poizvedb, ki ne omogočajo amortizacije izgradnje,
        \item visoko število poizvedb, ki omogočajo amortizacije izgradnje, pri čemer indeks je lahko v celoti shranjen v delovnem pomnilniku,
        \item visoko število poizvedb, ki omogočajo amortizacije izgradnje, pri čemer indeks ni v celoti shranjen v delovnem pomnilniku.
\end{enumerate} 
Podatkovna struktura, primerna za prvi scenarij, je priponsko polje, saj se ga najhitreje izgradi (Sliki \ref{fig:IzgradnjaGraf} in \ref{fig:IzgradnjaGrafSLO}) ter zasede malo prostora. Pri tem se iskanje lahko pospeši z uporabo LCP polja. Za drugi scenarij je najbolj primerna podatkovna struktura priponsko drevo, saj je od testiranih indeksov najhitrejši za iskanje vzorcev v besedi, če nismo prostorsko omejeni. Če pa smo prostorsko omejeni, kot smo v scenariju tri, pa uporabimo kompaktno priponsko drevo, saj potrebuje malo prostora, čas izgradnje je kratek ter je drugi najhitrejši indeks za iskanje vzorcev v besedi.

Dobljene rezultate lahko primerjamo tudi z rezultati od Välimäki idr. \cite{Valimaki2007}. Naši rezultati sovpadajo z rezultati, ki so jih dobili, za DNK sekvenco. Pri tem pa ne moremo primerjati rezultate za naravni jeziki, saj smo uporabljali kot vhodno besedo besedilo v slovenščini, oni pa so uporabljali besedilo v angleščini.