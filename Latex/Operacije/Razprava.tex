Iz rezultatov testiranja opazimo, da največje priponsko drevo zasede 3,56 GB pomnilnika za \DNK, ki je prikazan z viola barvo na Sliki \ref{fig:VelikostGraf}, ter 3,27 GB pomnilnika za \NK, ki je prikazan z viola barvo na Sliki \ref{fig:VelikostGrafSLO}. Čeprav priponsko drevo potrebuje manj pomnilnika kot ga razpolaga sistem (4 GB), ga operacijski sistem zasede 1,3 GB, torej je potrebno nekatere strani shraniti na Swap prostoru. Premalo pomnilnika je tudi razlog za omejitev testov na besede dolžine 1024000 znakov, saj je računalnik potreboval več kot 5 minut za gradnjo enega priponskega drevesa. Ti rezultati so v skladu z razliko v hitrosti dostopa do podatkov med notranjim pomnilnikom in zunanjim pomnilnikom, ki je v tem primeru trdi disk. Trdi disk potrebuje 7-krat več časa za dostopati do zaporedno zapisanih podatkov, kar se zgodi pri ne polno zasedenem Swap prostoru med testiranjem priponskega drevesa za besedo dolžine 1024000. Meritev hitrosti branja in pisanja med različnimi tipi pomnilnika je izmeril tudi Jacobs \cite{Jacobs2009}, ki je prišel do istih izmerjenih rezultatov. Ostali indeksi zasedejo približno isto prostora. To pa je povzročilo problem pri natančnem razlikovanju prostora dveh podatkovnih struktur s pomočjo pomnilniškega profilerja.

Naslednje vprašanje, na katerega želimo odgovoriti, je vpliv tipa abecede vhodne besede. Iz samega testiranja ni razvidnih bistvenih razlik pri času, potrebnem za gradnjo indeksov ter zasedenem prostoru. Podobno ni vidna nobena bistvena razlika v rezultatih iskanja vzorcev. Torej tip abecede vhodne besede ne vpliva na velikost in hitrost iskanja in gradnje podatkovnih struktur.

Zadnje vprašanje, na katerega želimo odgovoriti, je, kdaj se uporabi posamična struktura. Prvo stvar, ki jo vidimo, je približno ista časovna zahtevnost poizvedb med priponskim poljem brez in z dodatno LCP strukturo. Najbolj verjeten razlog je predčasna prekinitev razpolavljanja, kar zniža čas poizvedbe. Opazi se tudi dodatni čas, potreben za izgradnjo LCP polja. Najbolj verjeten razlog za veliko časovno razliko so neučinkoviti rekurzivni klici. Če pa se vrnemo nazaj na vprašanje, kdaj se uporabi posamični indeks, lahko ločimo tri načine uporabe. 
\paragraph{Nizko število poizvedb, ki ne omogočajo amortizacije izgradnje.} 
Najbolj primerna struktura za tak tip uporabe je priponsko polje, saj se ga najhitreje izgradi (Sliki \ref{fig:IzgradnjaGraf} in \ref{fig:IzgradnjaGrafSLO}) ter zasede malo prostora. Pri tem se iskanje lahko pospeši z uporabo LCP polja. 

\paragraph{Visoko število poizvedb, ki omogočajo amortizacije izgradnje, pri čemer je indeks lahko v celoti shranjen v delovnem pomnilniku.} Za tak tip uporabe je najbolj primerna podatkovna struktura priponsko drevo, saj je od testiranih indeksov najhitrejši za iskanje vzorcev v besedi, če nismo prostorsko omejeni.

\paragraph{Visoko število poizvedb, ki omogočajo amortizacije izgradnje, pri čemer indeks ni v celoti shranjen v delovnem pomnilniku.} Ko pa smo prostorsko omejeni, pa uporabimo kompaktno priponsko drevo, saj potrebuje malo prostora, čas izgradnje je kratek ter je drugi najhitrejši indeks za iskanje vzorcev v besedi.

%Podatkovna struktura, primerna za prvi scenarij, je priponsko polje, saj se ga najhitreje izgradi (Sliki \ref{fig:IzgradnjaGraf} in \ref{fig:IzgradnjaGrafSLO}) ter zasede malo prostora. Pri tem se iskanje lahko pospeši z uporabo LCP polja. Za drugi scenarij je najbolj primerna podatkovna struktura priponsko drevo, saj je od testiranih indeksov najhitrejši za iskanje vzorcev v besedi, če nismo prostorsko omejeni. Če pa smo prostorsko omejeni, kot smo v scenariju tri, pa uporabimo kompaktno priponsko drevo, saj potrebuje malo prostora, čas izgradnje je kratek ter je drugi najhitrejši indeks za iskanje vzorcev v besedi.

Dobljene rezultate lahko primerjamo tudi z rezultati od Välimäki idr. \cite{Valimaki2007}. Naši rezultati sovpadajo z rezultati, ki so jih dobili za \DNK. Pri tem pa ne moremo primerjati rezultate za naravni jezik, saj smo uporabljali kot vhodno besedo besedilo v slovenščini, oni pa so uporabljali besedilo v angleščini.