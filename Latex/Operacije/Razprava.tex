Iz vseh izvedenih testiranj nad priponskimi drevesi, ki so bila izgrajena nad besedilom v naravnem jeziku, se lahko sklepa, da je bolje uporabiti priponsko drevo za besedila do dolžine 4000 znakov, saj je potrebnega manj prostora in časa za izgradnjo drevesa, ter poizvedbe potrebujejo manj časa, da se izvršijo. Če pa je priponsko drevo zgrajeno nad daljšim besedilo in je lahko v celoti shranjeno v delovnem spominu, potem je potrebno preveriti ali je število iskanih vzorcev dovolj veliko, da vsaka poizvedba amortizira čas izgradnje besedila. Število vzorcev mora biti vsaj $O(n)$, da se lahko amortizira čas izgradnje. Če ni dovolj vzorcev za to storiti, potem je bolje uporabiti kompaktno priponsko drevo. Le to omogoča malo počasnejše iskanje vzorcev, ampak za daljša besedila od 4000 znakov sta potreben čas za izgradnjo in prostor na pomnilniku bistveno nižja od prostora in časa izgradnje priponskega drevesa.

Iz rezultatov se lahko tudi opazi, da se priponskemu drevesu, ki je delno shranjeno na \verb|swap| razdelku, bistveno poslabša čas potrebnem za izgradnjo in iskanje vzorcev v besedilu. Po tem takem ni priporočljivo uporabljati priponska drevesa, ki morajo biti shranjena izven delovnega spomina.

Pri primerjavi rezultatov obeh testiranj, testiranje nad DNK sekvenco in testiranje nad besedilom iz naravnega jezika, se lahko opazi, da so si rezultati zelo podobni. Edina bistvena razlika med obema testiranjema je skok v potrebnem času pri iskanju daljših vzorcev ter vzorcev dolžine $O(\log{n})$ v besedilu, ki je zapisano v naravnem jeziku. Najbolj verjeten razlog je prisotnost vzorca v besedilu, ki poveča časovno zahtevnost iskanja. Torej iz testiranja se lahko ugotovi, da ni nobene razlike med časom potrebnim za poizvedbo in izgradnjo ter prostorsko zahtevnostjo priponskega drevesa (oziroma kompaktnega priponskega drevesa), glede na vhodno besedilo priponskega drevesa. 

Obstaja pa izmerljiva razlika med priponskimi drevesi, ki so v celoti shranjeni v delovnem spominu, ter tistimi, ki so deloma shranjeni na \verb|Swap| razdelku. Izmerjena razlika se  pojavi tudi v primerjavi, ki so jo izdelali Välimäki idr. \cite{Valimaki2007}. Izmerjena razlika je skladna z razliko v času branja zaporednih podatkov med trdim diskom in notranjim spominom, ki je približno 7-krat počasnejši, kar je izmeril Jacobs \cite{Jacobs2009}. Iz ugotovitev od Jacobsa \cite{Jacobs2009} in Välimäki idr. \cite{Valimaki2007} je lahko pojasnjena velika časovna zahtevnost pri izgradnji priponskega drevesa velikost 2048000 znakov.