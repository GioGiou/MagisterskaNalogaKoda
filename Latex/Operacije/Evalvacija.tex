V tem podpoglavju bodo predstavljeni rezultati testiranja različnih indeksov besede, ki je bil izdelan z metodo in računalnikom predstavljenim v podpoglavju \ref{sec:opis}. Metoda meri prostor, ki ga zasede indeks, čas potreben za izgradnjo indeksov in čas potreben za različne poizvedbe z indeksom. Meritve lahko ločimo na dva dela, in sicer: meritvi vezani na gradnjo, ki sta čas izgradnje in velikost indeksa besede, ter poizvedbe z indeksom. Zato bo sledeče podpoglavje tudi razdeljeno na dva dela. V vsakem delu bodo ločeno predstavljeni rezultati testiranja za vsako testno besedo iz Tabele \ref{tab:besedila}. V podpoglavju \ref{sec:razprava} pa bo narejena bolj podrobna razprava o rezultatih testiranja, kot samo opis rezultatov.

\subsection{Gradnja}
Testiranje se začne z izgradnjo indeksov. Pri tem smo merli čas potreben za izgradnjoj indeksov za besede različnih dolžin. Enkrat ko je indeks izgrajen, se lahko izmeri tudi njegova velikost na pomnilniku.

\paragraph{Prostorska zahtevnost podatkovne strukture.}

In če se začne lih z velikostjo indeksov na pomnilniku, sto te meritve prikazane na Sliki \ref{fig:VelikostGraf} za DNK sekvenco in na Sliki \ref{fig:VelikostGrafSLO} za roman Na klancu. Iz obeh testiranj se vidi, da je prostorska zahtevnost priponskega drevesa večja, kot prostorska zahtevnost ostalih indeksov. Pri tem vidimo tudi, da prostorska zahtevnost linearno raste z velikostjo besede, za vse 4 testne primere.

\begin{figure}[htb]
    \centering
    \includesvg[inkscapelatex=false,width=\textwidth]{Slike/velikostDrecvesaNovPC.svg}
    \caption{Graf velikosti indeksov izgrajenih iz besed različni velikosti. Vhodna beseda je DNK sekvenca.} 
    \label{fig:VelikostGraf}
\end{figure}

Iz rezultatov testiranja za DNK sekvenco, Slika \ref{fig:VelikostGraf}, vidimo, da vse podatkovne strukture linearno rastejo. Pri tem pa priponsko polje (označeno z zeleno barvo), priponsko polje z LCP poljem (označenim z modro barvo) in kompaktno priponsko drevo (označeno z rdečo barvo) potrebujejo bistveno manj prostora, kot ga potrebuje priponsko drevo (označen z viola barvo).

\begin{figure}[tb]
    \centering
    \includesvg[inkscapelatex=false,width=\textwidth]{Slike/velikostDrecvesaNovPCSLO.svg}
    \caption{Graf velikosti indeksov izgrajenih iz besed različni velikosti. Vhodna beseda je roman Na klancu.} 
    \label{fig:VelikostGrafSLO}
\end{figure}

Podobno se lahko vidi tudi za roman Na klancu, Slika \ref{fig:VelikostGrafSLO}. Vse podatkovne strukture potrebujejo liearno rastejo z dolžino vhodne besede ter priponsko drevo (označen z viola barvo) je potrebuje bistveno več prostora. Pri tem pa se vidi, da je razlika v zasedenem prostoru med priponskim drevesom in ostalimi indeksi nižja in konstantna skozi celotno izvajanje testa, za razliko od testa nad DNK sekvenco.



\paragraph{Časovna zahtevnost gradnje.} 

Zdaj, ko poznamo velikost indeksoćv, pa si lahko pogledamo čas, ki ga potrebujemo za graditi indekse. Rezultati so prikazani na Sliki \ref{fig:IzgradnjaGraf} za DNK sekvenco ter na Sliki \ref{fig:IzgradnjaGrafSLO} za roman Na klancu. V obeh primerih vidimo, da čas gradnje indeksov raste linearno z dolžino besede. V obeh testnih primerih je priponsko polje (označeno z zelen barvo) se zgradi najhitreje. Iz slik se zdi, da kompaktno priponsko drevo (označeno z rdečo barvo) potrebuje konstantno časa za izgradnjo, ampak iz rezultatov testiranja se vidi majhno rast, ki ni vidna na slikah.

\begin{figure}[htb]
    \centering
    \includesvg[inkscapelatex=false,width=\textwidth]{Slike/izgradnjaDrecvesaNovPC.svg}
    \caption{Graf prikazuje čas izgradnje indeksov besde za različne dolžine vhodnih besed. Vhodna beseda je DNK sekvenca.} 
    \label{fig:IzgradnjaGraf}
\end{figure}

\begin{figure}[htb]
    \centering
    \includesvg[inkscapelatex=false,width=\textwidth]{Slike/izgradnjaDrecvesaNovPCSLO.svg}
    \caption{Graf prikazuje čas izgradnje indeksov besde za različne dolžine vhodnih besed. Vhodna beseda je roman Na klancu.} 
    \label{fig:IzgradnjaGrafSLO}
\end{figure}

Časi izmerjeni pri testiranju z DNK sekvenco, prikazani na Sliki \ref{fig:IzgradnjaGraf}, za indeksa priponsko drevo (označene z viola barvo) in priponsko polje z dodanimi LCP poljem (označen z modro barvo) so približno stokrat daljši od časov izgradnje priponskih polj (označenih z zeleno barvo). Opazi se tudi, da test priponskih polji potrebuje manj kot 1 ms za zgraditi priponska polja za besede krajše od vključno 4000 znakov oziroma prve štiri dolžine besed.



Podobni rezultati so izmerjeni tudi za roman Na klancu, prikazani na Sliki \ref{fig:IzgradnjaGrafSLO}. Priponsko drevo (označene z viola barvo) in priponsko polje z dodanimi LCP poljem (označen z modro barvo) potrebujeta približno stokrat daljši čas kot priponsko polje (označenih z zeleno barvo), da se zgradi. Opazi se tudi, da test priponskih polji potrebuje manj kot 1 ms za zgraditi priponska polja za besede krajše od vključno 4000 znakov oziroma prve štiri dolžine besed.


\subsection{Poizvedbe}
Zatem ko smo indekse besed zgradili, jih lahko uporabimo za poizvedovati v besedah. Rezultati testiranja so prikazani na Sliki \ref{fig:IskanjeGraf} za poizvedbe v DNK sekvenci ter na Sliki \ref{fig:IskanjeGrafSLO} za poizvedbe v romanu Na klancu. Na obeh slikah so prikazani grafi rezultatov poizvedb za vzorce dolžine 5 znakov (prvi graf), vzorce dolžine 50 znakov (drugi graf), vzorce dolžine 500 znakov (tretji graf) in vzorce dolžine $\log{n}$ znakov (četrti graf).

\begin{figure}[htb]
    \centering
    \includesvg[inkscapelatex=false,width=\textwidth]{Slike/IskanjeNovPC.svg}
    \caption{Graf prikazuje čas iskanja vzorcev različnih dolžin v različnih indeksih besede. Vhodna beseda je DNK sekvenca.} 
    \label{fig:IskanjeGraf}
\end{figure}

\begin{figure}[htb]
    \centering
    \includesvg[inkscapelatex=false,width=\textwidth]{Slike/IskanjeNovPCSLO.svg}
    \caption{Graf prikazuje čas iskanja vzorcev različnih dolžin v različnih indeksih besede. Vhodna beseda je roman Na klancu.} 
    \label{fig:IskanjeGrafSLO}
\end{figure}

\paragraph{Časovna zahtevnost poizvedbe za vzorec dolžine 5.}

Za DNK sekvenco, prikazano na Sliki \ref{fig:IskanjeGraf} (prvi graf), se vidi, da je iskanje v kompaktnem priponskem drevesu (označeno z rdečo barvo) najhitrejše. V priponskih drevesih (označenih z viola barvo) in kompaktnih priponskih drevesih je čas iskanja neodvisen od dolžine vhodne besed. Iskanja s priponskimi polji (označenih z zeleno barvo in z modro barvo) pa se večata za daljše besede.

Podobno za roman Na klancu, prikazano na Sliki \ref{fig:IskanjeGrafSLO} (prvi graf), so še bolj vidne razlike med implementacijami priponskih dreves ter med implementacijami s priponskimi polji. Pri iskanju vzorca v besedi dolžine 1024000 s pomočjo priponskega drevesa se skok v potrebovanem času, saj so nekateri strani pomnilnika morale biti ponovno naložene v pomnilnik iz Swap razdelka.

\paragraph{Časovna zahtevnost poizvedbe za vzorec dolžine 50.}

Ko se velikost vzorca poveča na 50 znakov, se čas poizvedbe za DNK sekvenco, prikazano na Sliki \ref{fig:IskanjeGraf} (drugi graf), loči na dva dela, in sicer na hitrejše iskanje s priponskimi drevesi (označenih z viola barvo) in kompaktnimi prionskimi drevesi (označeno z rdečo barvo) ter na počasnejše iskanje s priponskimi polji (označenih z zeleno barvo in z modro barvo). Pri tem se opazi tudi premik strani iz Swap razdelka na notranji pomnilnik, pri testiranju priponskega drevesa besede dolžine 1024000.

Rezultati testiranja romana Na klancu, prikazano na Sliki \ref{fig:IskanjeGrafSLO} (drugi graf), prikažejo podobne rezultate kot DNK sekvenca. Visi se lahko, da je iskanje s priponskimi drevesi oziroma kompaktnimi priponskimi drevesi desetkrat hitrejše kot s priponskimi polji. Opazi se lahko tudi vpliv premika strani, kot se je opazil drugih že opravljenih testih. 

\paragraph{Časovna zahtevnost poizvedbe za vzorec dolžine 500.}

Če nadaljujemo na vzorce dolžine 500 znakov, so za DNK sekvenco, prikazani na Sliki \ref{fig:IskanjeGraf} (tretji graf), podobni rezultatom iskanja vzorcev dolžine 50 znakov. Pri tem je iskanje najhitrejše v priponskih drevesih (označenih z viola barvo), razen za daljše besede, ki ne stoji v celoti v delovnem pomnilniku. Pri tem pa se ne vid tako drastičnega poslabšanja kot za vzorce dolžine 50 znakov.

Rezultati testiranja romana Na klancu so prikazano na Sliki \ref{fig:IskanjeGrafSLO} (tretji graf). Ter lahko opazimo, da so približno enako kot za DNK sekvenco in za vzorce dolžine 50 znakov. Za razliko od DNK sekvence se v tem primeru izrecno vidi čas potreben za zamenjavo strani v notranjem pomnilniku na iskanje s priponskim drevesom (označenim z viola barvo). Vidi se tudi, da priponsko drevo je za vse teste razen za zadnja dva testa najhitrejši način iskanja vzorcev v besedah.

\paragraph{Časovna zahtevnost poizvedbe za vzorec dolžine $\log{n}$.}
Zadnja izdelana testa sta prisotnost vzorcev dolžine $\log{n}$ s pomočjo indeksov. Za DNK sekvenco, prikazano na Sliki \ref{fig:IskanjeGraf} (četrti graf), se vidi, da za krajše besede (besed, kiniso daljše od 1000 znakov) vse štiri podatkovne strukture potrebujejo približno isto časa. Za daljše besede pa je priponsko drevo (označeno z viola barvo) najhitrejši način iskanja. To velja zgolj dokler drevo ne preraste velikost notranjega pomnilnika in potrebuje zamenjave strani, kar negativno vpliva na rezultat iskanja.

Za roman Na klancu, prikazano na Sliki \ref{fig:IskanjeGrafSLO} (četrti graf), pa so rezultati podobni. Pri tem pa je priponsko drevo (označeno z viola barvo), konstantno najhitrejši način iskanja. To velja zgolj dokler drevo ne preraste velikost notranjega pomnilnika in potrebuje zamenjave strani, kar negativno vpliva na rezultat iskanja, kar je v tem testu bolj jasno vidno kot v DNK sekvenci.
