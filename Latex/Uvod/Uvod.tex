Pogosti problem pri obdelavi daljših besed je iskanje vzorcev v njih. V procesiranju naravnih jezikov vlogo vzorca pogosto prevzame iskana beseda naravnega jezika in vlogo vhodne besede pa prevzame besedilo, v katerem želimo iskati besedo. Medtem ko v bioinformatiki vlogo vzorca prevzame specifični gen ali druga DNK sekvenca, pri tem pa je vhodna beseda genomu. 

Ko v besedi $T$ dolžine $n$ iščemo zgolj en vzorec $P$ dolžine $m$, se za iskanje lahko uporabi Knuth–Morris–Pratt (KMP) algoritem s časovno zahtevnostjo $O(n+m)$. V primeru, da se v besedi išče več vzorcev, je smiselno nad besedo zgraditi indeks, recimo priponsko drevo, ki je lahko zgrajeno v času $O(n)$, in nato iskati vzorce. Časovna zahtevnost iskanja vzorca v priponskem drevesu je sorazmerna dolžini vzorca, $O(m)$. Ko je vzorec prisoten v besedi, se vsaj ena pripona besede začne z iskanim vzorcem. 
Torej se vzorec zagotovo nahaja na začetku sprehoda iz korena proti listom priponskega drevesa, zgrajenega nad besedo. 
Peter Weiner je kot prvi predstavil priponska drevesa leta 1973 \cite{Weiner1973} in jih uporabil v algoritmu za iskanje vzorcev v besedi. Časovna zahtevnost podanega algoritma za iskanje vzorcev v besedi je $O(m)$. Velikost priponskega drevesa je sorazmerna z dolžino vhodne besede, torej pri velikih besedah preraste velikost delovnega pomnilnika, kar močno vpliva na hitrost iskanja. Rešitev te težave je kompaktna predstavitev priponskega drevesa \cite{Navarro2016}.

Namen magistrske naloge je preučevanje vpliva kompaktne predstavitve na izgradnjo priponskih dreves in na operacije nad njimi. V nalogi bodo predstavljene časovne zahtevnosti osnovnih operacij in poizvedb ter izgradnje priponskih dreves, priponskih polj in kompaktne predstavitve priponskih dreves. Izdelana bo tudi empirična primerjava različnih implementacij priponskih dreves in drugih indeksov besed.

\section{STRUKTURA}\label{sec:struktura}

Magistrska naloga je razdeljena na pet delov. V drugem poglavju bodo predstavljene notacije uporabljene v magistrski nalogi. Predstavljen bodo tudi osnovne podatkovne strukture in operacije nad njimi, ki bodo uporabljene za implementacijo kompaktne predstavitve priponskega drevesa.

V tretjem poglavju naloge bo podrobno predstavljena podatkovna struktura priponsko drevo (angl. \textit{Suffix Tree} oziroma ST). Poleg predstavitve strukture bo v poglavju prikazana tudi metoda, ki sprotno (angl. \textit{on-line}) gradi priponsko drevo v času $O(n)$ ter implementacija poizvedb nad vhodno besedo s pomočjo priponskega drevesa. Zatem sledi v četrtem poglavju predstavitev alternativnega indeksa besedila, priponsko polje (angl. \textit{Suffix Array} oziroma SA). Predstavljena bo tudi metoda izgradnje priponskega polja v $O(n)$ času ter implementacija poizvedb s priponskim poljem.

%\newpage

V petem poglavju bo predstavljena podatkovna struktura kompaktno priponsko drevo (angl. \textit{Compressed Suffix Tree} oziroma CST), ki omogoča iste operacije kot priponsko drevo, pri tem pa potrebuje manj prostora. V tem poglavju bo tudi predstavljen algoritem za izgradnjo kompaktnega priponskega drevesa, ki skozi celotno izgradnjo ohranja kompaktnost podatkovne strukture. Zatem pa bodo predstavljene implementacije poizvedb nad vhodno besedo s pomočjo kompaktne predstavitve priponskega drevesa.

V šestem poglavju bo izdelana primerjava med različnimi indeksi, ki so bili predhodno predstavljeni. Izdelana bo empirična primerjava različnih indeksov, ki bo merila prostorsko zahtevnost podatkovnih struktur in časovno zahtevnost izgradnje ter poizvedb nad njimi.