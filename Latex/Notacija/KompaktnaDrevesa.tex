Z uporabo bitnih polj je mogoče predstaviti mnogo podatkovnih struktur. Ker pa se magistrska naloga osredotoča na priponska drevesa, bo zato predstavljena uporaba bitnih polj za kompaktno predstavitev topologije dreves. Običajno je podatkovna struktura drevo sestavljena iz vozlišč ter referenc na njih. V vsakem vozlišču je shranjena vrednost, ki jo predstavlja vozlišče, ter referenca na svoje otroke. Drevo z $n$-timi vozlišči potrebuje $O(n\log{n})$ bitov za shraniti topologijo drevesa (v praksi potrebuje $O(nw)$ bitov). Kompaktna predstavitev topologije dreves zniža prostorsko zahtevnost na $2n+o(n)$ bitov. Za vsako vozlišče je potrebno predstaviti referenco na vozlišče ter predstaviti, da je vozlišče bilo že obiskano, zato vsako vozlišče potrebuje 2 bita oziroma za predstavitev celotnega drevesa je potrebnih $2n$ bitov.

Pri tem obstajajo tri vrste kompaktne predstavitve dreves: Uravnoteženi oklepaji (angl. \textit{Balanced Parentheses} oziroma BP), Zaporedje eniških zapisov stopenj vozlišč po plasteh (angl. \textit{Level Order Unary Degree Sequence} oziroma LOUDS) in Zaporedje eniških zapisov stopenj vozlišč v globino (angl. \textit{Depth-First Unary Degree Sequence} oziroma DFUDS). Naslednja definicija prikaže operacije nad drevesi.


\begin{defi}\label{def:drevo}
    Podatkovne struktura drevo mora podpirati naslednje operacije:
    \begin{enumerate}
        \item $\Koren$ vrne koren drevesa,
        \item $\List{v}$ vrne \textit{true}, če je vozlišče $v$ list, sicer pa vrne \textit{false},
        \item $\stOtrok{v}$ vrne število otrok vozlišča $v$,
        \item $\Otrok{v}{i}$ vrne vozlišče $w$, ki je $i$-ti otrok vozlišča $v$,
        \item $\PrviOtrok{v}$ vrne vozlišče $w$, ki je prvi otrok vozlišča $v$,
        \item $\NBrat{v}$ vrne vozlišče $w$, ki je desni (naslednji) brat od vozlišča $v$,
        \item $\PBrat{v}$ vrne vozlišče $w$, ki je levi (predhodni) brat od vozlišča $v$,
        \item $\Stars{v}$ vrne vozlišče $w$, ki je starš od vozlišča $v$,
        \item $\Globina{v}$ vrne število vozlišč na poti iz korena do vozlišča $v$, %(vklučno z \textit{koren}-om in vozliščem $v$)
        \item $\Lca{v}$ vrne najnižjega skupnega prednika od $v$ in $w$.
    \end{enumerate}
\end{defi}


Iz definicije se lahko opazi, da večina operacij  mogoče implementirati zgolj z operacijo \textit{Otrok}$(v,i)$ ter operacijo \textit{starš}$(v)$, s katerimi se lahko sprehajamo po drevesu. S pomočjo operacij iz definicije je možno implementirati ostale operacije na drevesih, ki so bolj specifične za posamično implementacijo drevesa, na primer \texttt{vstavi}, \texttt{izbriši}, \texttt{najmanjši\_element} (v kopici) in ostale.

\subsection{Zaporedje eniških zapisov stopenj vozlišč po plasteh}\label{sec:LOUDS}

Prvi način zapisa topologije drevesa je zaporedje eniških zapisov stopenj vozlišč po plasteh (LOUDS). Ideja zapisa temelji na sprehodu po plasteh po drevesu in zapisu stopenj vozlišč ob njihovem obisku. Pri tem pa se pojavi problem, kako implementirati sprehod po polju stopenj in posledično po drevesu. Potrebni sta dodatni strukturi, in sicer za vsoto stopenj vozlišč do trenutnega vozlišča ter strukturo za štetje vozlišč. Obe strukturi lahko implementiramo kot polji števil. Z uporabo eniškega zapisa, pa se te dve strukturi prevedeta na operaciji $rang$ in $izbira$ nad bitnim poljem z eniškim zapisom.    

Topologijo drevesa se predstavi kot bitno polje $B$ dolžine $2n+1$, pri čemer je $n$ število vozlišč v drevesu. Zapis drevesa se začne z nizom $10$, temu pa sledijo stopnje vsakega vozlišča v eniškem zapisu. Kot je razvidno iz imena predstavitve, so vozlišča obiskana po nivojih: vozlišča z isto globino so zaporedno predstavljena \cite{Navarro2016}.

Primer takega zapisa je predstavljen na Sliki \ref{fig:LOUDS}. Na sliki so vozlišča v zaporedju ločena s sivimi črtami. Koren drevesa je predstavljen z $B[3:7]=11110$. Niz $B[3:7]$ vsebuje 4 bite z vrednostjo $1$, ker ima koren $4$ otroke.

\begin{figure}[htb]
    \begin{center}
        \includesvg{Slike/KokosSTLOUDR.svg}
        \caption{Primer predstavitve drevesa z metodo LOUDS za priponsko drevo besede »KOKOŠ$\$$«.} 
        \label{fig:LOUDS}
    \end{center}
\end{figure}

\begin{table}[htb]
    \centering
    \caption{Implementacija operacij drevesa v LOUDS.}
    \begin{tabular}{r|l}
\textbf{Operacija}& \textbf{Implementacija v LOUDS} \\\hline
         $\Koren$ &         3\\
         $\List{v}$ &       $B[v]==0$\\
         $\stOtrok{v}$&     $\Naslednik{0}{B}{v}-v$\\
         $\Otrok{v}{i}$ &   $\Izbira{0}{B}{\Rang{1}{B}{v-1+1}}+1$\\
         $\PrviOtrok{v}$&   $otrok(v,1)$\\
         $\NBrat{v}$ &      $\Naslednik{0}{B}{v}+1$ \\
         $\PBrat{v}$ &      $\Predhodnik{0}{B}{v-2}+1$ \\
         $\Stars{v}$ &      $\Predhodnik{v}{B}{\Izbira{1}{B}{\Rang{0}{B}{v-1}}}+1$ \\
         $\Globina{v}$ &        / \\
         $\Lca{v}{w}$ &     Algoritem \ref{alg:LOUDSlca}\\

    \end{tabular}
    \label{tab:LOUDSop}
\end{table}

Med izgradnjo kompaktne predstavitve se vozlišča indeksira s števili $i$ med 1 in $n$. Število $i$ predstavlja zaporedno število obiskanega vozlišča, torej koren ima vrednost $1$, prvi otrok korena ima vrednost $2$ in tako dalje. Indeks se uporabi za shranjevanje vrednosti, ki so po navadi shranjene znotraj vozlišča. V Tabeli \ref{tab:LOUDSop} so predstavljene implementacije operacij z uporabo predstavitve drevesa LOUDS. Indek vozlišča se izračuna kot $\Izbira{1}{B}{v}+1$ \cite{Navarro2016}.


Vse operacije razen operacije $globina$ in $lca$ so izvršene v konstantnem času. Pri tem je treba izgraditi podatkovno strukturo zgolj za $izbira_0$, saj operacija $izbira_1$ ni potrebna za pravilno delovanje operacij drevesa. Operacija $globina$ ni podprta, saj LOUDS predstavitev ne omogoča učinkovitega iskanja. Operacija globina je lahko implementirana s štetjem, koliko krat se uporabi operacija $\Stars{v}$ za doseči \textit{koren} drevesa. Pri tem  pa velja, da je $\Globina{u}\ge\Globina{v}$, ko je $u>v$. S pomočjo tega dejstva se lahko implementira operacijo $lca$, kot je prikazano v Algoritmu \ref{alg:LOUDSlca} \cite{Navarro2016}.
 
\begin{algorithm}[hbt]

\Vhod{Bitno polje $B$, vozlišča $v$ in $w$}
\Izhod{Vozlišče $u$}
\caption{Operacija $\Lca{v}{w}$ (LOUDS)}\label{alg:LOUDSlca}
{
    \While{$v\ne w$}{
        \Ce{$v>w$}
            {$v\leftarrow \Stars{v}$}
            \Sicer{$w\leftarrow \Stars{w}$}
    }
    \Vrni{$v$}    
    
}
\end{algorithm}

%\newpage

\subsection{Zaporedje eniških zapisov stopenj vozlišč v globino}\label{sec:DFUDS}

Naslednja predstavitev topologije drevesa je zaporedje eniških zapisov stopenj vozlišč v globino (DFUDS). Ideja predstavitve temelji na premem sprehodu (angl. \textit{Preorder traversal}) po drevesu.  Ko prvič obiščemo, zapišemo njegovo stopnjo. Pri tem lahko opazimo, da so poddrevesa otrokov vozlišča $v$ zapisana v zaporednih celicah polja. Posledično se poddrevo naslednja brat od $v$ začne po skrajno desnem listu v poddrevesu s korenov v $v$. Torej se naslednje poddrevo ne začne, dokler se ne \enquote{zapre} trenutnega poddrevesa. 
\begin{figure}[htb]
    \begin{center}
        \includesvg{Slike/KokosSTDFUDS.svg}
        \caption{Primer predstavitve drevesa z metodo DFUDS za priponsko drevo besede »KOKOŠ$\$$«.} 
        \label{fig:DFUDS}
    \end{center}
\end{figure}

Drevo je predstavljeno z bitnim poljem $B[1:2n+2]$. Zapis temelji na uporabi zapisa stopenj vozlišč, zato je potrebno dodati na začetek bitnega polja $B[1:3]=110$. Zatem pa so zapisane stopnje vozlišč z eniškim zapisom. Zapis je zgrajen z uporabo pregleda v globino, kot to ponazarja ime predstavitve. Vsa vozlišča imajo indeks, ki je zaporedno število obiskanega vozlišča. Predstavitev omogoča preprostejše implementacije operacij ter posledično manjše dodatne podatkovne strukture. Vseeno pa omogoča hitrejše implementacije nekaterih operacij v primerjavi z uporabo predstavitve LOUDS. Primer zapisa topologije drevesa z DFUDS je prikazan na Sliki \ref{fig:DFUDS}. Vozlišča so v zaporedju ločena s sivimi črtami.

\begin{table}[htb]
    \centering
    \caption{Implementacija operacij drevesa z DFUDS}
    \begin{tabular}{r|l}
\textbf{Operacija}& \textbf{Implementacija v DFUDS} \\\hline
         $\Koren$ &          $4$\\
         $\List{v}$ &        $B[v]==0$\\
         $\stOtrok{v}$&      $\Naslednik{0}{B}{v}-v$\\
         $\Otrok{v}{i}$ &    $\Zapri{B}{\Naslednik{0}{B}{v} - i}+1$\\
         $\PrviOtrok{v}$&     $\Naslednik{0}{B}{v}+1$\\
         $\NBrat{v}$ &       $\IskanjeNaprej{B}{v-1}{-1}+1$ \\
         $\PBrat{v}$ &       $\Zapri{B}{\Odpri{B}{v-1}+1}+1$ \\
         $\Stars{v}$ &        $\Predhodnik{0}{B}{\Izbira{1}{B}{v-1}}+1$ \\
         $\Globina{v}$ &         / \\
         $\Lca{v}{w}$ &      $\Stars{\rmq{B}{\Naslednik{0}{B}{w}}{v-1}+1};v<w$\\

    \end{tabular}
    \label{tab:DFUDSop}
\end{table}

V Tabeli \ref{tab:DFUDSop} so predstavljene implementacije operacij nad drevesom, implementirane z DFUDS. Indeks $\Izbira{0}{B}{v}+1$ je uporabljen za shranjevanje dodatnih informacij o vozlišču, saj je vozlišče $v$ predstavljeno s položajem vozlišča v bitnem polju $B$. Operacija $\Zapri{B}{i}$, ki zapre trenutno poddrevo, je definirana z uporabo operacije \textit{višek} tako, da vrne prvi $j>i$, pri čemer je $\Visek{B}{j}=\Visek{B}{i}-1$. Podobno se lahko definira tudi operacijo $\Odpri{B}{i}$, ki vrne koren trenutnega poddrevesa, z uporabo operacije \textit{višek} tako, da vrne zadnji $j<i$, pri čemer $\Visek{B}{j}=\Visek{B}{i}+1$. V tabeli je vidno, da operacija $\Globina{v}$ ni podprta, saj ni mogoče implementirati te operacije brez sprehoda od vozlišča $v$ do korena ter pri tem šteti potrebne korake. Operacija $\Lca{v}{u}$ je lahko implementirana brez sprehoda po drevesu za razliko od implementacije z LOUDS.

Vse operacije, ki temeljijo na operaciji $rang$ in $izbira$, se izvršijo v $O(1)$ času. Operacije, ki temeljijo na $rmM$-drevesu, pa potrebujejo $O(\log{n})$ časa. Pri tem so potrebne dodatne podatkovne strukture za $izbira_1$, $izbira_0$, $rang$ ter $rmM$-drevo, pri čemer vsaka potrebuje $o(n)$ dodatnih bitov. Prostorsko zahtevnost $rmM$-drevesa je mogoče zmanjšati, tako da se shranita zgolj polji $e$ in $m$. Tako se razpolovi prostorsko zahtevnost $rmM$-drevesa, ki pa še vedno zahteva $o(n)$ dodatnih bitov, brez škode za implementacijo operacij drevesa.

Za lažjo implementacijo dodatnih operacij nad listi je mogoče ustvariti dve dodatni podatkovni strukturi za $rang_{00}$ in $izbira_{00}$ v konstantnem času, ki omogočata iskanje in indeksiranje listov v drevesu. Podatkovni strukturi sta implementirani na podoben način kot podatkovni strukturi za $rang_0$ in $izbira_0$, pri tem pa $rang_{00}$ in $izbira_{00}$ temeljijo na številu ponovitev oziroma položaju $i$-te ponovitve dveh zaporednih bitov z vrednostjo 0.

\subsection{Uravnoteženi oklepaji}\label{sec:oklepaji}

Zadnji način zapisa topologije drevesa je zapis z zaporedjem uravnoteženih oklepajev (BP). Drevo se predstavi kot bitno polje $B$ dolžine $2n$, pri čemer je $n$ število vozlišč v drevesu. Zaporedje se zgradi z uporabo pregleda v globino. Ko je vozlišče prvič obiskano, se na konec do sedaj zapisane sekvence zapiše '('. Uklepaj je v bitnem polju predstavljen s številom $0$. Ko se zapiše celotno poddrevo vozlišča, pa se na konec sekvence zapiše ')'. Zaklepaj pa je v bitnem polju zapisan s številom $1$. Primer priponskega drevesa, ki je predstavljen z uporabo zaporedja uravnoteženih oklepajev, je prikazan na Sliki \ref{fig:BP}. Na sliki so vsa notranja vozlišča obarvana z različnimi barvami za lažje prepoznavanje le-teh v zaporedju uravnoteženih oklepajev. Ker so listi oštevilčeni, so ta števila tudi zapisana nad vsakim listom v zaporedju uravnoteženih oklepajev \cite{Navarro2016}.

\begin{figure}[htb]
    \begin{center}
        \includesvg{Slike/KokosSTBP.svg}
        \caption{Primer predstavitve drevesa z metodo BP za priponsko drevo besede »KOKOŠ$\$$«.} 
        \label{fig:BP}
    \end{center}
\end{figure}

Torej je vsako vozlišče predstavljeno kot par '(' in ')'. Tako je mogoče definirati sledeče operacije nad oklepaji: $odpri$, $zapri$ ter $oklepa$. Operacija $\Odpri{B}{i}$ vrne položaj '(', ki odpre ')' na $i$-tem mestu v $B$. Operacija $\Zapri{B}{i}$ pa v tej notaciji predstavlja ')', ki zapira '(' na $i$-tem mestu $B$. Operacija $\Oklepa{B}{i}$ pa vrne položaj $j$ od '(' v bitem polju $B$, pri čemer velja, da $j<i<\Zapri{B}{j}$. Z uporabo operacije \textit{višek} operacija $Oklepa{B}{i}$ vrne položaj največjega $j<i$, pri čemer je $\Visek{B}{j-1}=\Visek{B}{i}-1$ \cite{Navarro2016}.

Podobno kot pri predstavitvi drevesa LOUDS, tudi predstavitev BP potrebuje dodatno podatkovno strukturo za operaciji $izbira$ in $rang$. Pri tem sta potrebni zgolj podatkovni strukturi za $0$, saj  operaciji $rang_1$ in $izbira_1$ nista potrebni za pravilno delovanje drevesa v tej predstavitvi. Podatkovni strukturi potrebujeta vsaka $o(n)$ dodatnih bitov in operaciji se izvršita v konstantnem času.

\begin{table}[htb]
    \centering
      \caption{Implementacija operacij drevesa z BP}
    \begin{tabular}{r|l}
\textbf{Operacija}& \textbf{Implementacija v BP} \\\hline
         $\Koren$ &   1\\
         $\List{v}$ &    $B[v]==0 \wedge B[v+1]==1$\\
         $\stOtrok{v}$&  $\MinStetje{B}{v}{\Zapri{B}{v}-2}$\\
         $\Otrok{v}{i}$ &    $\MinIzbira{B}{v}{\Zapri{B}{v}-2}{i}+1$\\
         $\PrviOtrok{v}$&    $v+1$\\
         $\NBrat{v}$ &      $\Zapri{B}{v}+1$ \\
         $\PBrat{v}$ &      $\Odpri{B}{v-1}$ \\
         $\Stars{v}$ &      $\Oklepa{B}{x}$ \\
         $\Globina{v}$ &    $2\cdot \Rang{0}{B}{v}-v$ \\
         $\Lca{v}{w}$ &     $\Oklepa{B}{\rmq{B}{v}{w}+1};v<w$\\
    \end{tabular}  
    \label{tab:BPop}
\end{table}

V Tabeli \ref{tab:BPop} so prikazane implementacije operacij, ki so potrebne za pravilno delovanje drevesa. Indeks $\Izbira{0}{B}{v}$ je uporabljen za pridobiti dodatne informacije o vozlišču, saj vrednost $v$ predstavljata položaja '(' v zaporedju $B$, ne pa indeksa v tabeli z dodatnimi informacijami o vozlišču.

Zapis topologije drevesa s BP omogoča dodatne operacije. Primer operacije je oštevilčenje in iskanja listov drevesa, kar je storjeno z uporabo posplošitve operacij ranga in izbire na poljubno dolge nize. Ker imajo listi obliko \enquote{()} (oziroma $01$, ko so zapisani v bitnem polju $B$), se lahko implementirata operaciji $\Rang{01}{B}{i}$ in $\Izbira{01}{B}{i}$. Operaciji potrebujeta konstanten čas, da se izvedeta, saj se lahko izgradi podobno dodatno strukturo, kot za osnovno verzijo ranga in izbire. V BP ni mogoče izbrisati vrednosti $M$ in $n$ iz listov $rmM$-drevesa, saj se uporablja poizvedbo $\rMq{B}{i}{j}$, ki potrebuje vrednost $M$ pri implementaciji dodatnih operaciji. Primer take operacije je \textit{najglobljeVozlišče}$(v)$, ki vrne najgloblje vozlišče v poddrevesu s korenom $v$ \cite{Navarro2016}.