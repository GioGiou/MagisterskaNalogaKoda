%UVODNO BESEDILO

V tem poglavju bodo predstavljene osnovne oznake in definicije, ki bodo uporabljene skozi magistrsko nalogo. Za tem pa bodo predstavljene osnovne podatkovne strukture in predstavitve podatkovnih struktur, ki so potreben za implementacijo kompaktne predstavitve priponskih dreves.

%V nadaljevanju poglavja bodo predstavljena še predstavitve in podatkovne strukture ter operacije nad njimi, ki so osnova za kompaktno predstavitev drugih podatkovnih struktur (na primer kompaktna prdstavitev dreves), ter kompaktne predstavitve dreves, pri čemer bo ena od predstavitev uporabljena v podatkovni strukturi kompaktno priponsko drevo. Predstavlje bo tudi eniški zapis, ki bo uporabljen v kompaktni predstavitvi priponskih dreves.

\section{OZNAKE IN OSNOVNE DEFINICIJE}\label{sec:def}
Skozi magistrsko nalogo bo $\Sigma$ predstavljala abecedo. Abeceda je množica znakov $\Sigma=\{\sigma_1,\dots,\sigma_a\}$. Vhodna beseda $T$ dolžine $n$ znakov, iz katere bomo zgradili priponsko drevo, priponsko polje ali kompaktno priponsko drevo, je sestavljena iz znakov abecede $\Sigma$, torej je $T=\Sigma^n$. Na podoben način $P$ predstavlja iskani vzorec dolžine $m$ znakov, katerega se želi preveriti prisotnost v $T$. Tako kot beseda $T$ je tudi vzorec $P$ sestavljen iz znakov abecede $\Sigma$, torej je $P=\Sigma^m$.

Niz znakov iz abecede $\Sigma$, ki je del besede in se začne na $i$-tem in se konča na $j$-tem znaku besede, imenujemo podniz in ga označimo kot $T[i:j]$. Posebna vrsta podniza, ki se začne na $i$-tem znaku in konča na koncu besedila, imenujemo pripona ter se jo označi kot $T[i:n]$ oziroma $T[i:]$. Poljubni nizi oziroma podnizi so označeni z grškimi črkami $\alpha$, $\beta$ in $\gamma$ ter pri tem je dopisano, ali gre za podniz ali niz. Formalno je podniz definiran na sledeč način:
\begin{defi}
    Niz $\alpha\in\Sigma^*$ je podniz besede $T$, ko obstaja tak $1\le i\le n$, za katerega velja, da je  $\alpha=T[i:i+|\alpha|]$. 
\end{defi}
Dva poljubna niza se lahko združita v novi daljši niz s stikom, ki je definiran na sledeči način:
\begin{defi}
    Stik nizov $\alpha$ in $\beta$ je niz $\gamma=\alpha\cdot\beta=\alpha\beta$, pri čemer je $\alpha=\gamma[1:|\alpha|]$ in $\beta=\gamma[|\alpha|+1:|\alpha|+|\beta|]$.
\end{defi}

Za razliko od nizov in podnizov so znaki iz abecede $\Sigma$ označeni s črkama $x$ ali $t$. Oznaka $T[i]$ pa predstavlja znak na $i$-tem mestu vhodne besede.

Ker bo nad besedo $T$ izgrajeno priponsko drevo ali kompaktno priponsko drevo, ki je uporabljeno za hitrejše iskanje vzorcev v besedi, so skozi nalogo vozlišča označena s črkami $s$, $v$, $w$ in $u$. Listi priponskega drevesa pa bodo označeni s črko $l$, ko se govori na splošno o listih, sicer pa bodo označeni z $l_i$, pri čemer $i$ označuje zaporedno število lista od leve proti desni.

Skozi nalogo bodo $\log_2{n}$ označeni kot $\log{n}$. Če zapisan logaritem ni dvojiški logaritem, bo osnova le tega označena.


\section{ENIŠKI ZAPIS}\label{sec:Eniski_Zapis}
\import{.}{Notacija/EniskiZapis}

\section{MODEL RAČUNANJA}\label{sec:Model_racunanja}
Preden se lahko predstavi način shranjevanja bitnega polja, je potrebno predstaviti uporabljen model računanja. Splošen model računanja bo uporabljen, saj se različne arhitekture mikro procesorjev razlikujejo med seboj in bi bilo potreno narediti analizo za vsako arhitekuro posebaj. Zato bo izbran model računanja Računalnik z naključnim dostopom (angl. \textit{random-access machine} oziroma RAM). Bolj natančno bo uporabljen besedni RAM  (angl. \textit{word RAM}).

Osnovna različica RAM modela podpira, da so vse aritmetične operacije nad celimi števili in logične operacijami nad biti izvedene v konstantnem času. Poleg tega RAM omogoča dostopa do pomnilnika v konstantnem času ter linearni čas za dodelitev zaporednega spomina. Vse te operacije so storjene v enakem času tudi v besednem RAM modelu. Edina razlika med modeloma je, da RAM predpostavlja neskončno velik pomnilnik za razliko od besednega RAM, ki pa omeji velikost pomnilnika na $U$ naslovov in posledično je velikost ene pomnilniške besede na $w=\log{U}$ bitov, kar je velikost enega naslova \cite{Fredman1990,Morin2013,Navarro2016}. Pri tem se lahko definira tudi \enquote{cela števila} (angl. \textit{Integer}). \enquote{Cela števila} so definirana kot podmnožica celih števil, ki so lahko predstavljena v dvojiškem zapisu z eno računalniško besedo \cite{Navarro2016}.

\section{BITNO POLJE}\label{sec:Bitno_Polje}
\import{.}{Notacija/BitniVektor}

%\section{ENIŠKI ZAPIS}\label{sec:Eniski_Zapis}
%\import{.}{Notacija/EniskiZapis}

\section{KOMPAKTNA PREDSTAVITEV DREVES}\label{sec:kompaktna_drevesa}
\import{.}{Notacija/KompaktnaDrevesa}