%Preden se lahko bitno polje uporabi za kompaktno predstavitev dreves, je potrebno predstaviti še eniški zapis. Eniški zapis je uporabljen v dveh kompaktnih predstavitvah dreves. 
Kot je iz samega imena zapisa razvidno, bodo števila predstavljena v eniškem sistemu. Torej bo nenegativno celo število $s$ predstavljeno kot $1^s0$, pričemer je abeceda $\Sigma=\{0,1\}$.s $s$-timi enicami ter števila so med seboj ločena z ničlo. Na primer, seznam števil $[1,3,0,4,1]$ bi bil v eniškem zapisu predstavljen kot niz \texttt{10111001111010}.

Eniški zapis bo uporabljen pri kompaktni predstavitvi dreves. Razlog za uporabo eniškega zapisa števil namesto dvojiškega zapisa števil je lažja pretvorba operacij nad drevesi na operacije nad bitimi polji, saj imajo na ta način enice in ničle vsaka svoj pomen. Ker je zapis uporabljen za predstavitev stopenj vozlišč, potem predstavlja zaporedje $1^*0$ eno vozlišče. Vsaka enica v vozlišču predstavlja otroka vozišča, ničla pa predstavlja, da je bilo vozlišče že obiskano v zapisu.

