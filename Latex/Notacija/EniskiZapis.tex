Preden se lahko bitno polje uporabi za kompaktno predstavitev dreves, je potreno predstaviti še eniški zapis. Eniški zapis je uporabljen v dveh komapktnih predstavitvah dreves. Kot je iz samega imena zapisa zavidno, bodo števila predstavljena v eniškem sistemu. Torej bo vsako ne negativno celeo število $s$ predstavljen z $s$-timi enicami ter števila so med seboj ločena z ničlo. Na primer seznam števil $[1,3]$ bi bil v eniškem zapisu predstavljen kot $101110$.

Razlog za uporabo eniškega zapisa števil namesto dvojiškega zapisa števil v kompaktni predstavitvi dreves, je lažja pretvorba operacij nad drevesi na operacije nad bitimi polji, saj na ta način imajo enice in ničle vsaka svoj pomen. Ker je zapis uporabljen za predstavitev stopen vozlišč, potem predtsavlja zaporedjen $1^*0$ eno vozlišče. Vsaka enica v vozlišču predstavlja otroka vozišča, ničla pa predstavlja, da je bilo vozlišče že obiskano v zapisu.

V naslednjem podpoglavju bo predstavljena uporaba eniškega zapisa za kompaktno predstavitev drevesa. Eniški zapis se uporabi v kompaktnih predsatvitvah imenovanih Zaporedje eniških zapisov stopenj vozlišč po plasteh in Zaporedje eniških zapisov stopenj vozlišč v globino.
