Bitno polje (angl. \textit{bit vector}) $B$ dolžine $n=|B|$ kot samo ime podtkovne strukture, je polje, v katerem je shranjinh $n$ bitov, do katerih se želi dostopati v konstantnem času. Zato sega lahko uporabi kot osnovna podatkovna struktura za kompaktne predstavitve. Primer take podatkovne strukture, ki uporabja bitno polje za svojo kompaktno predstavitev, je drevo.

V nadajevanju tega podpogavja bo predstavljeno, kako učinkovito shraniti bitno polje ter ohraniti dostop do podatkov v konstantem času. Predstavljene bodo še druge operacije nad bitnimi polji, ki so uporabne za reševanje problemov. predstavljene operacije so: osnovni operaciji $rang(B,i)$ (angl. \textit{rank}) in $izbira(B,i)$ (angl. \textit{select}), sestavljen operacij $predhodnik(B,y)$ in $naslednik(B,y)$ ter dodatne operacije nad odseki bitenega polja (angl. \textit{range querry}). Vse predstavljene operacije želimo opraviti čim bolj učinkovito ter pri tem uporabiti čim manj prostora za storiti le to.

%
    %Najbolj osnovna podatkovna struktura, ki bo tudi osnova za vse ostale kompaktne predstavitve podatkovnih struktur, je bitno polje (angl. \textit{bit vector}). Kot samo ime podatkovne strukture pove, bitno polje $B$ dolžine $n$ je polje, v katerem je shranjeni $n$ bitov, do katerih želimo dostopati v konstantnem času. Poleg dostopa do bitov, bitno polje podpira še dve osnovni operaciji: $rang_v(B,i)$ (angl. \textit{rank}) in $izbira_v(B,i)$ (angl. \textit{select}).
    %
    %je kompaktna podatkovna struktura, ki je osnova za ostale kompaktne podatkovne strukture. Bitna polja so osnova za kompaktno predstavitev topologijo drevesa. Univerzalna množica vseh bitnih polj dolžine $n$ je velikosti $\left |N\right |=2^n$, torej bitno polje ima prostorsko zahtevnost $n+o(n)$ bitov. Pri tem je potrebnih $n$ bitov za shraniti celotno bitno polje ter dodatnih $o(n)$ bitov za implementacijo operacij nad njim. Bitno polje $B$ podpira tri operacije:  ter $dostop(B,i)$ (angl. \textit{access}), ki vrne $i$-ti bit bitnega polja $B$.

\paragraph{Model računaja in dostop:}
Preden se lahko predstavi način shranjevanja bitnega polja, je potrebno predstaviti uporabljen model računanja. Splošen model računanja bo uporabljen, saj se različne arhitekture mikro procesorjev razlikujejo med seboj in bi bilo potreno narediti analizo za vsako arhitekuro posebaj. Zato bo izbran model računanja Računalnik z naključnim dostopom (angl. \textit{random-access machine} oziroma RAM). Bolj natančno bo uporabljen zlogovni RAM  (angl. \textit{word RAM}).

OSnovna različica RAM modela podpira vse aritmetične operacije nad celimi števili in logične operacijami nad biti v konstantnem času. Poleg tega RAM omogoča konstanten čas dostopa do spomina ter linearni čas za dodelitev zaporednega spomina. Vse te operacije so storjene v enakem času tudi v zlogovnem RAM modelu. Edina razlika med modeloma je, da RAM prededpostavlja neskončni spomin zlogovni RAM pa omeji velikost spomina na $U$ nalovov in posledično je velikost enega zloga $w=\log_2 U$\footnote{v nadaljevanju bo $\log_2 n$ zapisan kot $\log n$ } bitov, kar je velikost enega naslova \cite{Fredman1990,Morin2013}.


Uporaba Računalnika z naključnim dostopom (angl. \textit{random-access machine} oziroma RAM) (točen model word RAM) modela
    
    -O(1) časa za izvesti vse osnovne aritmetične operacije

    - omejen prostor na U naslovov, ki hranijo $w=\log_2{U}$ bitov prostora (velikost enega nalova)  U je strogo večji od $n$ (velikosti podatkovne strukture), saj sicer jo nemoremo shraniti

name RAM modela je teoretično simulirati fizične računalnike.

\cite{Fredman1990,Morin2013}


Bitno polje je shranjeno kot polje celih števil $W\left[1,\left\lceil\frac{n}{w}\right\rceil\right]$ \cite{Navarro2016}.

Torej dostop do $i$-tega bita v bitem polju $B$, ki je označen kot $\textit{dostop}(B,i)=B[i]$, je izračunan kot:
$$
    B[i]=\left\lfloor W\left[\left\lceil\frac{i}{w}\right\rceil\right]/2^{w-r}\right\rfloor \mod{2},
$$
pri čemer $r$ je točna pozicija $i$-ja znotraj celega števila oziroma $r=((i-1)\mod{w}) +1$. Torej $2^{w-r}$ predstavlaj $w-r$ bitnih premikov (angl. \textit{bit shift})  v desno.

RAM

\paragraph{Rang:}

\begin{defi}
    Operacija $rang_v(B,i)$ vrne število pojav vrednosti $v\in\{1,0\}$ v bitnem vektorju $B$ do položaja $i$.
\end{defi}
 
V nadaljevanju bo operacija $rang(B,i)$ predstavljala $rang_1(B,i)$. To je možno, saj velja sledeča relacija med $rang_0(B,i)$ in $rang_1(B,i)$:
\begin{equation}\label{eq:rang}
    rang_1(B,i)=i-rang_0(B,i).
\end{equation}
Operacija rang ima časovno zahtevnost $O(1)$, pri tem pa potrebuje dodatno podatkovno strukturo, ki zasede $o(n)$ bitov. V nasprotnem primeru bitno polje ni več kompaktna podatkovna struktura. Relacija iz enakosti \ref{eq:rang} omogoča izgradnjo podatkovne strukture zgolj za operacijo $rang_1$.

Pomožna struktura, ki omogoča konstantni čas operacije $rang(B,i)$, shranjuje range različnih elementov bitnega polja $B$ v polju $R$. Naivni pristop shrani v polju $R$ vrednosti vseh elementov $B$-ja. Problem tega pristopa je prevelika prostorska zahtevnost, saj $R$ ni več velikosti $o(n)$ bitov. Rešitev tega problema je vzorčenje rangov, tako da v polju $R$ so shranjeni zgolj rangi nekaterih elementov $B$-ja. Polje $R$ razdeli $B$ na $s=kw$ delov, pri čemer je $k$ poljubno število ter je $w$ dolžina računalniške besede. Element $R[i]=rang_1(B,is)$, pri čemer $0\le i \le \left\lfloor\frac{n}{s} \right\rfloor$. Na ta način je operacija rang implementiran kot 
\begin{equation}
    rang_1(B,i)=R\left[\left\lfloor\frac{i}{s} \right\rfloor\right] + \texttt{popcount}\left(B\left[\left\lfloor\frac{i}{s} \right\rfloor s,i\right]\right),
\end{equation}
kjer je \texttt{popcount} funkcija, ki prešteje število bitov z vrednostjo $1$. Zaradi uporabnosti te funkcije je le ta implementirana v različnih modernih arhitekturah procesorjev. Polje $R$ ima velikost $\lfloor\frac{n}{k}\rfloor$ bitov, saj shranjuje $\lfloor\frac{n}{s}\rfloor$ števil dolžine $w$. Pri tem je potrebno funkcijo \texttt{popcount} pognati največ $k$ krat, kar naredi čas operacije rang $O(k)$ \cite{Navarro2016}.

Podobno kot polje $R$ se definira polje $R'$. Polje $R'$ hrani na $i$-tem mestu število bitov z vrednostjo $1$ po vedru $R[\left\lfloor\frac{i}{k} \right\rfloor]$ ali $R'[i]=rang_1(B,iw)-R[\left\lfloor\frac{i}{k} \right\rfloor]$. Na ta način se zniža število klicev funkcije \texttt{popcount} na 1 klic. Ker v $R'$ so sharnjena zgolj števila med 0 in $s-w$ (vsako vedro v $R$ ima $w$ elementov v $R'$), se lahko $R'$ shrani v $\lfloor\frac{n}{w}\rfloor\log{s}$ bitov, kar je $o(n)$ bitov. Operacija rang je zato implementirana s pomočjo $R$ in $R'$ na sledeči način:
\begin{equation}
    rang_1(B,i)=R\left[\left\lfloor\frac{i}{kw} \right\rfloor\right]+ R'\left[\left\lfloor\frac{i}{w} \right\rfloor\right]+\texttt{popcount}\left(B\left[\left\lfloor\frac{i}{w} \right\rfloor,\left\lfloor\frac{i}{w} \right\rfloor+(i \mod w)\right]\right).
\end{equation}

V praksi se izkaže, da se najboljše rezultate pridobi, ko je $k=8$ za računalniško besedo dolžine $w=64$ (uporabljena v vseh modernih procesorjih). To omogoča shranjevanje vedra v $R$ in vseh pripadajočih veder v $R'$ v dveh besedah, ker prva beseda shrani vrednost vedra v $R$ druga pa vse vrednosti pripadajočih veder v $R'$. Ta implementacija potrebuje $1,25*n$ bitov za vse tri podatkovne strukture skupaj \cite{Navarro2016}.

\paragraph{Izbira:}
Podobno kot operacija rang, tudi operacija izbira potrebuje pomožno podatkovno strukturo za izvedbo v konstantnem času. 
\begin{defi}
    Operacija $izbira_v(B,i)$ vrne položaj $i$-tega pojava vrednosti $v\in\{1,0\}$ v bitnem vektorju $B$.
\end{defi}
Operacijo izbira si lahko predstavimo, kot inverzno operacijo od operacije rang, saj velja relacija $j=rang_v(B,izbira_v(B,j))$. Pri tem pa ne obstaja povezava med operacijo $izbira_1(B,i)$ in $izbira_0(B,i)$. To pomeni, da rešitev z pomožno podatkovno strukturo za $izbira_1(B,i)$ ne more biti uporabljena pri iskanju rešitve za $izbira_0(B,i)$. V nadaljevanju bo opisan postopek iskanja $izbire_1(B,i)$, saj se lahko $izbira_0(B,i)$ implementira na podoben način \cite{Navarro2016}.

Ko operacija izbira ni ključna pri reševanju problemov, se lahko uporabi binarno iskanje v poljih $R$ in $R'$, ki potrebuje $O(\log{n})$ časa. Z binarnim iskanjem se najde območje dolžine $k$, pri čemer je potrebno še dodatnih $k$ preiskav, da se najde natančno vrednost.
Čas operacije se lahko zniža na $O(\log\log n)$ z uporabo pomožnih podatkovnih struktur. Podobno kot pri operacij rang se bitno polje razdeli na $\lceil \frac{m}{s} \rceil$ veder, pri čemer $m$ je število bitov z vrednostjo 1 v bitnem polju in s je število takih bitov v vsakem vedru. Polje $S[0,\lceil \frac{m}{s} \rceil]$ hrani vrednosti $S[i]=izbira_1(B,i*s+1)$, pri čemer $S[\lceil \frac{m}{s} \rceil]=n+1$. Polje $S$ potrebuje $w(\lceil \frac{m}{s} \rceil +1)$ bitov. Ko je $s=w^2$, potem podatkovna struktura potrebuje $\lceil\frac{m}{w^2}\rceil=o(m)=o(n)$ bitov \cite{Navarro2016}.

Vedra niso enako velika, zato se lahko razdelijo na velika vedra in mala vedra, pri čemer je vedro veliko natanko tedaj, ko je večje kot $s=\log^2 n$ bitov, sicer je malo vedro. Pri tem je potrebno shraniti velikost vedra v bitno polje $V$, kjer $V[i]=1$, če $i$-to vedro je veliko, sicer  $V[i]=0$. Bitno polje $V$ potrebuje tudi dodatno podatkovno strukturo za rang. Za velika vedra se izračuna vseh $s$ vrednosti izbire, pri čemer so shranjeni v polju $I$ položaji bitov z vrednostjo ena znotraj vedra. Za mala vedra, pa se sproti naračuna izbira znotraj vedra. Operacija izbira se izračuna na sledeči način:
\begin{equation}
    izbira_1(B,j)=\left\{
    \begin{array}{rl}
       S\left[\left\lceil \frac{j}{s} -1\right\rceil\right] + I\left[rang_1(V,\left\lceil \frac{j}{s} \right\rceil)s+x\right], & V\left[\left\lceil \frac{j}{s} \right\rceil\right] = 1\\ 
       S\left[\left\lceil \frac{j}{s} -1 \right\rceil\right] + k, & V\left[\left\lceil \frac{j}{s} \right\rceil\right] = 0\\
       n+1, & m < j
    \end{array}\right.,
\end{equation}
kjer $x$ predstavlja $((j-1) \mod{s})+1$ in $k$ je položaj $((j-1) \mod{s})+1$-tega bita z vrednostjo 1 na intervalu $B\left[ S\left[\left\lceil \frac{j}{s} -1 \right\rceil\right], S\left[\left\lceil \frac{j}{s} \right\rceil-1\right]\right]$.

Polje $I$ mora shraniti $s\lceil\log n\rceil$ bitov za vsako veliko vedro, katerih je $\frac{n}{s(\log n)^2}$, torej potrebuje $O\left(\left\lceil\frac{n}{\log n}\right\rceil\right)=o(n)$ bitov. Bitno polje $V$ pa potrebuje $\lceil \frac{m}{s} \rceil$ bitov za shraniti velikosti blokov ter dodatne bite za izvajanje operacije rang v konstantnem času, torej tudi $V$ potrebuje $o(n)$ bitov \cite{Navarro2016}.

Operacija se lahko izvedene v konstantnem času. To je doseženo z dodatnim deljenjem majhnih veder, na podoben način, kot je bilo to storjeno nad celotnim bitnim poljem $B$. Vsako majhno vedro se dodatno razdeli na $s'=(\log\log n)^2$ mini veder, pri čemer mini vedro je veliko, ko je večje od $s'(\log\log n)^2$. Vsako majhno mini vedro potrebuje $s'\log{(s(\log{n})^2)}=O((\log\log n)^3)=o(n)$ bitov ter vsako veliko mini vedro potrebuje isto prostora, pri čemer pa jih je največ $O\left(\frac{n}{(\log\log n)^4}\right)$, torej vsa velika mini vedra skupaj potrebujejo $O\left(\frac{n}{\log\log n}\right) =o(n)$ bitov \cite{Navarro2016}.


Vse dodatne podatkovne strukture potrebne za izvajanje operacij rang in izbira v konstantnem času, so lahko izgrajene v dveh sprehodih po bitnem polj $B$, pri čemer vsak sprehod traja $O(n)$ časa. Pri tem je ves potrebni spomin že predhodno dodeljen.

\paragraph{Predhodnik/Naslednjik:}

Operacija predhodnik elementa $y$ najde element $x_i$ del zaporedja, za katerega velja $x_i \le y < x_{i+1}$ in $1\le i\le m$, kjer je $m$ število ponovitev $v$-ja v $B$. Operacija je implementirana kot
    $$predhodnik_v(B,y)=izbira_v(B,rang_v(B,y)),$$
pri čemer je $v\in \{1,0\}$. Na podoben način je definirana tudi operacija naslednik. Operacija naslednik elementa $y$  najde položaj elementa $x_i$ v zaporedju $x_1\le x_2 \le \dots \le x_m$, pri čemer velja, da $x_{i-1}< y \le x_i$. Pri tem je implementirana kot
    $$naslednik_v(B,y)=izbira_v(B,rang_v(B,y-1)+1),$$
kjer je  $v\in \{1,0\}$. Zaporedje $x_1\le x_2 \le \dots \le x_n$ predstavlja vozlišča v zaporedju eniških zapisov stopenj vozlišč po plasteh.

\paragraph{Range querry:}

Zaporedje oklepajev se lahko definira višek

Operacija $vi$\textit{š}$ek(B,i)$ vrne število '(', ki niso bili še zaprti in je
$$
    \textit{višek}(B,i)=rang_0(B,i)-rang_1(B,i)=2rang_0(B,i)-i.
$$

Višek

operacije nad odseki bitnega polja:

Te operacije so sledeče: $rmq$, $rMq$, $minizbira$ in $min$\textit{š}$tetje$. Operacija $rmq(B,i,j)$ (angl. \textit{range minimum query}) vrne položaj $k$, kjer je $i\le k\le j$, $vi$\textit{š}$ek(B,k)$ je najnižji višek v območju med $i$ in $j$ ter $k$ je prvi pojav najnižjega viška na tem območju. Podobno je definirana tudi operacija $rMq(B,i,j)$ (angl. \textit{range maximum query}), ki pa vrne položaj $k$, pri čemer je $i\le k\le j$ in $vi$\textit{š}$ek(B,k)$ je najvišji višek v območju med $i$ in $j$ ter $k$ je prvi pojav najvišjega viška na območju. Operacija $minizbira(B,i,j,t)$ vrne položaj $t$-tega pojava najmanjšega viška na območju med $i$ in $j$. Operacija $min$\textit{š}$tetje(B,i,j)$ pa vrne število pojav najnižjega viška na območju med $i$ in $j$ \cite{Navarro2016}.

Vse predstavljene operacije (razen operacije $vi$\textit{š}$ek$, ki potrebuje $O(1)$ časa za izvajanje) se lahko implementira s časovno zahtevnostjo $O(\log{n})$. Pri tem pa je potrebno zgraditi dodatno podatkovno strukturo rmM-drevo (angl. \textit{range minumum maximum tree}), ki je lahko shranjeno z $O(n/\log{n})=o(n)$ dodatnimi biti. Podatkovna struktura razdeli bitno polje $B$ na $\frac{n}{b}$ veder velikosti $b$ ter za vsako vedro se ustvari list rmM-drevesa. Drevo je levo poravnano in se lahko zapiše kot polje vozlišč (podobno kot podatkovna struktura kopica). Torej so otroci $i$-tega vozlišča na $2i$-tem in $2i+1$-vem mestu v polju ter starš $i$-tega vozlišča se nahaja na $\left\lfloor\frac{i}{2}\right\rfloor$-mestu. Vsako vozlišče v drevesu hrani štiri podatke: $e$ razlika viška med začetkom in koncem pokritega območja, $m$ najmanjši relativni višek v območju, $M$ največji relativni višek na območju in $n$ število najmanjših viškov na območju, ki ga pokriva vozlišče. Pri tem vozlišče pokriva celotno območje, ki ga pokrivata oba otroka, in listi pokrivajo zgolj eno vedro dolžine $b$. Torej koren drevesa pokriva celotno bitno polje $B$. Vse predstavljene operacije so implementirane s pomočjo sprehoda po rmM-drevesu 
\cite{Navarro2016}.

rmM drevo